\begin{tikzpicture}

% Define grid dimensions, 10x8 grid
\def\qubits{7} 
\def\timesteps{6}
\def\xspace{.8}
\def\qubitss{\qubits*\xspace}

% Draw the background grid
\foreach \x in {0,...,\qubits} {
  \draw[black, thick] (\x*\xspace,0.25) -- (\x*\xspace,\timesteps+.9);
  %\fill[black] (\x,-.75) circle (.2);
}
\foreach \t in {0,...,\timesteps} {
  \node[] at (-1.5,\t+.5) {$t=\t$};
}

%\draw[red, very thick, fill=red!20] (0,\timesteps+1.2) circle (.15);
%\draw[arrow, thick] (-1,-1) -- (-1,\timesteps+1.5) node[anchor=west] {\LARGE$t$}; 

\foreach \t in {1,...,\timesteps} {
    \foreach \x in {0,...,\qubits} {
        % Check if we're on an even or odd row
        \pgfmathparse{int(mod(\t-1,2))}
        \let\rowtype=\pgfmathresult
        % Check if we're on an even or odd column
        \pgfmathparse{int(mod(\x,2))}
        \let\coltype=\pgfmathresult
        
        % Place rectangle if row and column types match
        \ifnum\rowtype=\coltype
            % If we're at the last column and need a rectangle
            \ifnum\x=\qubits
                % Right half
                \fill[petrol!60] (\qubitss-0.2,\t-0.2) rectangle (\qubitss+0.3,\t+0.2);
                \draw[black,very thick] (\qubitss-0.2,\t-0.2) -- (\qubitss-0.2,\t+0.2);  % Right border 
                \draw[black,very thick] (\qubitss-.22208,\t-0.2) -- (\qubitss+0.3,\t-0.2);      % bottom border
                \draw[black,very thick] (\qubitss-.22208,\t+0.2) -- (\qubitss+0.3,\t+0.2);      % top border
                
                % Left half - no border on right side
                \fill[petrol!60] (-0.3,\t-0.2) rectangle (0.2,\t+0.2);
                \draw[black,very thick] (0.2,\t-0.2) -- (0.2,\t+0.2);                % left border
                \draw[black,very thick] (-0.3,\t-0.2) -- (.22208,\t-0.2);                   % bottom border
                \draw[black,very thick] (-0.3,\t+0.2) -- (.22208,\t+0.2);                   % top border
              \else
                % Normal rectangle for non-boundary positions
                \draw[black,very thick, fill=petrol!60] (\xspace*\x-.2,\t-0.2) rectangle
                (\xspace*\x+1,\t+0.2);
            \fi
            %\draw[blue, fill=blue!30] (\x+0.1,\t-0.15) rectangle (\x+0.9,\t+0.15);
        \fi
    }
  }

\def\loopend{6}
% Place red circles on vertical edges with 0.5 probability
\foreach \x in {0,...,\qubits} {
    \foreach \t in {0,...,\loopend} {
        \pgfmathrandominteger{\randnum}{0}{2}
        \ifnum\randnum=1
            \draw[black, very thick, fill=white] (\x*\xspace,\t+0.5) circle (0.1);
        \fi
    }
  }

  \draw[black, thick] (\qubits,4) -- (\qubits,5);
  \draw[black, thick] (\qubits+.75,4) -- (\qubits+.75,5);
  \draw[black, thick, fill=petrol!60] (\qubits-.15,4.3) rectangle
  (\qubits+.9,4.7);
  \node[anchor=west] at (\qubits+1,4.5) {random unitary};

  \draw[black, thick] (\qubits+.25,2) -- (\qubits+.25,3);
  \draw[black, thick, fill=white] (\qubits+.25,2.5) circle (.08);
  \node[anchor=west] at (\qubits+1, 2.55) {local measurement};

\end{tikzpicture}
