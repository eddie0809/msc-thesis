\chapter{Fragments and Tangents}
\label{ch:fragments}

\section{On tangents}
While writing this thesis my thoughts have wandered off once or twice. Almost
always while writing out something completely different. Incidentally, this is
how the paragraph you're reading right now came to be.

Sometimes I'd write down something that amuses me, sometimes I'd get frustrated
and investigate my frustrations in text. Maybe I had even had a profound
thought once in a while.  Regardless of how these tangents got there, or what
they are about, off-topic tangents are not something you'll want to put in the
main body of your thesis, no matter how ingqnious or witty they are.  Come to
think of it, they probably do not belong in the appendix either.

Either way, I would have been quite distraught to delete these paragraphs
without any of them seeing the inside of a printer. So, in order for them not
to go to waste, I decided to collect them here for mine and my friends'
amusement. These texts really are silly sometimes.

\section{On references}
I like how references are provided in physics: all those little numbers in
little brackets or as superscript above a claim to support it. It allows me to
follow the bibliography in parallel to the paper and check some pertinent
publications easily. This is, of course, very different in many other
disciplines, e.g. psychology or philosophy, where it is common-practice to give
your sources in alphabetical order sorted by the author's surname.

I claim that physicists are not appreciative enough of this. It will never fail
to frustrate me if people cite some shit that has nothing at all to do with the
claim in question. This is especially so if the claim is better supported by
the results of other papers; papers that a lot of times make it into the
\texttt{.bib} file of at least one of the authors, or why else would they be
prominently cited in a section, where it doesnt make sense to do so.
Ultimately, the highest degree of frustration for me is if the references are
off by one, or cite a previous project by the same authors. This kind of
situation lies in the uncanny valley of "is it them or me?". And more often
than not, it should be \emph{me}, right? But then you go on and on, read the
cited papers thoroughly, try to understand the context in which it is
referenced in the work you originally went through, and now, multiple layers
deep in the reference rabbithole are you ultimately forced to realize that it's
not you, it's them. 

