\chapter{Classical \ Simulation \ of \ Stabilizer \ Circuits}
\label{ch:mixed}
\epigraph{Always programming a new type of antidote in your perimeter}{
\citeauthor{quasimotoDiscipline99Pt2000}
%on \citefield{quasimotoDiscipline99Pt2000}{title}
}

In this chapter we present the simulation algorithm and 
the modifications to the existing infrastructure (cf. ref.
\cite{langCliffordCircuitSimulator2022}) that
were necessary for the numerical experiments of this thesis. Usually, these
would fall into the \emph{Methods} sections of the respective
chapters. However, the modifications were extensive enough to warrant the
dedication of an entire chapter to them. We will first briefly introduce the
simulation algorithm based on the results from Gottesman and Knill
\cite{gottesmanHeisenbergRepresentationQuantum1998}, and Gottesman and Aaronson
\cite{aaronsonImprovedSimulationStabilizer2004}. Then, we briefly present
the functions
necessary for computing the linear cross entropy $\chi$, from \cref{ch:lxe}, as
this was only a small addition. We
then finish off with the larger discussion of mixed states in the stabilizer
formalism and the realizations of the mixed state simulation algorithms, as
well as the functions for the various entropic quantities, implementing our
results from \cref{sec:rel-ent-stab}.

%Ref. \cite{aaronsonImprovedSimulationStabilizer2004} will need to do some heavy
%lifting in giving primers on stabilizer simulation.
%
%It will later prove useful to implement mixed states into the existing
%stabilizer simulator. As of yet, this does not exist.
%
%In this paper \cite{audenaertEntanglementMixedStabilizer2005} they present an
%algorithm to perform a partial trace on a stabilizer state. They need an
%auxilliary algorithm, the Reduced row echelon form algorithm. I think this is a
%promising approach, but needs implementation of the RREF and the ptrace
%algorithm. From what I've seen, the first one is going to take longer than the
%second to implement. Maybe look for this kind of alg in the existing codebase?
%
%Note from 10 minutes later: its basically what we already had, only a bit more
%formalized. 
%
%Lecture notes on stabilizers: \cite{arabLectureNotesQuantum2024}
%
%This chapter serves as a primer on the nitty-gritty details of simulating
%stabilizer circuits on a classical computer. As already touched on in
%\cref{ch:basics}, according to the \emph{Gottesman-Knill Theorem} we can
%perform efficient classical simulations of quantum circuits using only Clifford
%gates. In the following sections we will introduce the so-called tableau
%algorithm, which allows simulations in polynomial time, and expand it to
%properly deal with mixed states. We assume the reader to be familiar with the
%corresponding sections in \cref{sec:stab-basics}.

\subsubsection*{A computer science primer}
%%% WAS HIER URSPRÜNGLICH MAL STAND:
% This chapter addresses classical computation and algorithms, focussing on the
% simulation of a particular class of quantum circuits using classical computers.
% As such, it will necessarily feature terminology commonly associated with the
% field of computer science. While this thesis as a whole is a work on natural
% science – i.e. theoretical physics – our focus lies more on simulations as a
% tool rather than as the primary object of study. Nonetheless, to ensure
% clarity, we will briefly introduce key terminology from computer science.
%%%
This chapter is something of an outlier compared to the other chapters. While
the previous part of this work pertains to \emph{natural} science -- namely,
theoretical physics -- where simulations served us as a tool rather than as the
primary object of study, a substantial portion of this thesis has been
dedicated to expanding an existing stabilizer simulator to deal with new
problems (cf.  \cref{sec:lxe-numeric,sec:rel-ent-stab,sec:upperbound-numerics}). As a result, our
discussion thereof will necessarily feature
%cursed language devised by the utterly deranged: computer scientists. To
%prepare for this, we will introduce some of its vocabulary here.
terminology commonly associated with the field of \emph{computer} science.
To ensure clarity, we will briefly introduce key computer science concepts
relevant to our discussion.  

The first concept we introduce is asymptotic behavior, along with the so-called
\emph{Big }$\mathcal{O}$ notation. It is not exclusive to computer science as such, as we
often deal with asymptotic behavior in physics as well.  In computer science,
the Big $\mathcal{O}$ notation is used to describe the space (i.e. data storage) and
time requirements of algorithms with increasing input size. The notation is
defined precisely in \cref{defn:bigo-theta}.
\begin{defn}[$\Theta$ and $\mathcal{O}$ notation
  \cite{cormenIntroductionAlgorithms2009}]\label{defn:bigo-theta}
  Let $g(n)$ be a positive function. Then $\Theta(g(n))$ is the set of
  functions 
  \begin{align*}
    \Theta(g(n)) = \left\{ \ f(n) \mid \liminf_{n\to\infty} \frac{f(n)}{g(n)} > 0 \
    \wedge \ \limsup_{n\to\infty} \frac{f(n)}{g(n)} < \infty \ \right\}
  .\end{align*}
  Similarly, $\mathcal{O}(g(n))$ is the set of functions 
  \begin{align*}
    \mathcal{O}(g(n)) = \left\{ \ f(n) \mid 
    \limsup_{n\to\infty} \frac{f(n)}{g(n)} < \infty \ \right\}
  .\end{align*}
\end{defn}
We will later see some examples of this notation in use. For now it suffices to
note that different problems and the algorithms we use to solve them (or verify
the validity of a solution) are grouped into a variety of \emph{complexity
classes}. For instance, an algorithm is said to scale polynomially (in time) if
its asymptotic behavior is $\mathcal{O}(n^\alpha)$ with $\alpha \geq 1$, and
exponentially if it is $\mathcal{O}(2^n)$. These classes constitute an
important subfield in theoretical computer science in the form of
complexity theory, where numerous classes exist.\footnote{A continuously
updated list can be found in \cite{ComplexityZoo}.} The two examples from above
fall in the classes \textsf{P} (polynomial time) and \textsf{EXP} (exponential
time) respectively.  While there are lots of caveats to this highly simplified
explanation, a deeper examination of complexity theory lies beyond the scope of
this thesis. However, the key takeaway is summarized in
\cref{defn:efficient-alg}.
\begin{defn}[Algorithmic efficiency \cite{ComplexityZoo}]\label{defn:efficient-alg}
  A problem is considered efficiently solvable on a classical computer if it
  belongs to the complexity class \textsf{P}.
\end{defn}
Having laid out these foundational concepts, we can now delve deeper into the
subtleties of stabilizers and the problem of simulating stabilizer
circuits, potentially gaining a new appreciation for their intricacies.

\section{Consequences of the Gottesman-Knill theorem}\label{sec:sim-stab}
In this section we explore the implications of \cref{thm:gottesman-knill} and
provide a proper introduction to the simulation algorithm that forms the
foundation for the numerical experiments in this thesis.  In particular, we
present the \emph{tableau algorithm}, as proposed by Aaronson and Gottesman in
\cite{aaronsonImprovedSimulationStabilizer2004}.  Consequently, this section
draws largely from outside sources, namely Refs.
\cite{aaronsonImprovedSimulationStabilizer2004, arabLectureNotesQuantum2024,
gottesmanStabilizerCodesQuantum1997,
gottesmanHeisenbergRepresentationQuantum1998}. As this following section serves
to introduce the computational algorithms based on the stabilizer formalism, it
can (and should) be read as a sequel to \cref{sec:stab-basics}.
%This section provides an introduction to the tableau algorithm. It can (and
%should)
%be read as the sequel to \cref{sec:stab-basics}. Note that this section is
%based on the work of \citeauthor{aaronsonImprovedSimulationStabilizer2004} in
%\cite{aaronsonImprovedSimulationStabilizer2004}.

We begin by recalling that the stabilizer group does not need to be stored in
full to unambiguously describe the state. Since the stabilizer group is finite,
its structure can be fully encapsulated by storing only its \emph{generators}
in memory. This reduces the amount of data to be stored to memory from $2^n$
to $n$, owing to the well-known fact from group theory that a finite group $G$
has a generating set of size $\log \abs{G}$. That is, an $n$-qubit pure state
$\ket{\phi}$ with stabilizer group $S\left(\ket{\phi}\right)$ has a generating
set of size $\log 2^n = n$.

To determine the actual memory requirements we examine the generators
themselves.  Each generator consists of an array of $n$ Pauli matrices and a
sign. Since there are $4$ Pauli matrices (including the identity), we require 2
bits to encode each of them, along with an additional bit for the sign.
Consequently, the memory requirements for encoding a pure state in the
stabilizer formalism is $n(2n+1)$. In other words, storing only the stabilizer
generators reduces the space complexity of stabilizer simulations from
$\mathcal{O}\left( 2^n \right)$ to $\mathcal{O}\left( n^2 \right)$.

For practical purposes, these bits can be assorted to two $n\times n$ matrices
and a vector containing the $n$ signs. This way of writing the generators is
called the \emph{Tableau Representation}. As an example, consider the state
$\ket{\phi}=\ket{0000}$, which is stabilized by Stab$(\ket{\phi}) = \langle
Z_1, Z_2, Z_3, Z_4\rangle$. The stabilizer tableau $\mathcal{T}$ of
$\ket{\phi}$ is then given by
\begin{align}\label{eq:stabtab}
\mathcal{T}_{\ket{\phi}} = 
\left[
  \begin{array}{cccc|cccc|c}
    0 & 0 & 0 & 0 & 1 & 0 & 0 & 0 & 0 \\
    0 & 0 & 0 & 0 & 0 & 1 & 0 & 0 & 0 \\
    0 & 0 & 0 & 0 & 0 & 0 & 1 & 0 & 0 \\
    0 & 0 & 0 & 0 & 0 & 0 & 0 & 1 & 0
  \end{array}
\right]
%\left[
%  \begin{array}{cccc|cccc|c}
%    \multicolumn{4}{c|}{\underbrace{
%    \begin{matrix}
%    0 & 0 & 0 & 0 \\
%    0 & 0 & 0 & 0 \\
%    0 & 0 & 0 & 0 \\
%    0 & 0 & 0 & 0
%    \end{matrix}}_{X\text{-matrix}}} &
%    \multicolumn{4}{c|}{\overbrace{
%    \begin{matrix}
%    1 & 0 & 0 & 0 \\
%    0 & 1 & 0 & 0 \\
%    0 & 0 & 1 & 0 \\
%    0 & 0 & 0 & 1
%    \end{matrix}}^{Z\text{-matrix}}} &
%    \underbrace{
%    \begin{matrix}
%    0 \\
%    0 \\
%    0 \\
%    0
%    \end{matrix}}_{\text{signs}}
%  \end{array}
%  \right]
.\end{align}
Each row in \cref{eq:stabtab} represents a generator of Stab$(\ket{\phi})$. The
first four columns are the $X$-matrix, the next four are the $Z$-matrix and the
last column represents the sign. The $i$-th column of the $X$ and $Z$ matrix
encode the Pauli matrix at position $i$ in the tensor product of the
corresponding generator, where $00\equiv I$, $01\equiv Z$, $10\equiv X$, and
$11\equiv Y$. 

At this point, we have merely stored the state in memory. However,
\cref{thm:gottesman-knill} implies that simulations of stabilizer cicruits can
also be done efficiently on a classical computer. As stated in
\cref{defn:efficient-alg}, for an algorithm to be
considered \enquote{efficient}, it must have polynomial time complexity.
Thus, the Gottesman-Knill theorem can be rephrased to state that the bits encoding
$\ket{\phi}$ can be updated in polynomial time after a Clifford gate is
applied to $\ket{\phi}$. Before we show this, let us first consider how the individual
gates act on our generator tableau. 
\begin{alg}[Application of Clifford gates to stabilizer
  tableau]\label{alg:cliff-gates-alg1}
  The following modifications have to be applied to the stabilizer tableau
  after the application of
  \begin{itemize}
    \item $H$ to qubit $a$:

      Swap column $a$ of the $X$ matrix with the $a$-th column of the
      $Z$-matrix.
    \item $S$ to qubit $a$:

      Modify column $a$ of the $Z$-matrix such that the
      new $z_{ia} = z_{ia} \texttt{XOR} x_{ia}$, i.e. apply bitwise
      \texttt{XOR} from column $a$ of $X$ into column $a$ of $Z$.
    \item CNOT from control $a$ to target $b$:
     
      Apply bitwise \texttt{XOR} from
      column $a$ to $b$ of $X$, and from column $b$ to column $a$ of $Z$.
    \item Random or deterministic outcome of computational basis measurement on
      qubit $a$:

      The outcome is deterministic iff. column $a$ of the $X$-matrix is all 0s.
    \item Measurement of qubit $a$ with a random outcome

      Let $x_{ia}=1$. We first apply bitwise \texttt{XOR} from row $i$ into any
      subsequent row $j$, where $x_{ja}=1$. We then set row $i$ to $0$
      everywhere, except $z_{ia}=1$ and set the sign to $0$ or $1$ randomly.
  \end{itemize}
\end{alg}
These rudimentary algorithms follow the rules of the stabilizer formalism, as
laid out in \cref{sec:stab-basics}. Let us now consider the performance of
the algorithms in \cref{alg:cliff-gates-alg1}.

\begin{thm}[Simulating stabilizer gates]\label{thm:sim-stab-comp}
  Simulating a Clifford gate on an $n$-qubit stabilizer state requires $\Theta(n)$
  time, while a measurement gate is simulated in $O(n^2)$ or $O(n^3)$ time for
  random and deterministic outcomes respectively.
\end{thm}
\begin{proof}
  We already know that an $n$-qubit stabilizer state $\ket{\phi}$ can be represented as an
  $n\times(2n+1)$ tableau $\mathcal{T}$, where the rows are the $n$ generators in the
  afforementioned $(2n+1)$-bit representation. Any computational basis state can
  thus be represented by
  \begin{align}
    \mathcal{T} = 
    \left[
      \begin{array}{cccc|cccc|c}
        0 & 0 & \ldots & 0 & 1 & 0 & \ldots & 0 & \pm \\
        0 & 0 & \ldots & 0 & 0 & 1 & \ldots & 0 & \pm \\
        \vdots & \vdots & \ddots & \vdots & \vdots & \vdots & \ddots & \vdots & \vdots \\
        0 & 0 & \ldots & 0 & 0 & 0 & \ldots & 1 & \pm
      \end{array}
    \right]
  ,\end{align}
  where $Z$ and $I$ are encoded as $01$ and $00$ respectively, and $\pm$ is a
  placeholder for either $0$ or $1$ depending on the sign. Simulating a
  Clifford gate $U$ on $\ket{\phi}$ maps $g_i$, that is, the $i$-th row of
  $\mathcal{T}$, to $U g_i U$. As discussed in \cref{alg:cliff-gates-alg1},
  this operation updates at most two columns. Therefore, simulating $U$ takes
  $\Theta(n)$ time.

  To show the scaling behavior of measurements we consider computational basis
  measurements on qubit $a$, $Z_a$, without loss of generality.\footnote{We can
    consider those without loss of generality, since we can always apply
    $\mathcal{O}(n)$ unitaries to change basis in $\Theta(n)$ time, which
  doesn't affect the scaling behavior of $\mathcal{O}(n^2)$.} Each individual
  qubit in $\ket{\phi}$ is either in a computational basis state $\ket{0}$ or
  $\ket{1}$, where we have a deterministic outcome, or in a superposition of
  both states with equal amplitude, where either outcome is random with
  probability $p=1 / 2$. Recall from \cref{sec:stab-basics} that the process of
  determining the type of outcome (random or deterministic) is done by checking
  the commutation relations between the measurement operator and the generators
of $\ket{\phi}$. If some generator $g_k$ anti-commutes with $Z_a$, i.e.
\[ \{g_k, Z_a\} =0 \quad{\text{or}}\quad [g_k, Z_a] \neq 0,\]
the outcome is random,
  otherwise it is deterministic (this implies a scaling of $\mathcal{O}(n)$
  already, since we go through the entire matrix in the worst case). We now
  consider these cases separately.
  
  \par{Case 1, random outcome:}
  Let $g_k$ be a stabilizer generator with $\{g_k, Z_a\}=0$, implying a random
  outcome. As the outcome, we randomly choose $x\in\{0,1\}$. We then multiply
  any subsequent rows that anticommute with $Z_a$ by $g_k$ in order for them to
  commute with $Z_a$. We then replace $g_k$ by $\pm Z_a$ depending on the
  random outcome. Since this algorithm takes up to $n$ row multiplications,
  the runtime scales with $\mathcal{O}(n^2)$.

  \par{Case 2, deterministic outcome:}
  If we did not find an anticommuting generator, we know that our stabilizer
  group contains $Z_a$. However, this does not necessitate $Z_a$ being in the
  generating set.
  We thus need to modify
  $\mathcal{T}$ such that its row space contains $\left( 00\cdots 0_a\cdots 0\mid
  00\cdots 1_a \cdots 0 \mid \pm \right) = \pm Z_a$ and then read out the signs
  vector, which is the result. This is done by Gaussian
  elimination, which requires $\mathcal{O}(n^3)$ time.
\end{proof}
This proof, in principle, also proves \cref{thm:gottesman-knill}, since the
simulation of any of a circuit's constituents is in \textsf{P}. Consequently, a
finite number $m$ of them will be in $\mathcal{O}(mn^\alpha)$, and thus also in
\textsf{P}.

\subsection{The Aaronson-Gottesman Algorithm}\label{sec:tableau}
In the previous section we have shown that measurements take $\mathcal{O}(n^3)$
time in practice. However, Aaronson and Gottesman showed in
\cite{aaronsonImprovedSimulationStabilizer2004} that with the cost of a factor
2 increase in memory requirements we can improve measurements to have
quadratic time complexity $\mathcal{O}\left(n^2\right)$, independent of
\enquote{outcome type}. In particular, for each of the $n$ stabilizer
generators we store a destabilizer (also equivalently referred to as
antistabilizer) generator, which are also tensor products of Pauli operators.
These $2n$ operators together generate the full $n$-qubit Pauli group
$\mathcal{P}_n$.

The tableau idea applied in \cref{eq:stabtab} can be expanded into
\begin{align}\label{eq:new-stabtab}
  \mathcal{T}_{\ket{\phi}} = 
  \left[
    \begin{array}{ccc|ccc|c}
      x_{11} & \cdots & x_{1n} & z_{11} & \cdots & z_{1n} & r_1 \\
      \vdots & \ddots & \vdots & \vdots & \ddots & \vdots & \vdots \\
      x_{n1} & \cdots & x_{nn} & z_{n1} & \cdots & z_{nn} & r_n \\ \hline
      x_{(n+1)1} & \cdots & x_{(n+1)n} & z_{(n+1)1} & \cdots & z_{(n+1)n} & r_{n+1} \\
      \vdots & \ddots & \vdots & \vdots & \ddots & \vdots & \vdots \\
      x_{(2n)1} & \cdots & x_{(2n)n} & z_{(2n)1} & \cdots & z_{(2n)n} & r_{2n}
    \end{array}
  \right]
\end{align}
for an arbitrary stabilizer state $\ket{\phi}$. The first $n$ rows represent
the newly introduced destabilizer states, while the last $n$ rows constitute
the tableau of stabilizers we have already discussed. For later convenience, an
additional $(2n+1)$st row is added to the tableau. A minimal example of the
generalized tableau is the state $\ket{00}$, which has
\begin{align}\label{eq:new-stabtab-example}
  \mathcal{T}_{\ket{00}} = 
  \left[
    \begin{array}{cc|cc|c}
      1 & 0 & 0 & 0 & 0 \\
      0 & 1 & 0 & 0 & 0 \\ \hline
      0 & 0 & 1 & 0 & 0 \\
      0 & 0 & 0 & 1 & 0 \\ \hline
      0 & 0 & 0 & 0 & 0 
    \end{array}
  \right]
.\end{align}
A quick remark on notation: we will refer to the $i$th row of $\mathcal{T}$ by
$R_i$, where it is clear from the respective context if the row refers to a
stabilizer or destabilizer generator. If we explicitly refer to one of them, we
write $g_i$ for the $i$th stabilizer generator and $h_i$ for the $i$th
antistabilizer generator. Matrix elements of the $X$ and $Z$ matrices are
denoted by $x_{ab}$ and $z_{ab}$ respectively. Entries of the sign vector are
denoted by $r_i$ (cf. \cref{eq:new-stabtab}).

Although it is simply stated, it is not immediately obvious how (a) we choose
destabilizer generators, and (b) how this improves the scaling of measurements.
For now we say that the simulation algorithm starts in the state
$\ket{0}^{\otimes n}$, where the initial tableau is taken as the $n$-qubit
generalization of \cref{eq:new-stabtab-example}. Any other stabilizer state can
then be arrived at via Clifford or Pauli measurement gates. The former can be
implemented as follows.
\begin{alg}[Improved simulation of Clifford gates]\label{alg:tab-clifford}
  Let $\oplus$ denote bitwise \verb|XOR|. The implementations are
  \begin{itemize}
    \item CNOT from control $a$ to target $b$:

      $\forall i \in \{1,\ldots,2n\}$ $r_i := r_i \oplus x_{ia}z_{ib}\left(
      x_{ib} \oplus z_{ia} \oplus 1 \right)$, $x_{ib} := x_{ib} \oplus x_{ia}$,
      and $z_{ia} := z_{ia}\oplus z_{ib}$.
    \item $H$ on qubit $a$:

      $\forall i \in \{1,\ldots,2n\}$ $r_i := r_i \oplus x_{ia}z_{ia}$, then
      swap $x_{ia}$ and $z_{ia}$.
    \item $S$ on qubit $a$:

      $\forall i \in \{1,\ldots,2n\}$ $r_i := r_i \oplus x_{ia}z_{ia}$, then
      set $z_{ia}$ to $z_{ia} \oplus x_{ia}$.
  \end{itemize}
\end{alg}
We know from earlier considerations that simulating measurement gates,
requires the multiplication of
generators. To this end, we introduce 
a subroutine called \texttt{rowsum(h,i)}.
It updates the $h$-th generator to be $h+i$
and keeps track of the phase $r_h$.
Its implementation is given in \cref{alg:rowsum}.
\begin{alg}[rowsum]\label{alg:rowsum}
  First we define a function $g(x_1, z_1, x_2, z_2)$ taking 4 bits as input and
  outputting the exponent of the imaginary unit $i$ after $x_1 z_1$ and $x_2 z_2$ are multiplied.
  We focus on four cases explicitly, namely
  \[
  \begin{array}{ccc}\toprule
      x_1 & z_1 & g(x_1,z_1,x_2,z_2) \\ \midrule
      0 & 0 & 0 \\
      0 & 1 & x_2\left( 1-2z_2 \right) \\
      1 & 0 & z_2\left( 2x_2-1 \right) \\
      1 & 1 & z_2 - x_2 \\ \bottomrule
  \end{array}
  \]
  For the rowsum routine we set $r_h$ to $0$ if
  \begin{align}\label{eq:congruentdings}
    2r_h + 2r_i + \sum_{j=1}^n g\left(x_{ij},z_{ij},x_{hj},z_{hj}\right) \equiv
    0 \text{ (mod } 4)
  .\end{align}
  or set $r_h$ to $1$ if the sum in \cref{eq:congruentdings} is congruent to
  $2$ mod $4$. Next, we set $x_{hj}$ to $x_{ij}\oplus x_{hj}$ and $z_{hj}$ to $z_{ij}
  \oplus z_{hj}$ for all $j$, where $\oplus$ denotes bitwise \verb|XOR|.
\end{alg}
Next it will prove useful to define the \emph{symplectic inner product} between
two Pauli operators $A$ and $B$ in tableau representation as
\begin{align}\label{eq:symp-inner-prod}
  A\cdot B = x_{a1}z_{b1} \oplus \cdots \oplus x_{an}z_{bn} \oplus x_{b1}
  z_{a1} \oplus \cdots \oplus x_{bn}z_{an}
.\end{align}
This inner product is $0$ if $A$ and $B$ commute and $1$ if they anti-commute.

Equipped with rowsum and the symplectic inner product, we can now examine the
simulation of measurement gates.
\begin{alg}[Improved simulation of Measurement gates]\label{alg:tab-measure}
  Suppose we measure $\hat{O}$. As a $0$th step,
  check if there exists a $p\in\{n+1,\ldots,2n\}$ (the stabilizer
  generators) such that $[\hat{O},R_p] \neq 0$. This can be done by
  multiplication with respect to \cref{eq:symp-inner-prod}. Then there are two
  cases:

  Case 1, such a $p$ exists. First, call rowsum($i,p$) for all $i \in
  \{1,\ldots,2n\}$ such that $i\neq p$ and $[\hat{O},R_i] \neq 0$. Next,
  set the $(p-n)$th row equal to the $p$th row. Then, set row $p$ equal to
  $\hat{O}$ and set $r_p$ to 0 or 1 with equal probability. Finally, return
  $r_p$ as measurement outcome.

  Case 2, such a $p$ does not exist. First, set the entire $(2n+1)$st row (the one
  added for convenience earlier) to 0. Next, call rowsum($2n+1, i+n$) for all
  $i\in\{1,\ldots,n\}$ (the destabilizer generators) such that
  $[\hat{O},R_i]\neq 0$. Finally, return $r_{2n+1}$ as measurement outcome. 
\end{alg}

With \cref{alg:tab-clifford,alg:rowsum}, all possible allowed modifications to
the stabilizer tableau $\mathcal{T}$ are defined. \Cref{prop:comm-tab} collects
some symmetries of this simulator. That is, some commutation relations are
invariant under operations of the Aaronson-Gottesman tableau algorithm. It is
these relations we want to keep intact, when setting out to expand the existing
simulator with more functionalities.
\begin{prop}[Invariants of the tableau algorithm]\label{prop:comm-tab}
  The following are invariant under operations of the tableau algorithm
  \begin{enumerate}
    \item $R_{n+1},\ldots,R_{2n}$ generate $S(\ket{\phi})$, and $R_1, \ldots,
      R_{2n}$ generate $\mathcal{P}_n$.
    \item $R_1, \ldots, R_n$ commute.
    \item $\forall h \in \{1,\ldots,n\}, \ \{R_h, R_{h+n}\} = 0$
    \item $\forall i,h \in \{1,\ldots,n\}, \ \text{with } i\neq h, \ [R_i, R_{h+n}] = 0$
  \end{enumerate}
\end{prop}
We will conclude this section with a note of caution; the simulator we set out
to expand has, in contrast to how it was defined in this section, the
stabilizers in even-numbered rows of the tableau, and each associated
antistabilizers in the row below it. Although it is merely a change of indices,
it is a major one, and for didactic reasons we chose to forego this change.
Since it only matters which antistabilizer is associated with each stabilizer,
we try to limit the mention of any explicit ordering of generators in the
tableau.

%\section{Expanding the Tableau algorithm}\label{sec:expanding}
%
\section{New functions on pure states}\label{sec:purestates}
The first function implemented is the \emph{project} function. It is used in
the algorithm for computing the linear cross entropy, as successful projections
in the circuit yield $\chi = 1$ and $\chi = 0$ otherwise. Luckily for us,
projections are already a part of the existing simulator in the form of
projective measurements. The important difference between the measurement
algorithm and projections is that we already know the measurement outcome as an
argument of our function. That is, we take the usual measurement algorithm, but
add a new argument for the measurement result, which is projected onto. As
such, the projection always works for a random result, since one of the steps
in the measurement was to flip a coin for the sign, which is now a fixed value.
However, for the deterministic case we are not as agnostic. Since we try to
project onto a state, which is already in the stabilizer, the signs of what we
have and what we pass as argument should match. If they don't, the function
returns \verb|false|. This is then used as break condition when computing the
LXE. The project algorithm is outlined in
\cref{alg:projection}.
\begin{alg}[Projection onto Pauli eigenstates]\label{alg:projection}
  Suppose we want to apply $P = \frac{I + \hat{O}}{2}$. As step 0, we again
  need to check if projection changes the state,
   by checking if there exists a stabilizer generator $R_p$
  such that $[\hat{O},R_p]\neq 0$. Then there are two cases:

  Case 1, projection modifies the state. We repeat the steps from case 1 of
  \cref{alg:tab-measure}, but instead of randomly choosing the sign, we use the
  one we want to project onto. Finally, we return \verb|true| since the
  projection was successful.

  Case 2, projection does not modify the state. We similarly repeat the steps
  from case 2 of \cref{alg:tab-measure}, but return \verb|true| or \verb|false|
  depending on if the sign we pass as function argument matches $r_{2n+1}$ or
  not.
\end{alg}
An important point to note is that this function does not resemble anything we
could do in an experiment. There is no experimental apparatus conceivable to
perform the operation we are simulating. It is possible only because we
perform a classical simulation and know mathematically what a projection
operation does. This kind of degree of freedom is noteworthy, and we should
keep it in mind going forward.

\subsubsection{Forced projection}
As we just discussed, the projection can fail and return \verb|false|. This is rather
unhelpful, since this would terminate the simulation. In
\cref{sec:naive-approach}, this behavior is undesirable. Consequently, we need
to add a function, which forces a specific projection by altering the structure
of the tableau.

The only way we know if a projection fails is by computing the outcome of the
deterministic measurement and checking if the value is equal. This process
involves reconstructing the measurement operator with the help of anticommuting
antistabilizers (cf. \cref{alg:tab-measure}). If the result is not equal, we do
not have the generator with the correct sign in the stabilizer group. As a
consequence, we need to alter the tableau in a certain way. To this end, we
employ the method of case 1 in \cref{alg:tab-measure} and place the correct
sign. We thus pretend to be able to project successfully. This algorithm is
summarized in \cref{alg:force-project}.
\begin{alg}[Forced projection]\label{alg:force-project}
  The algorithm works the same way as \cref{alg:projection}, but instead of
  returning \verb|false| upon failed projection, we manipulate the tableau
  analogously to case 1 of \cref{alg:tab-measure} and insert the correct sign.  
\end{alg}
\section{Implementing mixed states}\label{sec:mixed-states}
The tableau algorithm in the form it is outlined in \cref{sec:tableau}
currently supports stabilizer circuit simulations exclusively for pure states.
This however,
neglects a more general class of quantum states, namely, mixed states. It has
become apparent in \cref{ch:rel-ent} why this adaption is necessary for our
purposes. Ideally, we want to build on the existing infrastructure we have been
using for pure states and extend it such that pure states arise as a special
case within the simulator. 
Let us approach this problem
heuristically by thinking about an intuitive approach to incorporate mixed
states to the simulator.

Consider the density matrix of a general $N$-qubit stabilizer
state with generating set Stab$\left( \rho \right) = \langle g_1, \ldots, g_n
\rangle$,
\begin{align}
  \rho = \frac{1}{2^N} \prod_{i=1}^n I + g_i
.\end{align}
Here, $\rho$ describes a pure state iff. $n=N$ and a mixed state otherwise.
That is, by reducing the number of generators, we increase the state's
mixedness, with the maximally mixed state represented by the trivial group,
generated by the empty set.

What about the action of unitary transformations, i.e. Clifford gates, on our
state? Since applying unitaries to density matrices works by conjugation with
$U$ and $U^\dagger$, their application remains unchanged from the pure-state
case. Measurements, however, introduce a new challenge:
One can show that when measuring any
Pauli operator $\hat{O}$ on a qubit in a maximally mixed state, the outcomes
will be random and their probabilities uniformly distributed. However, this
contrasts the previous instances where measurement outcomes were random, since
there are no anticommuting generators of $\rho$ with $\hat{O}$. This is
certainly something to take note of in modifying the existing measurement
algorithm.

With mixed states come new possibilities for quantities we could query on our
system. For instance, the Von Neumann entropy (cf. \cref{defn:vonneumann}) of a
pure state is always $0$. It is only non-zero for a mixed state.
Furthermore, the cross and relative entropy only
really get particularily interesting when considering mixed states, as we have
discussed in detail in \cref{sec:rel-ent-stab}. We should therefore include the
ability to access these entropic quantities, and also compare two states to
another, as is done for the cross and relative entropy.

%Finalizing our philosophizing on mixed states is possibly the most important
%question of how to obtain them.
One central question remains left unanswered: how \emph{do} we obtain mixed
states within this framework?  That is, if the previous program pertained to
pure states only, what would be a natural way to introduce mixedness?  One
approach could certainly be to start with a pure state of a larger system and
then trace out entangled qubits.  So, a natural inclusion to our algorithm
would be a partial trace function.  Alternatively, we can recognize once more
that we are not bound by the limits of nature; similarly to the project
function, which has no natural or experimental pendant, we can artificially
introduce mixedness by selectively removing stabilizers.
%Alternatively, mixedness can be introduced artificially by selectively removing
%stabilizers.
For instance, removing a generator that acts only on one qubit (which is not
coupled to other qubits) should yield a mixed state for that qubit. This allows
outside control of mixedness within the simulation.
%Next, we could artificially introduce mixedness by removing certain generators
%from the stabilizers. If there is a generator acting exclusively on one qubit,
%with no other stabilizer acting on it, and we shrink our generating set by said
%stabilizer, we expect this qubit (and only this qubit) to be in a mixed state.
%Thus, we could artificially introduce mixedness, by bringing our stabilizers in
%such a form and removing this qubit.

Let us briefly summarize this train of thought:
\begin{description}
  \item[Minimal reworking] The current algorithm (cf. \cref{sec:tableau})
    should be extended with minimal changes. Any previously written simulation
    based on it may not break.
  \item[Generator count] An $N$-qubit pure state has $N$ stabilizer generators,
    while a mixed state has $0\leq n<N$ generators.
  \item[Unitaries] The application of unitaries is agnostic to mixedness. They
    should work \emph{exactly} the same way they have before.
  \item[Measurements] Measurements introduce a new contingency for random
    results. The existing measurement function should be made to be able to handle it.
  \item[Entropy] Mixed states allow for a broader spectrum of entropic
    quantities to be computed. %They should be included in the simulator.
  \item[Partial trace] There should be a function, which implements the partial trace over (at
    least) one qubit.
  \item[Classical advantages] Classical simulation allows us to construct artificial ways of
    introducing mixedness.
\end{description}
In the following sections we will go over the algorithms and their
implementations of each of the
above points in detail.

\subsubsection{On \enquote{minimal reworking}}
The point about \enquote{minimal reworking} is somewhat unrelated to
physics, but rather computer science (once again). Going forward, our working
philosophy should, of course, be the faithful implementation of physical
phenomena to classical computers. However, we also try to follow common
practices and principles of software development. In particular, we aim for as
little repetition as necessary, by the DRY (don't repeat yourself) principle,
and introduce new subroutines only if deemed unavoidable. If there exists a
function that does what we want, we will use it.

\subsection{The mix attribute}
%\textcolor{orange}{Each instance of the \texttt{Qubit} object now comes with a
%new attribute called \texttt{mix}! Its like a buy one get one free type deal! I
%hecking love Object Oriented Programming!}
%
For the first point\footnote{Considering \enquote{Minimal reworking} is our
working philosophy, the thing to keep in mind, we count it as $0$th point}  --
the generator count --, we introduce the \emph{mix} attribute to the simulator.
In the existing implementation, each multi-qubit stabilizer system is an
instance of the \texttt{Qubits} class. Its member functions are the possible
actions on a stabilizer state. By introducing the \verb|mix| attribute, which
is an integer $0\leq mix \leq N$, we keep track of how many stabilizers
contribute to the description of the physical state. The technical use of this
attribute is to control which rows of the tableau correspond to those
generators which describe the physical state, and which rows are generators
completing the tableau. To clarify, if we were to take the stabilizer tableau
of an arbitrary pure state, its \verb|mix| value will be $0$. Incrementing this
value by $1$ increases the mixedness and constraints the rows of the tableau we
interpret as the stabilizer generators. Consider the example of the two-qubit
bell state $\ket{\phi} = \left( \ket{00} + \ket{11} \right) /\sqrt{2}$ with
generating set $\langle ZZ, XX\rangle$. The corresponding stabilizer tableau
(omitting the scratch row) is
\begin{align}
\mathcal{T}_{\ket{\phi}} = 
  \left[
    \begin{array}{cc|cc|c}
      1 & 0 & 0 & 0 & 0 \\
      0 & 0 & 0 & 1 & 0 \\ \hline
      0 & 0 & 1 & 1 & 0 \\
      1 & 1 & 0 & 0 & 0 \\ 
    \end{array}
  \right]
.\end{align}
If we now increment \verb|mix| by $1$, the last (anti-)stabilizer row is still
carried around in the simulation, but does not correspond to any stabilizer of
the state. For the above state, incrementing \verb|mix| reduces the generating
set to $\langle ZZ \rangle$, which corresponds to a mixed state with density
matrix
\begin{align}\label{eq:sample-mixedstate}
  \rho = \frac{1}{4}\left( I + ZZ \right) = \frac{1}{2}\mqty(\dmat[0]{1,0,0,1})
.\end{align}
The corresponding bit matrix in the tableau representation then reads
\begin{align}
\mathcal{T}_{\rho} = 
  \left[
    \begin{array}{cc|cc|c}
      1 & 0 & 0 & 0 & 0 \\ \hdashline
      0 & 0 & 0 & 1 & 0 \\ \hline
      0 & 0 & 1 & 1 & 0 \\ \hdashline
      1 & 1 & 0 & 0 & 0 \\ 
    \end{array}
  \right]
\end{align}
where the dashed lines correspond to the increment of \verb|mix|. Note that we,
in principle, have all the freedoms laid out in \cref{prop:comm-tab} to choose
the generators below the dashed line as it is only needed for mathematical and
technical convenience. We will see the advantages of keeping tabs on the
full-rank tableau in the following sections.
\subsection{Unitaries and measurements}
Now that we have introduced a method to describe a mixed state to the
simulator, let us further investigate how its member functions need to be
adapted in order to faithfully simulate the behavior of mixed states. Luckily
for us, adapting unitary transformations is a rather simple task, since it
works the same as for mixed states. It thus requires no further inquiry. 

The same cannot be said about measurements. As introductory example, consider
the mixed state described by the density matrix in \cref{eq:sample-mixedstate}.
Suppose we now perform a $Z$ measurement on the first qubit. The fact that the
measurement operator commutes with every generator\footnote{$ZZ$ in this simple
example} might lead us to believe that the measurement outcome is
deterministic. However, we also have
\begin{align}
  p(m_Z = +1) &= \Tr[\frac{I+Z_1}{2} \frac{1}{4} (I+ZZ)] \nonumber \\
              &= \Tr[\frac{1}{8}\left(I + Z_1Z_2 + Z_1 + Z_2  \right)] \nonumber \\
              &= \frac{1}{8} \Tr[I] = \frac{1}{2}
.\end{align}
This seems to contradict our previous method of determining the \emph{type} of
measurement outcome. As a consequence, we need to include this new contingency
into the existing measurement function. At this point, recall that we still
keep the whole tableau of generators, as if we had a pure state, and just track
how many are not descriptors of the physical state. Therein also lies the
solution to the problem; we can intercept the case for random outcomes in a
mixed state by checking the tableau below the mix line as well. Note that we
also include the antistabilizers in this case. The commutation relations laid
out in \cref{prop:comm-tab} ensures the success of this algorithm. The
augmentations of \cref{alg:tab-measure} is given in \cref{alg:sim-meas-mixed}.

\begin{alg}[Simulation of measurement gates on mixed
  states]\label{alg:sim-meas-mixed}
  Suppose we measure $\hat{O}$. As a $0$th step, check if there exists a $p \in
  \{ n+1, \ldots, 2(n-mix) \}$ (generators describing the state) such that
  $[\hat{O}, R_P] \neq 0$. Then there are three cases:

  Case 1, such a $p$ exists. This case is the same as case 1 of
  \cref{alg:tab-measure}.

  Case 2, such a $p$ does not exist. First, check analogously to step $0$ if
  there exists a $p \in \{n-mix, \ldots, n, 2(n-mix), \ldots, 2n\}$ such that
  $[\hat{O}, R_P] \neq 0$. if such a $p$ does not exist, this case reduces to
  case 2 of \cref{alg:tab-measure}. If it exists, we have the new third case.

  Case 3, there exists a $p \in \{n-mix, \ldots, n, 2(n-mix), \ldots, 2n\}$
  such that $[\hat{O}, R_P] \neq 0$. We perform the steps from case 1 of
  \cref{alg:tab-measure} with this $p$. Then we swap rows such that $R_p$ is
  included in the topmost row of the \verb|mix| stabilizers. (Depending on $p$,
  this might include a swap of an antistabilizer to the stabilizers.) Then, we
  decrement \verb|mix| by 1.
\end{alg}
\subsection{Entropies}\label{sec:sim-entropies}
By taking a broader class of quantum states into our consideration, we unlock a
broader set of entropic quantities to be investigated and computed. We will
here introduce the functions that compute the von Neumann, cross and relative
entropy.
\subsubsection{Von Neumann entropy}
The first new function the \verb|mix| attribute allows us to realize is the von
Neumann entropy. We know from \cref{sec:rel-ent-stab} that the von Neumann
entropy has a simple expression for stabilizer states, namely
\begin{align}
  S\left( \rho \right) = -\Tr[\rho\log\rho] = N - \abs{S_\rho}
\end{align}
with the number of qubits $N$ and the rank of the stabilizer group
$\abs{S_\rho}$. Since the rank of a group is the size of its smallest
generating set, we already have the von Neumann entropy baked into our
extension. By definition, \verb|mix| is the number of generators the state has
fewer than a pure state, which has $N$ generators. Thus, the von Neumann
entropy is just \verb|mix|. Due to its simplicity we will forego a detailed
listing of the function here. Nonetheless, the implementation of the von
Neumann entropy in the simulator (written in \verb|C++|) can be found in
\cref{ch:apdx-code}.

\subsubsection{Cross and relative entropy}
We want to include a way to compute the cross and relative entropy between two
stabilizer states in our simulator. However, this requires an auxiliary
subroutine, which will be introduced beforehand.

\subsubsection{\texttt{is\_subgroup\_of}}
Since we know from \cref{prop:subgroup} that the relative or cross
entropy $\Tr[\rho\log\sigma]$ only takes on finite values if and only if the
stabilizer group of $\sigma$, Stab$(\sigma)$, is a subgroup of the stabilizer group
of $\rho$, Stab$(\rho)$. As such, before continuing with the implementation of
other entropic quantities, it seems a worthwhile endeavor to introduce a
subroutine that verifies if one stabilizer group is a subgroup of the other.
To simplfy notation, we fix the Qubit objects such that we ask if $\sigma$ is a
subgroup of $\rho$.

The function \verb|is_subgroup_of| should, as its name implies, return a
boolean; \verb|true| if $\sigma$ is a subgroup of $\rho$, and \verb|false| if
it is not. Its implementation could, for instance, be as a member function of
the Qubit class, which takes as input a reference to an instance of another Qubit
object. This way, the function is still within the scope of the class,
and already has access to the tableau and consequently the generators. 

Answering the question if one stabilizer group is the subgroup of another is
equivalent to answering the question if we can generate the smaller group from
group elements of the larger one. That is, if there is no generating set of
Stab$(\rho)$ that \emph{does not} also generate Stab$(\sigma)$, we know that
Stab$(\rho)\not\leq$ Stab$(\sigma)$. Put differently, if there is an element in
the stabilizer of $\sigma$ that cannot be constructed with $\rho$ stabilizers
by means of the group operation, the condition fails.
It follows that we need a way to determine if the generating set of $\sigma$ is
contained in $\rho$.

One advantage of the tableau representation is that, while it is not a proper
\enquote{representation} in the strictest mathematical sense\footnote{In
representation theory, the group elements are mapped onto GL$(V)$ and the group
operation becomes matrix multiplication}, it does feature a way to represent
the group operation, namely \emph{rowsum}. The question to answer can then be
abstracted to the following. Given two stabilizer tableaus, corresponding to
$\sigma$ and $\rho$, respectively, can the tableau of $\sigma$ be transformed
to contain only $0$s using only rowsum$(s,r)$, where $r$ and $s$ are rows of
$\mathcal{T}_\rho$ and $\mathcal{T}_\sigma$, respectively?

Note that this question pertains to matrices and row manipulation thereof. What
we arrived at through this chain of arguments is Gaussian elimination of a
combined stabilizer tableau. If we write the stabilizer tableaus of $\rho$ and
$\sigma$ above one another and then perform Gaussian elimination on the
resulting tableau, we can deduce the subgroup property by checking if the rows
corresponding to $\sigma$ have non-zero entries. If they do, there is a
generator of $\sigma$ we couldn't construct from generators of $\rho$.
Consequently, the function should return \verb|false| in these cases and
\verb|true| for all zeros. The algorithm of this function is summarized in
\cref{alg:is-subgroup-of}.
\begin{alg}[Is subgroup of]\label{alg:is-subgroup-of}
  Let $\mathcal{T}_\sigma$ and $\mathcal{T}_\rho$ be tableaus of (possibly
  mixed) $N$-qubit stabilizer states $\sigma$ and $\rho$, respectively. W.l.o.g
  we say that we want to determine if Stab$(\sigma)\leq \mathrm{Stab}(\rho)$.

  First, construct a new auxiliary matrix $M$ as follows. Rows $1, \ldots,
  N-\verb|rho.mix|$ of $M$ correspond to the \enquote{non-mix} stabilizer
  generators in $\mathcal{T}_\rho$. Rows $(N-\verb|rho.mix|+1), \ldots,
  (2N-\verb|rho.mix|-\verb|sigma.mix|)$ correspond to the analogous rows of
  $\mathcal{T}_\sigma$.

  Next, perform a standard Gaussian elimination algorithm on $M$. If, at any
  point during the algorithm, the pivot is found in a row corresponding to
  $\mathcal{T}_\sigma$, return \verb|false|. 

  Finally, if the elimination algorithm finished without returning \verb|false|, the
  pivot was never found in a row corresponding to the $\sigma$ tableau. Thus,
  these rows are all $0$s, and the function returns \verb|true|.
\end{alg}
\subsubsection{Cross and relative entropy, ctd.}
With the subgroup check in place, we can tend to the computation of the other
entropic quantities. If the subgroup condition does \emph{not} hold,
the functions for the cross and relative entropy should return
$\infty$. With finite computational resources, representing an infinite value
is rather impossible. Thus, in the case where the subgroup check fails, these
functions return \verb|quiet_nan|s.

In the other case, we employ the results from \cref{sec:rel-ent-stab}, namely
\cref{thm:cross-ent-stab,col:rel-ent-stab} for the cross and relative entropy,
respectively.  The cross entropy between $\rho$ and $\sigma$ then simply
becomes the von Neumann entropy of $\sigma$, while the relative entropy becomes
the difference between \verb|sigma.mix| and \verb|rho.mix|.

\newpage
\subsection{Partial trace}\label{sec:ptrace}
The next item on the list is a function that implements the partial tracing
over a subsystem. The function itself is rather short, but relies on two
auxiliary functions, which hide the work required. We first introduce the
subroutines to then combine it into the full partial trace function.

\subsubsection{\texttt{get\_state\_type}}
The first auxilliary subroutine needed for
the partial trace algorithm we call \texttt{get\_state\_type}. It
takes a qubit position $a$ as input and outputs the number of unique stabilizer
generators minus one on that qubit. The name of the function stems from the
fact that we can have $3$ different state types: entangled, product and mixed.
A qubit with two unique stabilizer generators, i.e. $g_{ia} = Z$ and $g_{ja} =
X$ with $i\neq j$, will be in an entangled state with another qubit $b\neq a$.
If there is only one unique stabilizer generator, we have qubit $a$ in a
product state, where the state is the state stabilized by the generator.
Finally, no stabilizers correspond to a mixed state, since the empty set
generates the trivial group, which corresponds to a mixed state in the
stabilizer formalism.

The algorithm works by checking the $q$-th column for
each stabilizer in the tableau. This is then decoded the same way we encoded
the Pauli matrices in the tableau algorithm ($00 \equiv I$, $01\equiv Z$, $10
\equiv X$, $11 \equiv Y$) and (in case of a non-zero value) stored into a
variable \texttt{dummy}. If we have two differing non-zero values for our Pauli
encoding, we know our qubit to be in an entangled state with at least one other
qubit, and we return $1$. If there are no other generators, we return $0$ and
if \texttt{dummy} is $0$, qubit $a$ is in a mixed state. This algorithm is
formally written out in \cref{alg:get_state_type}. Since this is a novel
subroutine, which has no counterpart in the previous simulator, we have
included a flowchart representation of the algorithm, shown in
\cref{fig:statetype-diag}.
%represented as pseudocode in \cref{alg:get_state_type1}, and as a flowchart in
%\cref{fig:statetype-diag}.
%
%\textcolor{kw-olive}{hier die frage: was dient alles zum besseren
%verst\"andnis und was ist overload? pseudocode muss imo. nicht sein z.b.}

\begin{alg}[Determine state type]\label{alg:get_state_type}
  Let $a$ be the qubit we want to determine the \enquote{state type} of. First,
  set a dummy variable to $0$. Then, loop over non-mix stabilizers and compute
  $2x_{ia} + z_{ia}$. If this quantitiy is non-zero, set the dummy variable
  equal to it, then continue looping if necessary. If $2x_{ia}+z_{ia}$ is
  non-zero again, and not equal to \verb|dummy|, return 1. If the loop ends and
  \verb|dummy| is non-zero, return 0. If the loop breaks with
  $\texttt{dummy}=0$, return $-1$.
\end{alg}
\begin{figure}[H]
  \centering
  \begin{tikzpicture}[node distance=2cm, auto]

    % Nodes
  \node [startstop] (start) {Start};
  \node [block, below of=start] (init) {\verb|dummy| = 0};
  \node [process, below of=init] (loop) {Loop over stabilizers $i<(N-\verb|mix|)$};
  \node [decision, text width = 8em, below of=loop] (valnot0) {$2X_{ia} + Z_{ia} = \mathrm{val}
  \neq 0$?};
  \node [decision, below of=valnot0, yshift=-.5cm] (dummy0)
  {\verb|dummy| $=0$?};
  \node [decision, left of=valnot0, xshift=-.5cm] (loopend) {End of loop?};
  \node [block, left of=dummy0, xshift=-1.5cm] (setdummy) {\verb|dummy| $=2X_{ia} +
  Z_{ia}$};
  \node [decision, below of=dummy0, yshift=-.5cm] (dummypauli)
{\verb|dummy| $\neq\texttt{val}$?};
  \node [return, right of=dummypauli, xshift=1cm] (return1) {return 1};
  \node [decision, left of=setdummy, yshift=-4.5cm] (dummynot0)
  {\verb|dummy| $\neq 0$?};
  \node [return, right of=dummynot0, xshift=1cm] (ret0) {return 0};
  \node [return, below of=dummynot0, yshift=-.5cm] (ret-) {return $-1$};

    %\node (start) [startstop] {Start};
    %\node (init) [process, right of=start] {Init dummy, val = 0};
    %
    %\node (loop) [process, below of=init] {Loop: i = 0 to 2(N-mix)};
    %
    %\node (calc_val) [process, below of=loop] {Calculate val};
    %
    %\node (check_dummy_0) [decision, below of=calc_val] {val != 0 \& dummy == 0?};
    %\node (set_dummy) [process, right of=check_dummy_0, xshift=3cm] {Set dummy = val};
    %
    %\node (check_val_dummy) [decision, below of=check_dummy_0, yshift=-1cm] {val != 0 \& val != dummy?};
    %\node (return_1) [process, right of=check_val_dummy, xshift=3cm] {Return 1};
    %
    %\node (check_loop_end) [decision, below of=check_val_dummy, yshift=-1cm] {End of loop?};
    %
    %\node (check_dummy) [decision, below of=check_loop_end, yshift=-1cm] {dummy != 0?};
    %\node (return_0) [process, right of=check_dummy, xshift=3cm] {Return 0};
    %\node (return_neg1) [process, below of=check_dummy, yshift=-1cm] {Return -1};

    %\node (end) [startstop, below of=return_neg1, yshift=-1cm] {End};

    % Arrows
    \draw [arrow] (start) -- (init);
    \draw [arrow] (init) -- (loop);
    \draw [arrow] (loop) -- (valnot0);
    
    \draw [arrow] (valnot0) -- node[anchor=west] {yes} (dummy0);
    \draw [arrow] (valnot0) -- node[anchor=south] {no} (loopend);
    
    \draw [arrow] (dummy0) -- node[anchor=south] {yes} (setdummy);
    \draw [arrow] (dummy0) -- node[anchor=west] {no} (dummypauli);
    \draw [arrow-notip] (dummypauli) -| node[anchor=east] {no} (setdummy);
    \draw [arrow] (dummypauli) -- node[anchor=south] {yes} (return1);
    \draw [arrow] (setdummy) -- (loopend);
    
    \draw [arrow] (loopend) |- node[anchor=east] {no} (loop);
    \draw [arrow] (loopend) -| node[anchor=east] {yes} (dummynot0);
    
    \draw [arrow] (dummynot0) -- node[anchor=south] {yes} (ret0);
    \draw [arrow] (dummynot0) -- node[anchor=east] {no} (ret-);

\end{tikzpicture}

  \caption{flowchart for the \texttt{get\_state\_type} algorithm for qubit $a$.}
  \label{fig:statetype-diag}
\end{figure}
%\begin{algorithm}[H]
%\caption{Determine State Type for Qubits}
%\label{alg:get_state_type1}
%\begin{algorithmic}[1]
%\REQUIRE qubit $a$, total number of qubits $N$, tableau $\mathcal{T}$
%\ENSURE Returns 1, 0, or -1 based on the conditions
%\STATE $\verb|dummy| \leftarrow 0$
%\STATE $\verb|pauli| \leftarrow 0$
%\FOR{$i \leftarrow 0$ \TO $(N - mix)$}
%  \STATE $\verb|pauli| \leftarrow 2\cdot X_{ia} + Z_{ia}$
%  \IF{$\texttt{pauli} \neq 0$ \AND $\texttt{dummy} = 0$}
%        \STATE $\texttt{dummy} \leftarrow \texttt{pauli}$
%        \ELSIF{$\texttt{pauli} \neq 0$ \AND $\texttt{pauli} \neq \texttt{dummy}$}
%        \RETURN 1
%    \ENDIF
%\ENDFOR
%\IF{$\texttt{dummy} \neq 0$}
%    \RETURN 0
%\ELSE
%    \RETURN -1
%\ENDIF
%\end{algorithmic}
%\end{algorithm}
%
\subsubsection{\texttt{rowreduce}}
The next subroutine we expand the simulator with is \texttt{rowreduce}, which
is also vital to the \verb|ptrace| algorithm. Remember that the tableau
algorithm is based on the stabilizer generators and we already know that adding
two rows together, i.e. multiplying two generators, leaves the commutation
relations invariant. This means that we can perform row reduction to row
echelon form on our tableau without effect on the described state.

However,
some subtleties need to be taken into account. In principle it is possible to
row reduce the entire tableau. But it turns out that we need only to reduce the
columns associated with one particular qubit, when epmloyed as subroutine to
\verb|ptrace|. Next, we need to pay attention to the fact that our stabilizer
tableau has dimensions $n\times 2n$. Ideally, \verb|rowreduce| should modify
our stabilizers in a way that there are at most one of $X$ or $Z$ stabilizers
for our qubit. A natural first step would then be to treat the respective $X$
and $Z$ column separately. This is where one needs to be careful. Although the $X$
and $Z$ stabilizers are in separate columns, they share the rows, e.g. in the
case where $X_{ia}=Z_{ia}=1\equiv Y_{ia}$, meaning that
a reduction of $X$ will influence the $Z$ column and vice versa. One way to
reconcile this is to reduce the $X$ stabilizers first, swapping the row
containing $X$ to the bottom if necessary, then doing $Z$ stabilizers. That way
we ensure that we do not introduce $X$ stabilizers when adding rows together
to get rid of $Z$ stabilizers. This already hints to the next subtlety we need
to take into account.

A priori, one would probably perform gaussian
elimination to obtain an upper triangular form. The algorithm thereof is widely
studied and the plight of many computer science first year students. It would
thus be natural to assume that we want to have the reduced rows as first rows
of the matrix. Nevertheless, our case is different; since we exclusively call
\verb|rowreduce| to trace out a qubit, it will later prove convenient to have
the reduced rows on the \emph{bottom} of the tableau. This way, we can later
simply set \verb| N=N-1;|. We refer to the later discussion of the partial
trace function for a more in-depth explanation why this is done.

The last subtlety we want to highlight is the fact that with each modification
of the stabilizer generators, we need an appropriate modification of the
\emph{anti}stabilizers to keep the commutation relations of
\cref{prop:comm-tab} intact. Although we do not technially need to modify the
antistabilizers for our purposes, it is still necessary if we want to continue
applying the tableau algorithm on the rowreduced tableau. To this end, cf.
\cref{prop:comm-tab-2}, where this statement is formalized and proven.
\begin{prop}\label{prop:comm-tab-2}
  Let $\mathcal{T}$ be a tableau with stabilizer and antistabilizer generators
  $S=\langle g_1, \ldots, g_n \rangle$ and $A=\langle h_1, \ldots, h_n \rangle$
  respectively, where the generators fulfil the commutation relations of
  \cref{prop:comm-tab}. 
  Replacing $g_j$ by $g_i g_j$ in the stabilizer generators leaves
  \cref{prop:comm-tab} invariant if $h_i$ is replaced by $h_i h_j$ in the antistabilizer
  generators.
%  When replacing $g_j$ by $g_i g_j$ in the stabilizer generators, $h_i$ has to
%  be replaced by $h_i h_j$ to 
%  Modifying $S$ by multiplication of two generators $g_i$
%  and $g_j$ needs to be accompanied by an appropriate modification of $A$
\end{prop}
\begin{figure}[H]
  \centering
  \begin{tikzpicture}[node distance=2cm, auto]

    % Nodes
  \node [startstop] (start) {Start};
  \node [block, below of=start] (init) {\verb|first_x| = $-1$, \verb|first_z| =
  $-1$};
  \node [process, below of=init] (xloop) {Loop over stabilizers $g_i,\ i\leq N$};
  \node [decision, below of=xloop] (checkx) {$X_{ia} = 1$?};
  \node [decision, below of=checkx] (checkfirstx) {\verb|first_x| = $-1$?};
  \node [decision, left of=checkx, xshift=-.5cm] (loopend) {End of loop?};
  \node [process, text width = 10em, right of=checkfirstx, xshift=2.2cm] (rowsumx)
  {rowsum(i, first\_x)\\rowsum(first\_x+1, i+1)};
  \node [block, left of=checkfirstx, xshift=-1.5cm] (setfirstx) {\verb|first_x|
$=i$};
  \node [decision, left of=setfirstx, yshift=-3cm] (xswap)
  {$-1 < \verb|first| < 2N$?};
  \node [block, right of=xswap, xshift=2cm] (swap) {swap \verb|first_x| to
  bottom};
  \node [process, below of=xswap, yshift=-.5cm] (repeat) {repeat loop with $Z$};
   
    %\node (start) [startstop] {Start};
    %\node (init) [process, right of=start, xshift=1cm] {Init variables};
    %
    %\node (x_loop) [process, right of=init, xshift=1.5cm] {Loop over qubit $a$};
    %\node (check_first_x) [decision, below of=x_loop] {$X_{ia}=1$?};
    %\node (set_first_x) [process, right of=check_first_x, xshift=2.5cm] {Set first\_x, inc num\_x};

    %\node (check_other_x) [decision, below of=check_first_x, yshift=-2cm] {Other x stabilizer?};
    %\node (rowsum_x) [process, right of=check_other_x, xshift=2.5cm] {rowsum};

    %\node (cond_x_swap) [decision, below of=check_other_x, yshift=-1cm] {x swap needed?};
    %\node (rowswap_x) [process, right of=cond_x_swap, xshift=2.5cm] {rowswap};

    %\node (z_loop) [process, below of=cond_x_swap, yshift=-1cm] {Loop over z stabilizers};
    %\node (check_first_z) [decision, below of=z_loop] {First z stabilizer?};
    %\node (set_first_z) [process, right of=check_first_z, xshift=2.5cm] {Set first\_z, inc num\_z};

    %\node (check_other_z) [decision, below of=check_first_z, yshift=-1cm] {Other z stabilizer?};
    %\node (rowsum_z) [process, right of=check_other_z, xshift=2.5cm] {rowsum};

    %\node (cond_z_swap) [decision, below of=check_other_z, yshift=-1cm] {z swap needed?};
    %\node (rowswap_z) [process, right of=cond_z_swap, xshift=2.5cm] {rowswap};

    %\node (end) [startstop, below of=cond_z_swap, yshift=-1cm] {End};

    % Arrows
    \draw [arrow] (start) -- (init);
    \draw [arrow] (init) -- (xloop);
    \draw [arrow] (xloop) -- (checkx);
    
    \draw [arrow] (checkx) -- node[anchor=west] {yes} (checkfirstx);
    \draw [arrow] (checkx) -- node[anchor=south] {no} (loopend);
    
    \draw [arrow] (checkfirstx) -- node[anchor=south] {yes} (setfirstx);
    \draw [arrow] (checkfirstx) -- node[anchor=south] {no} (rowsumx);
    \draw [line,dashed,gray] (rowsumx) -- (loopend);
    \draw [arrow] (setfirstx) -- (loopend);
    
    \draw [arrow] (loopend) |- node[anchor=east] {no} (loop);
    \draw [arrow] (loopend) -| node[anchor=east] {yes} (xswap);
    
    \draw [arrow] (xswap) -- node[anchor=south] {yes} (swap);
    \draw [arrow] (xswap) -- node[anchor=east] {no} (repeat);
    \draw [arrow] (swap) |- (repeat);
    
    %\path [line] (start) -- (init);
    %\path [line] (init) -- (x_loop);
    %\path [line] (x_loop) -- (check_first_x);
    %\path [line] (check_first_x) -- node[anchor=south] {yes} (set_first_x);
    %\path [line] (check_first_x) |- node[anchor=east] {no} (x_loop);
    %\path [line] (check_other_x) -- node[anchor=south] {yes} (rowsum_x);
    %\path [line] (check_other_x) -- node[anchor=east] {no} (cond_x_swap);
    %\path [line] (cond_x_swap) -- node[anchor=south] {yes} (rowswap_x);
    %\path [line] (cond_x_swap) -- node[anchor=east] {no} (z_loop);

    %\path [line] (z_loop) -- (check_first_z);
    %\path [line] (check_first_z) -- node[anchor=south] {yes} (set_first_z);
    %\path [line] (check_first_z) -- node[anchor=east] {no} (check_other_z);
    %\path [line] (check_other_z) -- node[anchor=south] {yes} (rowsum_z);
    %\path [line] (check_other_z) -- node[anchor=east] {no} (cond_z_swap);
    %\path [line] (cond_z_swap) -- node[anchor=south] {yes} (rowswap_z);
    %\path [line] (cond_z_swap) -- node[anchor=east] {no} (end);

    %\path [line] (set_first_x) |- (check_other_x);
    %\path [line] (rowsum_x) |- (cond_x_swap);

    %\path [line] (set_first_z) |- (check_other_z);
    %\path [line] (rowsum_z) |- (cond_z_swap);

    %\path [line] (rowswap_x) |- (z_loop);
    %\path [line] (rowswap_z) -- (end);

\end{tikzpicture}

  \caption{Flowchart representation of the rowreduce subroutine}
  \label{fig:rowreduce-diag}
\end{figure}
\begin{proof}[Proof of \cref{prop:comm-tab-2}]
  We will first recall the invariants as given in \cref{prop:comm-tab}.
  They read
  \begin{enumerate}
    \item $R_{n+1},\ldots,R_{2n}$ generate $S(\ket{\phi})$, and $R_1, \ldots,
      R_{2n}$ generate $\mathcal{P}_n$.
    \item $R_1, \ldots, R_n$ commute.
    \item $\forall h \in \{1,\ldots,n\}, \ \{R_h, R_{h+n}\} = 0$
    \item $\forall i,h \in \{1,\ldots,n\}, \ \text{with } i\neq h, \ [R_i, R_{h+n}] = 0$
  \end{enumerate}
  We prove the statement by showing that each of the above points still hold
  for the proposed modifications.
  \begin{enumerate}
    \item Since we merely multiplied generators, this holds by group theoretic
      arguments
    \item All of the antistabilizers did commute previously, therefore, their
      product commutes as well
    \item There are only two relations of relevance for this point
      \[
        \{h_i h_j, g_i\} \overset{\text{!}}{=} 0 \quad{\text{and}} \quad 
        \{h_j, g_i g_j\} \overset{\text{!}}{=} 0, 
      \]
      since all the other generators are left as they were.
      To show this anticommutation relation we employ the well-known identity
      \[ \{AB,C\} = A[B,C] + \{A,C\}B \]
      to obtain
      \begin{align*}
        \{h_i h_j, g_i \} &= h_i \underbrace{[h_j, g_i]}_{=0} +
        \underbrace{\{h_i, g_i\}}_{=0} h_j = 0 \qquad{\text{and}} \\
        \{h_j, g_i g_j \} &= \{g_i g_j, h_j\} = g_i \underbrace{[g_j, h_i]}_{=0} +
        \underbrace{\{g_i, h_i\}}_{=0} g_j = 0
      .\end{align*}
    \item As we started from a valid tableau, fulfilling the commutation
      relations, we need to verify this point only for one of the combinations,
      namely \[ [\tilde{g}_j, \tilde{h}_i] = [ g_i g_j, h_i h_j]. \]
      This is done with another commutator identiy,
      \[
        [AB,CD] = A[B,C]D + [A,C]BD + CA[B,D] + C[A,D]B.
      \]
      We thus have
      \begin{align*}
        [g_i g_j, h_i h_j] &=
          g_i \underbrace{[g_j, h_i]}_{=0} h_j + [g_i, h_i] g_j h_j + h_i g_i
          [g_j, h_j] + h_i \underbrace{[g_i, h_j]}_{=0} g_j \\
           &= (g_i h_i - h_i g_i)g_j h_j + h_i g_i (g_j h_j - h_j g_j) \\
           &= g_i h_i g_j h_j - h_i g_i g_j h_j + h_i g_i g_j h_j - h_i g_i h_j
           g_j \\
           &= \underbrace{\{g_i, h_i\}}_{=0} g_j h_j - h_i g_i
           \underbrace{\{g_j, h_j\}}_{=0} \\
           &= 0
      .\end{align*}
  \end{enumerate}
\end{proof}

With all the subtleties accounted for, we can begin to construct
\verb|rowreduce|. It should take a qubit position as input, and return
\verb|void|, since it merely modifies the matrix. We start by looping through
the $X$ stabilizers. The first time where $X_{ia}=1$, the row number $i$ is
stored in a variable \verb|first_x = i|. For each subsequent row with $X_{ka}=1$
we call \verb|rowsum(k,first_x)| and \verb|rowsum(first_x+1,k+1)|, where
\verb|h+1| is the associated antistabilizer to stabilizer \verb|h|. After
looping through the $X$ stabilizers, we move row \verb|first_x| to the bottom
if necessary. We then repeat the previous procedure with the $Z$ stabilizers,
also moving \verb|first_z| to the bottom if necessary. 
The rowreduce algorithm is summarized in \cref{alg:rowreduce} and represented
as a flowchart in \cref{fig:rowreduce-diag}.
\begin{alg}[Rowreduce]\label{alg:rowreduce}
  Let $a$ denote the qubit we want to reduce. First, set two helper variables
  \verb|first_x| and \verb|first_z| to $-1$. Then loop over all rows in the $X$
  matrix of the stabilizers. If $X_{ia} = 1$ for the first time, set
  \verb|first_x| to $i$. For any subsequent rows $j$ with $X_{ja} = 1$, call
  rowsum($j$, \verb|first_x|) and the dual in the antistabilizers according to
  \cref{prop:comm-tab-2}. Once the loop hits the end, swap row $i$ to the
  bottom of the tableau, depending on if \verb|first_x| $\neq-1$. Then repeat
  the previous steps for the $Z$ matrix.
\end{alg}
%A pseudocode
%representation of this algorithm is provided in \cref{alg:rowreduce}.

%\begin{algorithm}[H]
%\caption{Rowreduce stabilizers for qubit $a$}
%\label{alg:rowreduce}
%\begin{algorithmic}[1]
%\REQUIRE qubit $a$, total number of qubits $N$, tableau $\mathcal{T}$
%\STATE $\verb|first_x| \leftarrow -1$
%\FOR{$i \leftarrow 0$ \TO $2N$ \textbf{step} 2}
%\IF{$X_{ia} = 1$ \AND $\texttt{first\_x} = -1$}
%    \STATE $\verb|first_x| \leftarrow i$
%    \ELSIF{$X_{ia} = 1$ \AND $\texttt{first\_x} > -1$}
%    \STATE \verb|rowsum(i, first_x)|
%    \STATE \verb|rowsum(first_x+1, i+1)|
%  \ENDIF
%\ENDFOR
%\IF{ $-1 < \texttt{first\_x} < 2N$}
%  \FOR{$i \leftarrow$ \texttt{first\_x} \TO $2N$}
%    \STATE \verb|rowswap(i, i+2)|
%  \ENDFOR
%\ENDIF
%\STATE $\verb|first_z| \leftarrow -1$
%\FOR{$i \leftarrow 0$ \TO $2N$ \textbf{step} 2}
%\IF{$X_{ia} = 1$ \AND $\texttt{first\_z} = -1$}
%    \STATE $\verb|first_z| \leftarrow i$
%    \ELSIF{$X_{ia} = 1$ \AND $\texttt{first\_z} > -1$}
%    \STATE \verb|rowsum(i, first_z)|
%    \STATE \verb|rowsum(first_z+1, i+1)|
%  \ENDIF
%\ENDFOR
%\IF{ $-1 < \texttt{first\_z} < 2N$ \OR \texttt{first\_x} = $-1$}
%  \FOR{$i \leftarrow$ \texttt{first\_z} \TO $2N$}
%    \STATE \verb|rowswap(i, i+2)|
%  \ENDFOR
%\ENDIF
%%\STATE $\verb|first_z| \leftarrow -1$
%%\FOR{$i \leftarrow 0$ \TO $2N$ \STEP 2}
%%  \IF{$X_{ia} = 1$ \AND \verb|first_z| = $-1$}
%%    \STATE $\verb|first_x| \gets i$
%%  \ELSIF{$X_{ia} = 1$ \AND \verb|first_z| > $-1$}
%%    \STATE \verb|rowsum(i, first_z)|
%%    \STATE \verb|rowsum(first_z+1, i+1)|
%%  \ENDIF
%%\ENDFOR
%%\IF{$-1 < \verb|first_z| < 2N$ \OR \verb|first_x| = $-1$}
%%  \FOR{$i \gets \verb|first_z| \TO 2N$}
%%    \STATE \verb|rowswap(i, i+2)|
%%  \ENDFOR
%%\ENDIF
%\end{algorithmic}
%\end{algorithm}



\subsubsection{Partial trace}
We can now combine the two previous subroutines to an algorithm realizing the
partial trace, also refered to as \verb|ptrace| for short. The algorithm traces
out one qubit and modifies the remaining stabilizers accordingly. For instance,
tracing out a qubit in a product state will just remove this qubit, since it
doesn't correlate with any other qubit. If we do have correlations in the form
of entanglement, the algorithm will modify the remaining stabilizers in a way
that increases the mixedness. 

Consider the paradigmatic example of the two-qubit Bell state
\[
  \ket{\phi} = \frac{\ket{00} + \ket{11}}{\sqrt{2}} \quad{\text{with density
  matrix}} \quad \rho = \dyad{\phi}.
\]
After tracing out any of the two qubits we are left with a mixed state of the
form
$$\rho_i = \frac{1}{2} \left[\dyad{0}_i + \dyad{1}_i\right],$$
which is the maximally mixed state for qubit $i$. The analogous operation in
the stabilizer picture is starting with the state stabilized by
Stab$(\ket{\phi}) = \langle ZZ, XX \rangle$. Then, by means of partially
tracing out one qubit arriving at Stab$(\rho_i) = \langle \rangle = \{ I \}$.

Conversely, if we start out in a product state, such as $\ket{\phi} =
\ket{00}$, and then trace out one qubit, the state of the qubit is still pure.
Thus, the decrease in purity after the partial trace operation depends on the
correlations of the qubit to be traced out. This is what we need the
\verb|get_state_type| function for. One of the steps in the partial trace
function should be to increase or decrease the \verb|mix| attribute by an
amount depending on the state type. The value of this alteration is exactly
the return value of \verb|get_state_type| for the qubit to be traced out. With
that we have a sensible first step in the \verb|ptrace| algorithm, namely
calling \verb|get_state_type| and updating \verb|mix| accordingly.

The next step is to move the columns corresponding to qubit $a$ to position
$N$. The reason for this is the same as why we moved the rows to the bottom
in the \verb|rowreduce| subroutine (cf. \cref{alg:rowreduce}). Since we
decrease the system size by $1$, we set $N$ to $N-1$, and since the tableau
dimensions depend on this $N$ explicitly, moving a column to $N$ and then
decrementing $N$ by $1$ amounts to deleting the column. 

Then, we call rowreduce on the last column. After rowreduce has been called,
the stabilizer generators of qubit $a$ are in column $N$ and row $N$, or
possibly $N-1$ too.

Finally, we decrement $N$, thereby removing the
qubit from the system. The \verb|ptrace| function is summarized in
\cref{alg:ptrace}.
\begin{alg}[Partial trace]\label{alg:ptrace}
  Let $a$ be the qubit to be traced out. First, call \verb|get_state_type| (cf.
  \cref{alg:get_state_type}) and add its return value to \verb|mix|. Then, move
  column $a$ to $N$ by means of transposition. Next, call rowreduce on column
  $N$. Finally, decrement $N$ by 1.
\end{alg}
%\begin{figure}[H]
%  \centering
%  \begin{tikzpicture}[node distance=2cm]

    % Nodes
    \node (start) [startstop] {Start};
    \node (assertions) [process, below of=start] {Check assertions (q, N)};
    
    \node (init_helper) [process, below of=assertions] {Init helper\_tab, last\_stab, last\_anti};
    
    \node (move_qubit_loop) [process, below of=init_helper] {Loop: Move qubit q to the end};
    
    \node (update_mix) [process, below of=move_qubit_loop] {Update mix};
    
    \node (rowreduce) [process, below of=update_mix] {Call rowreduce(N-1)};
    
    \node (decrement_n) [process, below of=rowreduce] {Decrement N};
    
    \node (end) [startstop, below of=decrement_n] {End};

    % Arrows
    \draw [arrow] (start) -- (assertions);
    \draw [arrow] (assertions) -- (init_helper);
    \draw [arrow] (init_helper) -- (move_qubit_loop);
    \draw [arrow] (move_qubit_loop) -- (update_mix);
    \draw [arrow] (update_mix) -- (rowreduce);
    \draw [arrow] (rowreduce) -- (decrement_n);
    \draw [arrow] (decrement_n) -- (end);

\end{tikzpicture}

%  \caption{flowchart for ptrace algorithm}
%  \label{fig:ptrace-dig}
%\end{figure}

\subsection{Classical control of mixedness}\label{sec:other-dingers}
The last point we want to discuss is the control we wield over the quantum
simulation and the possible algorithms we can conceive to artificially
introduce mixedness to our system. The algorithms used in
\cref{sec:maximal-mixing,sec:minimal-mixing} were realized
by selectively choosing the appropriate qubits to be cast to a mixed state.
We therefore conclude the presentation of the algorithms added to the simulator
by explaining how they function.

\subsubsection{Minimal mixing}
The first algorithm introduced is the more involved of the two. Here, we
specifically choose a generator to be discarded such that the subgroup
condition (should) be met in principle. The physics of the algorithm (and why
it is a rather unphysical one) is discussed in detail in \cref{sec:minimal-mixing}.
Here, we only concern ourselves with the algorithm as it is implemented in the
simulator.

The basic idea of the algorithm is similar to the projection function, but
instead of returning \verb|true| or \verb|false| if a projection is successful
or not, we do some more work on the tableau after the default projection
algorithm would return \verb|false|. In a sense, it's a softer version of a
forceful projection, as it is done in the naive approach (cf.
\cref{sec:naive-approach}. Once we detect a faulty projection, we ensure that
the measurement operator is included with the correct sign. This is done by
imitating the case of a measurement with random outcome. In a way, we pretend
that the projection has to alter the tableau (which is has to anyway) and
insert the correct generator. We want to emphasize that one needs to take
\cref{prop:comm-tab-2} into account when altering the tableau in this manner.
Not being careful in this step could result in a tableau containing $-I$ as
generator, which is not allowed.

After the correct generator is in place, we
move it to the \enquote{mix} generators and increment \verb|mix| by 1. The
whole algorithm is summarized in \cref{alg:minimal-mixing}

\begin{alg}[Minimal mixing]\label{alg:minimal-mixing}
  The algorithm works the same way as \cref{alg:projection} with one key
  exception. When a projection fails, the pertinent generator gets assigned the
  correct sign by doing the steps in case 1 of \cref{alg:tab-measure}. Then,
  swap rows such that the generator is with other mix generators, then incement
  mix by 1.
\end{alg}

\subsubsection{Maximal mixing}
The second algorithm is rather simple. As we discard the entire generating set
once a projection fails, there is no complex manipulation of the tableau in the
subroutine. Instead, we set \verb|mix=N| once the measurement outcome in the
record does not match the expectation. For completeness' sake,
\cref{alg:maximal-mixing} summarizes the steps.

\begin{alg}[Maximal mixing]\label{alg:maximal-mixing}
  The algorithm works the same way as \cref{alg:projection} with a contingency
  added in case of failed projection. If a projection fails, set \verb|mix=N|.
\end{alg}

\section{Summary}

In this chapter we introduced a simulation algorithm used to simulate a wide
class of quantum circuits, known as Clifford circuits. We gave an overview of
the existing infrastructure present to perform numerical experiments of quantum
computations on a classical platform using the stabilizer formalism. We then
derived algorithms for new functionalities employed in the simulations in the
remainder of the thesis. In particular, we introduced the simulability of mixed
states by extending the previous functions and subroutines to handle this class
of quantum states appropriately. We furthermore introduced new functions that
only become well-defined on mixed states.

%\subsection{sample flowchart!}
%delete this subsection later!
%\begin{figure}[H]
%  \centering
%  \input{fig/tikz/example.tex}
%  \caption{dings}
%  \label{fig:asdf}
%\end{figure}
