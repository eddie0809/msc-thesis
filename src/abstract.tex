\chapter{Abstract}
\label{ch:intro}

\begin{itemize}
  \item Entanglement transitions are cool
  \item Entanglement is backbone of quantum computing
  \item Something something quantum error correction
  \item Entanglement exponentially hard to measure (requires exponentially many
    copies of state)
  \item Classical simulations predict phase transition in many hybrid random
    circuits
  \item PTIM \cite{langEntanglementTransitionProjective2020} measurement only
    circuit, that maps to majorana fermions, the colored cluster model and
    2-dimensional percolation
  \item 2D percolation and the CCM has been widely studied, from this we
    predict there to be the quantum phase transition from one area law to the
    other
  \item \textcolor{red}{check das mit dem area law ding auf jeden fall nochmal,
    ich glaub da wird schon auch wert drauf gelegt}
  \item No way to measure entanglement in experiment, since we can only measure
    observables
  \item Many attempts of probing the critical point of the phase transitions
    have been made
  \item Fisher's LXE: \cite{liCrossEntropyBenchmark2023}
  \item Fisher's LXE for PTIM:
    \cite{tikhanovskayaUniversalityCrossEntropy2023}
  \item In-House attempts also exist:
  \item The decoder in \cite{roserDecodingProjectiveTransverse2023} lower
    bounds the critical point (decoding threshhold)
  \item Can we upper bound it?
  \item In principle yes, klein's inequality (goes by \enquote{gibb's
    inequality} and \enquote{information inequality} as well) upper bounds
    von Neumann, and consequently entanglement entropy.
  \item In \cite{garrattProbingPostmeasurementEntanglement2023} this is done
    with hybrid circuits.
\end{itemize}


\lipsum[0-1]
