\chapter{Zusammenfassung}
Eine der faszinierendsten Ideen in der Quantenmechanik ist das Konzept der
Verschränkung. Dieses ist bekannt für seine zentrale Rolle in der
Quanteninformation sowie für seine verblüffenden
Eigenheiten, die unsere Vorstellungen von der Natur herausfordern. Allgemein
entsteht Verschränkung durch die Wechselwirkung von Quantensystemen mit ihrer
Umgebung. In kontrollierten Umgebungen, wie etwa Experimenten, wird die
Verschränkung zweier Systeme üblicherweise durch die Anwendung unitärer
Wechselwirkungen erreicht.

Ein zentrales Merkmal dieser nichtlokalen Korrelationen ist die Tatsache, dass
das Messen eines Bestandteils eines verschränkten Zustands einen Teil der
zugrunde liegenden Verschränkungsstruktur zerstört. Daher kann das Wachstum der
Verschränkung im Labor durch die Anwendung von unitären Operationen gehemmt
werden, wenn Teile des Systems lokal gemessen werden. Schaltkreise, die durch
das Zusammenspiel von unitären Operationen und lokalen Messungen gekennzeichnet
sind, werden als hybride Schaltkreise bezeichnet.

In den letzten zehn Jahren hat die Untersuchung solcher hybrider Schaltkreise
faszinierende Muster in der Dynamik der Verschränkung offengelegt. Bei
hinreichend häufigen Messungen zeigt sich, dass der Grad der Verschränkung
zwischen Teilsystemen einem Flächengesetz folgt. Im Gegensatz dazu ergibt sich
bei seltenen Messungen eine Skalierung nach einem Volumengesetz.
Bemerkenswerterweise gibt es einen Phasenübergang zwischen diesen beiden
Regimen, der durch eine kritische Messrate gekennzeichnet ist. Diese Übergänge
werden als messungsinduzierte Verschränkungsphasenübergänge oder kurz
Verschränkungsübergänge bezeichnet.

Verschränkungsübergänge wurden nicht nur in hybriden Schaltkreisen
theoretisiert, sondern auch in Konfigurationen, bei denen die (diskrete)
Zeitentwicklung allein durch projektive Messungen bestimmt wird, wobei
paarweise und lokale Messungen konkurrieren. Ein paradigmatisches Modell mit
einem solchen Übergang ist das projektive transversale Ising-Modell, das einen
Übergang zwischen zwei Flächengesetz-Phasen zeigt.

Die experimentelle Detektion von Verschränkungsübergängen erweist sich jedoch
als schwierige, scheinbare Sisyphusaufgabe, was auf das sogenannte
Postselektionsproblem---auch als Sampling-Problem bezeichnet---zurückzuführen ist.
Dieses Problem besagt, dass es aufgrund der probabilistischen Natur von
Quantenmessungen exponentiell unwahrscheinlich ist, denselben Zustand zweimal
zu erhalten, selbst wenn das identische Protokoll wiederholt angewendet wird.
Dies verschleiert die Natur des Zustands bei der Untersuchung, deren Kenntnis
erforderlich ist, um den Grad der Verschränkung zu quantifizieren.

Kürzlich wurden einige Protokolle vorgeschlagen, um das Sampling-Problem zu
umgehen. Insbesondere schlugen \citeauthor{liCrossEntropyBenchmark2023} eine
Größe vor, die als Ordnungsparameter für den messungsinduzierten Phasenübergang
in hybriden Schaltkreisen dient, die sogenannte lineare Kreuzentropie. Außerdem
schlugen \citeauthor{garrattProbingPostmeasurementEntanglement2023} eine
alternative Methode vor, die eine obere Schranke aus der
Quanteninformationstheorie nutzt, bekannt als Kleinsche Ungleichung.

In dieser Arbeit werden diese Protokolle im Kontext des projektiven
transversalen Ising-Modells getestet. Die Arbeit besteht aus vier Hauptkapiteln
und ist wie folgt aufgebaut:
\begin{description}
\item[\cref{ch:basics}] In diesem Kapitel werden die für die Arbeit relevanten
  Konzepte eingeführt. Wir beginnen mit einer Einführung in den
  Stabilizer-Formalismus, da dieser eine nützliche Darstellung von
  Quantenzuständen bietet und die Grundlage für den Simulationsalgorithmus
  bildet, der für die numerischen Ansätze in der Arbeit verwendet wird.
  Anschließend geben wir einen Überblick über das Konzept der
  Verschränkungsübergänge im Allgemeinen, bevor wir das projektive transversale
  Ising-Modell speziell einführen. Dieses Modell ist das Hauptobjekt unserer
  Untersuchungen. Abschließend wird das Postselektions- bzw. Sampling-Problem
  erörtert, welches den weiteren Verlauf der Arbeit motiviert.
\item[\cref{ch:lxe}] In diesem Kapitel untersuchen wir die von
  \citeauthor{liCrossEntropyBenchmark2023} definierte Größe im Kontext des
  projektiven transversalen Ising-Modells. Besonders wird die Motivation hinter
  Postselektionsalgorithmen beleuchtet, und unsere Ergebnisse werden mit
  früheren Untersuchungen ähnlicher Natur (vgl. Ref.
  \cite{tikhanovskayaUniversalityCrossEntropy2023}) verglichen, wobei auch
  Kritikpunkte aufgezeigt werden.
\item[\cref{ch:rel-ent}] In diesem Kapitel untersuchen wir den Ansatz von
  \citeauthor{garrattProbingPostmeasurementEntanglement2023} zur Detektion von
  Verschränkungsübergängen im projektiven transversalen Ising-Modell. Zunächst
  leiten wir einen Ausdruck für die Kreuz- und relative Entropie im
  Stabilizer-Formalismus her, bevor wir verschiedene numerische Ansätze testen,
  um das Postselektionsproblem zu umgehen. Abschließend werden zwei
  Regularisierungsansätze vorgestellt, um auftretende Divergenzen zu
  bewältigen.
\item[\cref{ch:mixed}] In diesem Kapitel wird der Simulationsalgorithmus
  eingeführt, der auf dem in \cref{ch:basics} beschriebenen
  Stabilizer-Formalismus basiert. Zunächst geben wir eine kurze Zusammenfassung
  bestehender Algorithmen. Anschließend werden die Erweiterungen erläutert, die
  für die numerischen Simulationen in der Arbeit erforderlich sind,
  insbesondere in Bezug auf gemischte Zustände und die in Kapitel 3
  abgeleiteten Entropien.
\end{description}
