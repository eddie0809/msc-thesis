\chapter{Implementing mixed states}
\label{ch:mixed}
\epigraph{Always programming a new type of antidote in your perimeter}{
\citeauthor{quasimotoDiscipline99Pt2000}
%on \citefield{quasimotoDiscipline99Pt2000}{title}
}

Ref. \cite{aaronsonImprovedSimulationStabilizer2004} will need to do some heavy
lifting in giving primers on stabilizer simulation.

It will later prove useful to implement mixed states into the existing
stabilizer simulator. As of yet, this does not exist.

In this paper \cite{audenaertEntanglementMixedStabilizer2005} they present an
algorithm to perform a partial trace on a stabilizer state. They need an
auxilliary algorithm, the Reduced row echelon form algorithm. I think this is a
promising approach, but needs implementation of the RREF and the ptrace
algorithm. From what I've seen, the first one is going to take longer than the
second to implement. Maybe look for this kind of alg in the existing codebase?

Note from 10 minutes later: its basically what we already had, only a bit more
formalized. 

Lecture notes on stabilizers: \cite{arabLectureNotesQuantum2024}

\section{Expanding the Tableau algorithm}
Tableau algorithm: \cite{aaronsonImprovedSimulationStabilizer2004}

\section{The algorithms}
\textcolor{red}{Hier w\"are es ganz nice f\"ur die ganzen algs so flowcharts
dabei zu haben, u.a. auch um listings aus dem weg zu gehen. Alg als flowchart
ist generischer als der (evtl unsauber geschriebene) \texttt{C++} code}

\subsection{\texttt{rowreduce}}

\subsection{\texttt{get\_state\_type}}

\subsection{\texttt{ptrace}}

\subsection{\texttt{is\_subgroup\_of}}

\subsection{\texttt{project\_or\_mix}}

\subsection{\texttt{force\_projection}}
