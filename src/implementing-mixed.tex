\chapter{Implementing mixed states}
\label{ch:mixed}
\epigraph{Always programming a new type of antidote in your perimeter}{
\citeauthor{quasimotoDiscipline99Pt2000}
%on \citefield{quasimotoDiscipline99Pt2000}{title}
}

Ref. \cite{aaronsonImprovedSimulationStabilizer2004} will need to do some heavy
lifting in giving primers on stabilizer simulation.

It will later prove useful to implement mixed states into the existing
stabilizer simulator. As of yet, this does not exist.

In this paper \cite{audenaertEntanglementMixedStabilizer2005} they present an
algorithm to perform a partial trace on a stabilizer state. They need an
auxilliary algorithm, the Reduced row echelon form algorithm. I think this is a
promising approach, but needs implementation of the RREF and the ptrace
algorithm. From what I've seen, the first one is going to take longer than the
second to implement. Maybe look for this kind of alg in the existing codebase?

Note from 10 minutes later: its basically what we already had, only a bit more
formalized. 

Lecture notes on stabilizers: \cite{arabLectureNotesQuantum2024}

This chapter serves as a primer on the nitty-gritty details of simulating a
stabilizer circuit on a classical computer. As already touched on in
\cref{ch:basics}, according to the \emph{Gottesman-Knill Theorem} we can
perform efficient classical simulations of quantum circuits using only clifford
gates. In the following sections we will introduce the so-called tableau
algorithm, which allows simulations in ploynomial time, and expand it to
properly deal with mixed states. We assume the reader to be familiar with
the relevant sections in \cref{sec:stab-basics}

\section{The Tableau Algorithm}\label{sec:tableau}
In this section we give a brief introduction to the tableau algorithm. It can
be seen as the sequel to \cref{sec:stab-basics}. Note that this section is
based on the work of \citeauthor{aaronsonImprovedSimulationStabilizer2004} in
\cite{aaronsonImprovedSimulationStabilizer2004}.

We already know that we do not need the full stabilizer group to describe our
state. We rather specify the set, which generates the group. This reduces the
number of operators that need to be saved to $n=\log_{2}2^n$. Since stabilizer
groups are built from Pauli matrices, we need $2n+1$ bits to specify 

The name of the algorithm stems from the fact that a state is represented by a
\emph{tableau}. This tableau is a binary $2n\times2n+1$ matrix. 

One subroutine of major importance to the tableau algorithm is \texttt{rowsum}.

\subsection{\texttt{rowsum}}\label{sec:rowsum}

\section{Expanding the Tableau algorithm}
Tableau algorithm: \cite{aaronsonImprovedSimulationStabilizer2004}

\textcolor{red}{Hier w\"are es ganz nice f\"ur die ganzen algs so flowcharts
dabei zu haben, u.a. auch um listings aus dem weg zu gehen. Alg als flowchart
ist generischer als der (evtl unsauber geschriebene) \texttt{C++} code}

The tableau algorithm as outlined in \cref{sec:tableau} is only able to
simulate stabilizer circuits on pure states. This however, neglects the most
general class of quantum states, namely, mixed states. It will become apparent
in \cref{ch:rel-ent} why this extension is necessary for our purposes. 

\subsection{\texttt{rowreduce}}
The first subroutine we expand the simulator with is \texttt{rowreduce}. 

%\begin{figure}[h]
%  \centering
%  \begin{tikzpicture}[node distance=2cm, auto]

    % Nodes
  \node [startstop] (start) {Start};
  \node [block, below of=start] (init) {\verb|first_x| = $-1$, \verb|first_z| =
  $-1$};
  \node [process, below of=init] (xloop) {Loop over stabilizers $g_i,\ i\leq N$};
  \node [decision, below of=xloop] (checkx) {$X_{ia} = 1$?};
  \node [decision, below of=checkx] (checkfirstx) {\verb|first_x| = $-1$?};
  \node [decision, left of=checkx, xshift=-.5cm] (loopend) {End of loop?};
  \node [process, text width = 10em, right of=checkfirstx, xshift=2.2cm] (rowsumx)
  {rowsum(i, first\_x)\\rowsum(first\_x+1, i+1)};
  \node [block, left of=checkfirstx, xshift=-1.5cm] (setfirstx) {\verb|first_x|
$=i$};
  \node [decision, left of=setfirstx, yshift=-3cm] (xswap)
  {$-1 < \verb|first| < 2N$?};
  \node [block, right of=xswap, xshift=2cm] (swap) {swap \verb|first_x| to
  bottom};
  \node [process, below of=xswap, yshift=-.5cm] (repeat) {repeat loop with $Z$};
   
    %\node (start) [startstop] {Start};
    %\node (init) [process, right of=start, xshift=1cm] {Init variables};
    %
    %\node (x_loop) [process, right of=init, xshift=1.5cm] {Loop over qubit $a$};
    %\node (check_first_x) [decision, below of=x_loop] {$X_{ia}=1$?};
    %\node (set_first_x) [process, right of=check_first_x, xshift=2.5cm] {Set first\_x, inc num\_x};

    %\node (check_other_x) [decision, below of=check_first_x, yshift=-2cm] {Other x stabilizer?};
    %\node (rowsum_x) [process, right of=check_other_x, xshift=2.5cm] {rowsum};

    %\node (cond_x_swap) [decision, below of=check_other_x, yshift=-1cm] {x swap needed?};
    %\node (rowswap_x) [process, right of=cond_x_swap, xshift=2.5cm] {rowswap};

    %\node (z_loop) [process, below of=cond_x_swap, yshift=-1cm] {Loop over z stabilizers};
    %\node (check_first_z) [decision, below of=z_loop] {First z stabilizer?};
    %\node (set_first_z) [process, right of=check_first_z, xshift=2.5cm] {Set first\_z, inc num\_z};

    %\node (check_other_z) [decision, below of=check_first_z, yshift=-1cm] {Other z stabilizer?};
    %\node (rowsum_z) [process, right of=check_other_z, xshift=2.5cm] {rowsum};

    %\node (cond_z_swap) [decision, below of=check_other_z, yshift=-1cm] {z swap needed?};
    %\node (rowswap_z) [process, right of=cond_z_swap, xshift=2.5cm] {rowswap};

    %\node (end) [startstop, below of=cond_z_swap, yshift=-1cm] {End};

    % Arrows
    \draw [arrow] (start) -- (init);
    \draw [arrow] (init) -- (xloop);
    \draw [arrow] (xloop) -- (checkx);
    
    \draw [arrow] (checkx) -- node[anchor=west] {yes} (checkfirstx);
    \draw [arrow] (checkx) -- node[anchor=south] {no} (loopend);
    
    \draw [arrow] (checkfirstx) -- node[anchor=south] {yes} (setfirstx);
    \draw [arrow] (checkfirstx) -- node[anchor=south] {no} (rowsumx);
    \draw [line,dashed,gray] (rowsumx) -- (loopend);
    \draw [arrow] (setfirstx) -- (loopend);
    
    \draw [arrow] (loopend) |- node[anchor=east] {no} (loop);
    \draw [arrow] (loopend) -| node[anchor=east] {yes} (xswap);
    
    \draw [arrow] (xswap) -- node[anchor=south] {yes} (swap);
    \draw [arrow] (xswap) -- node[anchor=east] {no} (repeat);
    \draw [arrow] (swap) |- (repeat);
    
    %\path [line] (start) -- (init);
    %\path [line] (init) -- (x_loop);
    %\path [line] (x_loop) -- (check_first_x);
    %\path [line] (check_first_x) -- node[anchor=south] {yes} (set_first_x);
    %\path [line] (check_first_x) |- node[anchor=east] {no} (x_loop);
    %\path [line] (check_other_x) -- node[anchor=south] {yes} (rowsum_x);
    %\path [line] (check_other_x) -- node[anchor=east] {no} (cond_x_swap);
    %\path [line] (cond_x_swap) -- node[anchor=south] {yes} (rowswap_x);
    %\path [line] (cond_x_swap) -- node[anchor=east] {no} (z_loop);

    %\path [line] (z_loop) -- (check_first_z);
    %\path [line] (check_first_z) -- node[anchor=south] {yes} (set_first_z);
    %\path [line] (check_first_z) -- node[anchor=east] {no} (check_other_z);
    %\path [line] (check_other_z) -- node[anchor=south] {yes} (rowsum_z);
    %\path [line] (check_other_z) -- node[anchor=east] {no} (cond_z_swap);
    %\path [line] (cond_z_swap) -- node[anchor=south] {yes} (rowswap_z);
    %\path [line] (cond_z_swap) -- node[anchor=east] {no} (end);

    %\path [line] (set_first_x) |- (check_other_x);
    %\path [line] (rowsum_x) |- (cond_x_swap);

    %\path [line] (set_first_z) |- (check_other_z);
    %\path [line] (rowsum_z) |- (cond_z_swap);

    %\path [line] (rowswap_x) |- (z_loop);
    %\path [line] (rowswap_z) -- (end);

\end{tikzpicture}

%  \caption{flowchart for rowreduce algorithm}
%  \label{fig:rowreduce-diag}
%\end{figure}

\subsection{\texttt{get\_state\_type}}

\subsection{\texttt{ptrace}}
\begin{figure}[h]
  \centering
  \begin{tikzpicture}[node distance=2cm]

    % Nodes
    \node (start) [startstop] {Start};
    \node (assertions) [process, below of=start] {Check assertions (q, N)};
    
    \node (init_helper) [process, below of=assertions] {Init helper\_tab, last\_stab, last\_anti};
    
    \node (move_qubit_loop) [process, below of=init_helper] {Loop: Move qubit q to the end};
    
    \node (update_mix) [process, below of=move_qubit_loop] {Update mix};
    
    \node (rowreduce) [process, below of=update_mix] {Call rowreduce(N-1)};
    
    \node (decrement_n) [process, below of=rowreduce] {Decrement N};
    
    \node (end) [startstop, below of=decrement_n] {End};

    % Arrows
    \draw [arrow] (start) -- (assertions);
    \draw [arrow] (assertions) -- (init_helper);
    \draw [arrow] (init_helper) -- (move_qubit_loop);
    \draw [arrow] (move_qubit_loop) -- (update_mix);
    \draw [arrow] (update_mix) -- (rowreduce);
    \draw [arrow] (rowreduce) -- (decrement_n);
    \draw [arrow] (decrement_n) -- (end);

\end{tikzpicture}

  \caption{flowchart for ptrace algorithm}
  \label{fig:ptrace-dig}
\end{figure}

\subsection{\texttt{is\_subgroup\_of}}

\subsection{\texttt{project\_or\_mix}}

\subsection{\texttt{force\_projection}}
