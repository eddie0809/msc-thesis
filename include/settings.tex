\setmainfont{EB Garamond}
\setsansfont{Libertinus Sans}
%\setmathfont[range=bfsfit/{greek,Greek,latin,Latin,num}]{EB Garamond}

\title{Cross entropy measures for detecting entanglement transitions}
\thesisname{Master's Thesis}
\author{Etienne Maurice Springer}
\university{Universität Stuttgart}
\city{Stuttgart}
\institute{Institute for Theoretical Physics {III}}
\supervisor{Prof. Dr. Hans-Peter Büchler}
\secondarycorrector{Prof. Dr. Maria Daghofer}

\hypersetup{
  colorlinks=true,
  linkcolor=maincolor5,
  citecolor=maincolor2,
  urlcolor=magma-40,
  linkbordercolor= {white},
  pdfauthor={Etienne Maurice Springer},
  pdfsubject={Master's Thesis},
  pdfpagelabels=true,
  breaklinks=true,
  plainpages=false,
  bookmarks, bookmarksnumbered=true
}

\graphicspath{{./fig/}}

% The following is from NASA: https://nasa.github.io/nasa-latex-docs/html/examples/listing.html

% Define a custom color
\definecolor{codegreen}{rgb}{0,0.6,0}
\definecolor{codegray}{rgb}{0.5,0.5,0.5}
\definecolor{codepurple}{rgb}{0.58,0,0.82}
\definecolor{backcolour}{rgb}{0.95,0.95,0.92}
% Define a custom lst style
\lstdefinestyle{myStyle}{
    backgroundcolor=\color{backcolour},   
    commentstyle=\color{codegreen},
    keywordstyle=\color{magenta},
    numberstyle=\tiny\color{codegray},
    stringstyle=\color{codepurple},
    basicstyle=\ttfamily\footnotesize,
    breakatwhitespace=false,         
    frame=single,
    breaklines=true,                 
    keepspaces=true,                 
    numbers=left,       
    numbersep=5pt,                  
    showspaces=false,                
    showstringspaces=false,
    showtabs=false,                  
    tabsize=2,
}

% The following \crefname declarations are needed only
% if your cross-references contain plural items
\crefname{thm}{Theorem}{Theorems}
\crefname{defn}{Definition}{Definitions}
\crefname{lem}{Lemma}{Lemmata} % or 'lemmata'?
\crefname{cor}{Corollary}{Corollaries} 
\crefname{conj}{Conjecture}{Conjectures}
\crefname{prop}{Proposition}{Propositions}
\crefname{alg}{Algorithm}{Algorithms}

\theoremstyle{definition}
\newtheorem{thm}{Theorem}[section]
\newtheorem{defn}[thm]{Definition}
\newtheorem{lem}[thm]{Lemma}
\newtheorem{cor}[thm]{Corollary}
\newtheorem{conj}[thm]{Conjecture}
\newtheorem{prop}[thm]{Proposition}
\newtheorem{alg}[thm]{Algorithm}
