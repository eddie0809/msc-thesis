%! TEX root = ../msc.tex
\chapter{Introduction}
\label{ch:basics}
\epigraph{I am fascinated by numbers}{
\citeauthor{baron-cohenAutismSpectrumQuotientAQ2001}}
%\cite{baron-cohenAutismSpectrumQuotientAQ2001}}

Was hier useful werden kann:

Nielsen: \cite{nielsenQuantumComputationQuantum2010}

Stabilizer Formalism, Gottesman PhD thesis: \cite{gottesmanStabilizerCodesQuantum1997}

Algorithm for simulating stabilizer circuits:
\cite{aaronsonImprovedSimulationStabilizer2004}

stabilizer lecture notes: \cite{arabLectureNotesQuantum2024}

Entanglement with Stabilizers: \cite{fattalEntanglementStabilizerFormalism2004}

Aaronson's quantum information theory I and II lecture notes:
\cite{aaronsonIntroductionQuantumInformation,aaronsonIntroductionQuantumInformationa}.
This one's a huge find!

another huge find! gottesman lecture notes on quantum error correction:
\cite{gottesmanSurvivingQuantumComputer2024}

Zee Group Theory in a nutshell: \cite{zeeGroupTheoryNutshell2016}

This chapter serves to familiarize the reader with the core concepts relevant
to this thesis. We will first introduce the stabilizer formalism, as it will
later enable us to perform efficient numerical experiments on a classical
computer. We then provide a general introduction to the field of entanglement
transitions and go over some important examples. 
\section{The Stabilizer Formalism}\label{sec:stab-basics}
In this section we review the most important concepts of the stabilizer
formalism. 

Section adapted from \cite{nielsenQuantumComputationQuantum2010} and
\cite{gottesmanStabilizerCodesQuantum1997}
\subsection{Basic notions of group theory}
Group theory is one of the most important algebraic notions in theoretical
phyiscs. From classifying crystalline structures (cite Ashcroft Mermin),
(gauge) symmetries in the standard model (cite some yang mills stuff), to the
classification of (topological) phase transitions (cite nicolai phd?), group
theory perpetually permeates theoretical physiscs.

The stabilizer formalism makes no exception. It is a clever application of
group theory, allowing for a more compact representation of quantum states,
compared to the state vector. We therefore introduce necessary prerequesites of
group theory needed for the stabilizer formalism and its role in this thesis,
starting with the notion of a group.

A group $G$ is a collection of some particulars $g$ that can be composed
according to some convention. To be called a group, the set of entities
$\{g\}$ and the operation\footnote{In the discussion on groups and group
  theory, the words composition, group operation, and multiplication are used
interchangibly} with which we compose them, need to obey a certain
ruleset. This ruleset, called the group axioms, reads as follows.

\begin{defn}[Group]\label{defn:group}
  A group $G$ is a non-empty set equipped with a binary operation (here denoted
  with $\cdot$) that satisfy
  \begin{description}
    \item[Associativity] The group operation is associative, i.e.
      $$\forall a,b,c \in G:\ (a\cdot b) \cdot c = a \cdot
      (b \cdot c)$$
    \item[Identity element] The group contains an identity element, which does
      nothing with respect to composition, i.e.
      $$\exists I\in G \ \forall g \in G : I\cdot g = g \cdot I = g$$
    \item[Inverse element] Each group element has a unique element associated
      to it that when composed with it gives the identity. In other words,
      $$\forall g \in G \ \exists g^{-1} \in G : g g^{-1} = g^{-1} g = I $$
  \end{description}
\end{defn}

Note that we do not require the group elements to commute with respect to
multiplication. Groups that satisfy commutativity for all their elements are
called \emph{abelian} groups.

Numerous different mathematical objects and concepts can fall under the
umbrella of group theory.\footnote{One can even describe the Rubik's Cube
puzzle in the language of group theory} Although we will discuss specific
groups in greater detail later, we do not want to fail to mention some other
important groups appearing all across physics. There are the rotation groups
$SO(n)$ and unitary groups $U(n)$ (both in $n$ dimensions), the permutation
group of $n$ elements $S_n$ and the group of square roots of $1$ under
multiplication, $\mathbb{Z}_2 = \{1, -1\}$.  The first two are examples of
continuous groups, while the others are discrete, meaning that they have a
finite number of elements. The number of elements in a discrete group $G$ is
called the \emph{order} of a group, which we write as $\mathrm{ord}(G)$.

For larger finite groups, it can become cumbersome to keep track of all the
elements and their relation to each other. Luckily, there is a way we can
condense all the information to construct the whole group in its
\emph{presentation}, also known as its \emph{generating set}. The elements of
such a generating set are referred to as \emph{generators}.
\begin{defn}[Generating set and generators]\label{defn:generators}
  Let $G$ be a finite group and $S$ a subset of $G$. $S$ is called a generating
  set of $G$ if all elements in $G$ can be obtained from (possibly repeated)
  multiplication of elements in the generating set. Generating sets are denoted
  by angled braces, such that we write
  \begin{align}
    S = \langle g_1, \ldots, g_n \rangle, \quad g_1,\ldots, g_n \in G
  .\end{align}
  The $g_i$ are called \emph{generators} of $G$. The trivial group $\left(
  \{I\}, \cdot \right)$ is generated by the empty set.
\end{defn}

While there are also generators of continuous groups, they take on a
fundamentally different form compared to the discrete counterpart. As an
example of a discrete generating set, consider the group of fourth roots of $1$
with multiplication, $Z_4 = \left( \{\pm 1,\pm i\}, \cdot \right)$. It is easy
to verify that the subset $S = \{i\}$ of the group uniquely reproduces all of
the elements in $Z_4$, by taking powers of $i$.

In the process of constructing a generating set we are generally faced with two
restrictions. The first is the fact that the entire group needs to emerge from
the multiplications of generators. We cannot simply choose arbitrary group
elements. $S=\{-1\}\subset Z_4$ would be a perfectly fine set of generators for
$\mathbb{Z}_2$, but not $Z_4$. Next, we ideally want to have the least number
of generators possible to build up the rest of the group. This restriction is
one we impose on ourselves rather than one imposed on us by necessity. Choosing
$S=\{-1, i, -i\}$ as generators also recovers $Z_4$, however, we have already
seen that $g=-1$ and $g=-i$ are redundant in this context.
\Cref{thm:maxsize-generators} gives a lower bound on the number of generators
needed to generate finite groups.

\begin{thm}\label{thm:maxsize-generators} 
  The minimum size of a generating set for a finite group $G$ of generators is
  at most $\log_2(\mathrm{ord}(G))$
\end{thm}

%\begin{proof} We prove the theorem by induction over the size of 
%
%  As base case we have Let $G$ be a finite group with minimal generating set
%  $S_n = \langle g_1, \ldots, g_n\rangle.$
%\end{proof}

With \cref{thm:maxsize-generators} we have that for $Z_4$, choosing a
generating set with $3$ elements should be cause for concern, since we would
need $2$ at most. However, we also saw that $Z_4$ is special in that way, since
we only needed one generator. A way of quantifying this quality is the
\emph{rank} of a group.  It is defined as the size of the smallest generating
set of $G$ and is denoted by $\abs{G}$. Thus, for the example of $Z_4$ we have
$\abs{Z_4}=1$. 
\newpage
When discussing subsets of groups, one naturally arising concept is the notion
of subgroups. Suppose we take some set of elements $\{g\}$ forming a group $G$
under multiplication and take a subset $\{h\}$ thereof. If the subset also
forms a group $H$, we call it a subgroup of $G$ and write $H \leq G$.

\begin{defn}[Subgroup]\label{defn:subgroup}
  A subgroup $H$ of $G$, written as $H \leq G$, is a non-empty subset $H$ of $G$, which forms
  a group under the same group operation as $G$. 
\end{defn}

Going back to some of the previous examples, we can consider $SO(2)$, rotations
along the unit circle, as rotations on the equator of a unit sphere. We can
consequently take $SO(2)$ as a subgroup of the rotation group of the unit
sphere $SO(3)$. The group of permutations of $m \leq n$ elements is just the
group of permutations of $n$ elements, where $n-m$ elements are left invariant,
and as such $S_m \leq S_n$. Note that for any group $G$ we have $G \leq G$ and
$(\{I\}, \cdot) \leq G$, where $\left( \{I\}, \cdot \right)$ is the trivial
group containing only the identity. These two special cases are referred to as
\emph{trivial subgroups}. Subgroups that are not trivial are called
\emph{proper subgroups} denoted by $H<G$.

Before introducing another important family of subgroups, we define a special
kind of operation known as \emph{conjugation}. If $h,g\in G$, the conjugate of
$h$ with respect to $g$ is $g^{-1} h g$.\footnote{This operation is
colloquially referred to as \enquote{sandwiching} $h$ with $g$.}
We can not only perform this operation on individual group elements, but also
subgroups of $G$. 
Consider the proper subgroup $H<G$ with elements $\{h_i\}$. 
%Another important family of subgroups are \emph{invariant} subgroups. Let $H$
%be a group of elements $\{h\}$ with $H<G$.
If we take any element $g \in G
\setminus H$, we can arrive at another subgroup of $G$ by conjugating all
elements of the
subgroup $H$ with $g$, written as
\begin{align}
  g^{-1}Hg = \{ g^{-1} h g \ \mid \ h \in H \}
.\end{align}
In general, the two subgroups $H$ and $g^{-1}Hg$ need not be the same. However,
if there are some $g$, which leave $H$ invariant under conjugation, we say that
these $g$ \emph{normalize} $H$. A collection of these normalizing elements can
be compiled together to form yet another subgroup of $G$, called the
normalizer.
\begin{defn}[Normalizer]\label{defn:normalizer}
  Let $G$ be a group and $H < G$ a proper subgroup of $G$. The normalizer of
  $H$ in $G$ is the subgroup of $G$ that leaves $H$ invariant under
  conjugation, i.e.
  \begin{align}
    N_G(H) = \{ g \in G \mid \ g^{-1} H g = H \}
  .\end{align}
\end{defn}
In the special case of every element of $G$ normalizing $H$, that is $N_G(H) =
G$, $H$ is called an \emph{invariant subgroup} of $G$. In the next section we
will introduce an important example of a finite group and its normalizer. 
%However,
%if, for an arbitrary choice of $g$, they are the same, then $H$ is called
%\emph{invariant subgroup} of $G$.
%\begin{defn}[Invariant subgroup]\label{defn:normal-subgroup}
%  Let $G, H$ be groups with $H\lt G$. We call $H$ an invariant (or normal)
%  subgroup of $G$ if
%  \begin{align}
%    \forall g \in G, \forall h \in H: \ g^{-1} h g \in H
%  .\end{align}
%\end{defn}
\subsection{The Pauli group and Clifford gates}

Consider the Pauli matrices with the identity,
\begin{align}
  \sigma_0 = I = \mqty(\pmat{0}),\quad \sigma_x = \mqty(\pmat{1}), \quad \sigma_y =
  \mqty(\pmat{2}), \quad \text{and}\quad \sigma_z = \mqty(\pmat{3})
.\end{align}
These well-known matrices are both hermitian and unitary, and consequently
square to the identity. For the latter three of them one can show that they
satisfy the following commutation and anticommutation relations,
\begin{equation}\label{eq:pauli-comm}
 \begin{split}
  [\sigma_i, \sigma_j] &= 2\mathrm{i}\epsilon_{ijk}\sigma_k \quad \text{and} \\
  \{\sigma_i, \sigma_j\} &= 2I \delta_{ij}
,\end{split} 
\end{equation}
with the Levi-Civita tensor $\epsilon_{ijk}$ (where Einstein summation
convention is implied) and the Kronecker delta $\delta_{ij}$. To ease up on the
indeces, one also writes the Pauli matrix with the corresponding capital
letter, $\sigma_x = X, \ldots$. These matrices also form a basis for hermitian
$2\times 2$ matrices. Recall that physical observables are represented by
hermitian matrices. We can therefore consider the Pauli matrices as a basis for
physical observables on qubits. 

While they also play
an important role in representation theory, especially of Lie and Clifford
algebras, they themselves also form a group known as the Pauli group
$\mathcal{P}$. The single-qubit Pauli group is defined as the Pauli matrices
with phases $\pm 1$ and $\pm i$,
\begin{align}
  \mathcal{P} = \{\pm I, \pm i I, \pm X, \pm i X, \pm Y, \pm i Y, \pm Z, \pm i
  Z \}
.\end{align}
This definition can also be generalized to $n$ qubits.
\begin{defn}[Pauli group]\label{defn:pauligroup}
  The Pauli group $\mathcal{P}_n$ is composed of tensor products of $I,\ X,\
  Y,$ and $Z$ on $n$ qubits with an overall phase of $\pm 1$ and $\pm i$.
\end{defn}
From \cref{eq:pauli-comm} it follows that $\mathcal{P}_n$ is not Abelian. 

\begin{itemize}
  \item (anti-)commutation relations
  \item definition pauli group
  \item non-abelian!
  \item commutation relations for larger pauli
  \item Clifford gates and Clifford group
\end{itemize}
It is well-known and easy to verify that the latter three of them do not
commute with each other and all 
The Pauli group $\mathcal{P}$ is the group consisting of 

\begin{itemize}
  \item Stabilizers are abelian subgroups of Pauli group
  \item Stabilizer group is the symmetry group of the code space $V_S$
    (\cref{defn:fixpointgroup})
  \item Clifford group is normalizer (\cref{defn:normalizer}) of Pauli group, Clifford gates can be
    simulated efficiently, cf. \cref{thm:gottesman-knill}
\end{itemize}

\subsection{The stabilizer group}

So far we have only examined groups in isolation. However, the major role group
theory plays in physics can best be demonstrated if one considers the action of
group elements on other mathematical objects outside of the group. These could
be, for example, Lagrangians, or more relevant for us, state vectors. If we
consider the two-qubit state vector $\ket{+} = \left( \ket{0} + \ket{1} \right)
/ \sqrt{2}$, we can clearly tell that this state is resistant to bitflips. The
more formal way to put this is to say that $\ket{+}$ is
$\mathbb{Z}_2$-symmetric. This notion of symmetry is where group theory finds
most of its utility in physics. The definition of a symmetry group is stated in
the following.

\begin{defn}[Symmetry group]\label{defn:fixpointgroup}
  Let $G$ be a group acting on a set $M$. Let $a\in M$. We then call the
  subgroup
  \[ H = \left\{ h \in G \mid ha = a \right\} \leq G \]
  \emph{symmetry group} or \emph{fixpoint group} of $a$.
\end{defn}
\begin{defn}[Stabilizer]\label{defn:stabilizergroup}
  Let $H^{\otimes N}$ denote the $N$-qubit Hilbert space. 
\end{defn}
Consider the 2-qubit Bell state
\begin{align}
  \ket{\psi} = \frac{\ket{00} + \ket{11}}{\sqrt{2}} 
.\end{align}

Note that the unitary operations $X_1 X_2$ and $Z_1 Z_2$ both have $\ket{\psi}$
as eigenstate with eigenvalue $+1$.

\begin{itemize}
  \item Stabilizer group
  \item code space
  \item generating set?
  \item unitaries by conjugation of stabilizer with cliffords
  \item measurements
  \item gottesman knill
  \item entanglement in the stabilizer formalism
\end{itemize}

\begin{defn}
  Let $S< \mathcal{P}_n$. We define $V_S$ as the set of $n$ qubit states stabilized by
  $S$.
\end{defn}

Note that while the generating set is explicitly not the entire group, the
distinction between the two is kept rather loosely. Oftentimes we will write
that some group $G$ is equal to its generating set. This is mostly a matter of
convenience and reabability. If context does not explicitly demand it, we write
the generating set and the group interchangibly.

Currently verbatim from \cite{nielsenQuantumComputationQuantum2010}! Beware!
Actual source is \emph{inside}
\cite{gottesmanHeisenbergRepresentationQuantum1998}, basically stating "trust
me bro" (one of the most important theorems of quantum computation is cited as
private communication in its original source\ldots)
\begin{thm}[Gottesman-Knill theorem]\label{thm:gottesman-knill}
  Suppose a quantum computation is performed which involves only the following
  elements: state preparations in the computational basis, Hadamard gates,
  phase gates, controlled-\verb|NOT| gates, Pauli gates, and measurements of
  observables in the Pauli group (which includes measurement in the
  computational basis as a special case), together with the possibility of
  classical control conditioned on the outcome of such measurements. Such
  computation may be efficiently simulated on a classical computer.
\end{thm}
We will forego a detailed discussion of \cref{thm:gottesman-knill} until
\cref{sec:tableau}. 

\section{Entanglement Transitions}\label{sec:ent-trans}

Random assortment of MIPT Paper (incomplete):

\begin{itemize}
  \item Fisher paper, das ich mir als allererstes mal durchgelesen hab
    (\citetitle{liMeasurementdrivenEntanglementTransition2019}):
    \cite{liMeasurementdrivenEntanglementTransition2019}
  \item MIPT general, war in \cref{ch:lxe}, aber leider keine ahnung mehr
    warum\ldots (\citetitle{baoTheoryPhaseTransition2020}) \cite{baoTheoryPhaseTransition2020}
  \item \citetitle{baoSymmetryEnrichedPhases2021}
    \cite{baoSymmetryEnrichedPhases2021}
  \item Why not: Measurement induced synchronization von Finn
    (\citetitle{schmolkeMeasurementinducedQuantumSynchronization2023}):
    \cite{schmolkeMeasurementinducedQuantumSynchronization2023}
  \item 2017 WR for quantum state tomography (10 qubits):
    (\citetitle{song10QubitEntanglementParallel2017})
    \cite{song10QubitEntanglementParallel2017}
  \item LXE: Definition: \cite{liCrossEntropyBenchmark2023}; PTIM:
    \cite{tikhanovskayaUniversalityCrossEntropy2023}
  \item Garratt/Altman Paper:
    (\citetitle{garrattProbingPostmeasurementEntanglement2023}) \cite{garrattProbingPostmeasurementEntanglement2023}
  \item self-cite for clout:
    (\citetitle{schmolkeBoostingInformationTransfer2024})
    \cite{schmolkeBoostingInformationTransfer2024}
\end{itemize}
\begin{figure}[H]
  \centering
  \includegraphics{Untitled.png}
  \caption{Hybrid Circuit Tikz picture??}
  \label{fig:hybrid-circuit}
\end{figure}

\subsection{Phenomenology}

\section{The Projective Transverse-Field Ising Model}
\begin{itemize}
  \item PTIM paper: \citetitle{langEntanglementTransitionProjective2020}
    \cite{langEntanglementTransitionProjective2020}
  \item Felix decoder paper: \citetitle{roserDecodingProjectiveTransverse2023}
    \cite{roserDecodingProjectiveTransverse2023}
  \item Colored Cluster Model!
  \item There should be a Skinner paper, but I don't have it in my zotero
    library (yet)
\end{itemize}

\begin{figure}[H]
  \centering
  \includegraphics{Untitled.png}
  \caption{Tikz sketch of PTIM setup with $\approx 10$ qubits}
  \label{fig:ptim-circuit}
\end{figure}

\begin{figure}[H]
  \centering
  \includegraphics{Untitled.png}
  \caption{plot of entanglement entropy (and mutual information??) as a
  function of $p$. multiple system sizes, possibly rather large ($\approx 512$
qubits?, maybe $N=\{128,256,512\}$ just for the fun of it). Would need to
include mutual information in the source code.}
  \label{fig:phase-transition}
\end{figure}

\section{Sampling Problem}\label{sec:sampling}
The metaphysics of this endeavour can
be condensed in the following way; we (a) know from experiments how quantum
systems behave under certain conditions, and (b) predict through theoretical
calculations what these systems might do in another experimental setting. In
the latter case however, there is an uncanny regime of utility, where we either
(a) cannot precisely pass predictions or (b) cannot perform the experiment on
the grounds of hardware limitations\footnote{The Higgs particle was predicted
40 years before it was discovered \textcolor{red}{citation}} or (c) try to
predict the behavior of quantities not directly measurable. In the case of
quantum computation, and especially in the field of entanglement transitions,
we face these bottlenecks in increasing severity. To make do with them, we
employ classical computer simulations. That is, we perform numerical
experiments. While traditional experiments still serve as the sole proprietor of claim to
ontology, numerical experiments can play a supporting role, 

In the year of our
lord 2024, we phyisicists are thankfully able to perform experiments at home
with cleverly assembled silicon.  That is, nowadays we make do with these
bottlenecks by performing numerical experiments. This is not to discredit the
utility of experiments as such, on the contrary! 

\subsection{Fisher: Linear Cross Entropy}
\cite{liCrossEntropyBenchmark2023}

\subsection{Altman: Upper Bound}
\cite{garrattProbingPostmeasurementEntanglement2024}
