\chapter{Relative Entropy}
\label{ch:rel-ent}

In this chapter we investigate if Klein's Inequality can assist us in obtaining
a sensible upper bound for the entanglement entropy and thus make the
entanglement transition visible.

Idea from Altman Paper:
\citetitle{garrattProbingPostmeasurementEntanglement2023}
\cite{garrattProbingPostmeasurementEntanglement2023}.
\section{The Idea}
In \cite{garrattProbingPostmeasurementEntanglement2023} they introduce a method to obtain an upper bound on the
entanglement entropy through classical means. In particular, they apply Klein's
inequality \cite{nielsenQuantumComputationQuantum2010}
\begin{align}
  \label{eq:kleins-ineq}
  S(\rho \mid\mid \sigma) \geq 0 \Leftrightarrow -\Tr[\rho\log\sigma]\geq
  S_\mathrm{E}
\end{align}
with the relative and entanglement entropy $S(\rho\mid\mid\sigma)$ and
$S_\mathrm{E}$ respectively.
Among other things, they make use of shadow tomography to circumvent the
post-selection problem. 
%However, they do MPS simulations, and not stabilizer. This, of course, calls
%for an adaptation (oh well\ldots).

%\subsubsection{Aside: On References}
%I like how references are provided in physics: all those little numbers in
%little brackets or as superscript above a claim to support it. It allows me to
%follow the bibliography in parallel to the paper and check some pertinent
%publications easily. This is, of course, very different in many other
%disciplines, i.e. psychology or philosophy, where it is common-practice to give
%your sources in alphabetical order sorted by the author's surname.
%
%I claim that physicists are not appreciative enough of this. It will never fail
%to frustrate me if people cite some shit that has nothing at all to do with the
%claim in question. This is especially so if the claim is better supported by
%the results of other papers; papers that a lot of times make it into the
%\texttt{.bib} file of at least one of the authors, or why else would they be
%prominently cited in a section, where it doesnt make sense to do so.
%Ultimately, the highest degree of frustration for me is if the references are
%off by one, or cite a previous project by the same authors. This kind of
%situation lies in the uncanny valley of "is it them or me?". And more often
%than not, it should be \emph{me}, right? But then you go on and on, read the
%cited papers thoroughly, try to understand the context in which it is
%referenced in the work you originally went through, and now, multiple layers
%deep in the reference rabbithole are you ultimately forced to realize that it's
%not you, it's them. 

\subsection{Shadow Tomography}
these papers are cited in
\cite{garrattProbingPostmeasurementEntanglement2023} for shadow tomography:
\begin{itemize}
  \item \citetitle{elbenRandomizedMeasurementToolbox2022}
    \cite{elbenRandomizedMeasurementToolbox2022}

    Review paper. Context: "It will be useful to first outline shadow
    tomography \cite{elbenRandomizedMeasurementToolbox2022} in its simplest
    incarnation for a single qubit"

    It has indeed been useful, but I gotta admit, i didn't get my outline from
    this one, but from:
  \item \citetitle{huangPredictingManyProperties2020}
    \cite{huangPredictingManyProperties2020} 

    In this paper, shadow tomography is the main thing investigated. Context:
    "Generalizing \ldots to multiple qubits is straightforward
    \cite{huangPredictingManyProperties2020}: one possibility is to construct
    $N$-qubit shadows as the tensor products of $N$ objects with the structure
    \ldots"
\end{itemize}

Some more shadow tomography papers:
\begin{itemize}
  \item \cite{aaronsonShadowTomographyQuantum2018} shadow tomography defined
    for the first time
  \item IDK where i want to include this one, but i think it will be
    interesting to study a bit more:
    \cite{tothEntanglementDetectionStabilizer2005} Entanglement witness
\end{itemize}

This one is not on shadow tomography, but is cited in
\cite{garrattProbingPostmeasurementEntanglement2023}, which got me hella
frustrated, since I did not know what the hell they meant with "Our discussion
in this section largely follows Ref.
\cite{garrattMeasurementsConspireNonlocally2023}" if it does not really match
the vibe of \emph{this section}.

\section{Na\"ive approach}
spoiler: it (sort of) doesn't work out (as expected).

One property of $S(\rho\mid\mid\sigma)$ is
\cite{leditzkyRelativeEntropiesTheir2016}
\begin{align}\label{eq:inf-cond}
  S(\rho\mid\mid\sigma) < \infty \Longleftrightarrow \mathrm{supp}(\rho) \subseteq
  \mathrm{supp}(\sigma)
,\end{align}
where supp$(\bullet)\equiv\ $ker$(\bullet)^\perp$ is the support of a linear
operator, which is defined as the orthogonal complement to the kernel.
Alternatively, for diagonalizable matrices, it is the subspace spanned by
eigenvectors with non-zero eigenvalues
\cite{schumacherRelativeEntropyQuantum2000}.

In this section we will provide and prove a necessary and sufficient condition
for $S(\rho\mid\mid\sigma)<\infty$ when $\rho$ and $\sigma$ are stabilizer
density matrices (\cref{prop:subgroup}). 
To prove \cref{prop:subgroup} we will need an auxilliary lemma, which will
be stated and proven in the following.
\begin{lem}\label{lem:supp-is-vs}
  Let $\rho$ be an $N$-qubit stabilizer density matrix with stabilizer group $S = \langle
  g_1, \ldots, g_n \rangle$, and $0\leq n \leq N$.
  Then \[ \mathrm{supp}(\rho) = V_{S}.\]
\end{lem}
\begin{proof}[Proof of \cref{lem:supp-is-vs}]
  By definition (cf. \cref{ch:basics}) we take $V_{S}$ to be the vector space
  stabilized by $S$.  Further, it is the intersection of subspaces fixed by
  each operator in $S$, i.e. the eigenvalue one eigenspaces of elements of $S$
  (cf.  \cref{ch:basics} and \cite{nielsenQuantumComputationQuantum2010}). More
  formally we can write
  \[ 
    V_S = \bigcap_{g \in S}  \mathrm{span}\left\{\ket{\psi} \mid g\ket{\psi} =
    \ket{\psi}\right\} = \mathrm{span}\left\{\ket{\psi} \mid g\ket{\psi} =
    \ket{\psi} \forall g \in S\right\}.
  \]
  This subspace is projected onto by
  \[ P_S \equiv \frac{1}{2^n} \prod_{g\in S} \left(\mathds{1} + g\right).\]
  Note that $\rho$ can be constructed from a product of projectors
  \cite{gottesmanStabilizerCodesQuantum1997}
  \[ \rho = \frac{1}{2^N} \prod_{g \in S} \left(\mathds{1} + g\right) = 2^{N-n} P_S \]
  We thus have
  \[ \mathrm{supp}(\rho) = \mathrm{supp}(P_S) = V_S. \]
\end{proof}

\begin{prop}\label{prop:subgroup}
  Let $\rho$ and $\sigma$ be $N$-qubit stabilizer density matrices with
  respective stabilizer groups $S_\rho$ and $S_\sigma$. Then
  \[ \mathrm{supp}(\rho)\subseteq \mathrm{supp}(\sigma) \Longleftrightarrow
  S_\sigma \leq S_\rho. \]
  That is, with \cref{eq:inf-cond}, $S(\rho\mid\mid\sigma)$ takes on finite values iff. $S_\sigma$ is a
  subgroup of $S_\rho$.
\end{prop}

\begin{proof}[Proof of \cref{prop:subgroup}]
  We will prove implication from both directions to prove equivalence.

  \enquote{$\Leftarrow$} Let $S_\sigma \leq S_\rho$. Then
  \begin{align*}
    \mathrm{supp}(\rho) = V_{S,\rho} &= \bigcap_{g\in S_\rho} \mathrm{span}\left\{ \ket{\psi} \mid
    g\ket{\psi} = \ket{\psi} \right\} \\
        &\overset{S_\sigma \leq S_\rho}{=} \underbrace{\bigcap_{g\in S_\sigma}
        \mathrm{span}\left\{\ket{\psi} \mid g \ket{\psi} =
        \ket{\psi}\right\}}_{V_{S,\sigma}}\ \ \ \cap \bigcap_{g \in S_\rho \setminus
        S_\sigma} \mathrm{span}\left\{\ket{\psi} \mid g \ket{\psi} =
        \ket{\psi}\right\}\\
        &= V_{S,\sigma} \cap \bigcap_{g \in S_\rho \setminus
        S_\sigma} \mathrm{span}\left\{\ket{\psi} \mid g \ket{\psi} =
        \ket{\psi}\right\}\\
        &\subseteq V_{S,\sigma} = \mathrm{supp}(\sigma)
  .\end{align*}
  which finishes the proof of this direction.

  \enquote{$\Rightarrow$} Let $V_{S,\rho} = \mathrm{supp}(\rho) \subseteq
  \mathrm{supp}(\sigma) = V_{S,\sigma}$ and let
  $\ket{\psi} \in \mathcal{B}\left(V_{S,\rho}\right)$ and $\ket{\phi} \in
  \mathcal{B}\left(V_{S,\sigma}\right) \setminus
  \mathcal{B}\left(V_{S,\rho}\right)$. Since the stabilizer group is the
  symmetry group of $V_S$, we can rewrite $S_\sigma$ as
  \[ 
    S_\sigma = \left\{ g \in G_N \mid g \ket{\psi} = \ket{\psi} \wedge g\ket{\phi} =
  \ket{\phi} \right\},
  \]
  (where $G_N$ is the $N$-qubit Pauli group) 
  as we want $S_\sigma$ to stabilize both $V_{S,\sigma}$ and its subset
  $V_{S,\rho}$. We can then split up the condition into the intersection of two
  sets, i.e.
  \begin{align*}
    S_\sigma &= \left\{ g \in G_N \mid g \ket{\psi} = \ket{\psi} \wedge g\ket{\phi} =
  \ket{\phi} \right\}\\
             &= \underbrace{\left\{ g \in G_N \mid g\ket{\psi}= \ket{\psi}
             \right\}}_{S_\rho} \cap \left\{ g \in G_N \mid g\ket{\phi} = \ket{\phi} \right\} \\
             &= S_\rho \cap \left\{ g \in G_N \mid g\ket{\phi} = \ket{\phi} \right\} \\
               &\leq S_\rho 
  .\end{align*}
  This concludes the proof.
\end{proof}

\subsection{\enquote{Failure} of the na\"ive approach}
\begin{itemize}
  \item subgroup condition requires errors to be detected (and dealt with in
    some way)
  \item errors which are not succeeded by a measurement will lead to a change
    in the group structure of $S_\rho$ such that $S_\sigma$ \enquote{never has
    the chance to adapt}
  \item maybe we should also implement some ideas from the shadow tomography
    dings, such that errors are circumvented
  \item Undetectable change in group structure happens if an error happens on a
    qubit in a state stabilized by an orthogonal operator, e.g. an $X$-error if
    the qubit is stabilized by $Z$.
  \item The probability of any error happening in the last layer (i.e. after
    all measurements in the circuit happenend) is $1-$ the probability of no
    error, which is given by the binomial distribution
    \[ P(\#\mathrm{err}>0) = 1-\binom{N}{0} (1-q)^N=1-(1-q)^N. \]
  \item For $q=\num{0.01}$ we have $P\simeq .275$
  \item Of course, this probability says nothing about the actual simulation
    (except for $p=0$) as it could be an error which was preceeded by an $X$
    measurement.
  \item That being said, you could technically compute this probability
    analytically, by going through each layer and multiplying the probabilites
    for $X$ and not $ZZ$ happening to get the actual probability of failure
   \item This is way too complicated, just to get an estimate on how many
     simulations will end in $\infty$ for \cref{eq:kleins-ineq}. 
\end{itemize}
\clearpage
\section{some relevant references}
\begin{enumerate}
  \item \citetitle{veitchResourceTheoryStabilizer2014}, entropic quantities of
    magic states; quantifying non-stabilizerness. This paper is with gottesman
    himself, so i thought it might be interesting to study. 
  \item \citetitle{niekampEntropicUncertaintyRelations2012}, more or less what
    it says. No regularization scheme or whatever. I think stabilizers are
    mentioned tangentially, thats why i found it.
  \item \citetitle{lashkariRelativeEntropiesConformal2014}, $\leftarrow$ me
    grasping for straws.
  \item \citetitle{leonePhaseTransitionStabilizer2024}, very recent, not too
    particularily relevant to our stuff as far as i can tell.
  \item \citetitle{wuEntanglementUpperBound2011} idfk\ldots
  \item \citetitle{vedralEntanglementMeasuresPurification1998} one of the more
    promising papers i've come across. probably helpful in proving the failure
    of the na\"ive approach. However, it was published in 1998. It most likely
    has nothing to do with stabilizers.
  \item \citetitle{lindenQuantumEntropyCone2013}, probably some interesting
    stuff in there, but not the thing i want.
  \item \citetitle{buExtremalityStabilizerStates2024}, this paper is where most
    of my confusions come from. 
\end{enumerate}
