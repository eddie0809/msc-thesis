\chapter{Theoretical Basics}
\label{ch:basics}
\epigraph{I am fascinated by numbers}{
\citeauthor{baron-cohenAutismSpectrumQuotientAQ2001}}
%\cite{baron-cohenAutismSpectrumQuotientAQ2001}}

Was hier useful werden kann:

Nielsen: \cite{nielsenQuantumComputationQuantum2010}

Stabilizer Formalism, Gottesman PhD thesis: \cite{gottesmanStabilizerCodesQuantum1997}

Entanglement with Stabilizers: \cite{fattalEntanglementStabilizerFormalism2004}

Algorithm for simulating stabilizer circuits:
\cite{aaronsonImprovedSimulationStabilizer2004}

\section{Basic notions of group theory}
blabla

\begin{defn}\label{defn:fixpointgroup}
  Let $G$ be a group acting on a set $M$. Let $a\in M$. We then call the
  subgroup
  \[ H = \left\{ h \in G \mid ha = a \right\} \leq G \]
  \emph{symmetry group} or \emph{fixpoint group} of $a$.
\end{defn}

\section{The Stabilizer Formalism}\label{sec:basics-stab}
Section adapted from \cite{nielsenQuantumComputationQuantum2010} and
\cite{gottesmanStabilizerCodesQuantum1997}

In this section we review the most important concepts of the stabilizer
formalism. 

Consider the 2-qubit Bell state
\begin{align}
  \ket{\psi} = \frac{\ket{00} + \ket{11}}{\sqrt{2}} 
.\end{align}

Note that the unitary operations $X_1 X_2$ and $Z_1 Z_2$ both have $\ket{\psi}$
as eigenstate with eigenvalue $+1$.



\begin{defn}
  Let $S\leq G_n$. We define $V_S$ as the set of $n$ qubit states stabilized by
  $S$.
\end{defn}
