\chapter{Relative Entropy}
\label{ch:rel-ent}

In this chapter we investigate if Klein's Inequality can assist us in obtaining
a sensible upper bound for the entanglement entropy and thus make the
entanglement transition visible.

Idea from Altman Paper:
\citetitle{garrattProbingPostmeasurementEntanglement2023}
\cite{garrattProbingPostmeasurementEntanglement2023}.

\section{The Idea \& Stabilizers}
Hier Klein's Inequality (mit Herleitung), Interpretation von KL-Divergence und
Cross Entropy.

In this section we will introduce the upper bound on entanglement entropy,
provide an information theoretic interpretation and derive an expression for
the quantity of interest in case of stabilizer states. First however, we will
freshen up on definitions of some important quantities.
\subsubsection{Entropy of entanglement}
The main quantity of interest in the whole field of entanglement transitions is
the entropy of entanglement. It is a measure how entangled one subsystem of a
bipartite state $\ket{\psi}_{AB}$ is with the other. We recall from
\cref{sec:ent-trans} that its definition can be stated as follows
\cite{fattalEntanglementStabilizerFormalism2004}.
\begin{defn}\label{defn:entanglement-entropy}
  Given a bipartite pure state $\ket{\psi}$ with subsystems $A$ and $B$,
  the entropy of entanglement is defined as
  \begin{align}
    S_\mathrm{E}\left(\ket{\psi}\right) \equiv - \Tr[\rho_B \log \rho_B],
  \end{align}
  where $\rho_B = \Tr_A[\dyad{\psi}]$ is the reduced density matrix of subsystem
  $B$. Conventionally, one uses the logarithm of base 2.
\end{defn}
This quantity can be efficiently computed in clifford circuits via the
stabilizer formalism. It is important to note that the density matrix of the
whole system $\rho = \dyad{\psi}$ describes a pure state. For a more universal
measure of entropy, where the entropy of entanglement is a special case, we
need to consider the more general \emph{Von Neumann entropy}.
\subsubsection{Von Neumann entropy}
Similar in appearance, but nonetheless quite different is the Von Neumann
entropy. It was originally introduced as an extension of the classical Shannon
entropy to quantum systems, as density matrices serve as extension of the
classical notion of (discrete) probability distributions
\cite{vonneumannMathematischeGrundlagenQuantenmechanik1968}.\footnote{Shannon
  entropy is a measure of the average information of probabilistic events.
  Since Information is defined as $I=-\log p_x$ for some event $x$ with
  probability $p_x$, we can write down an average, $S = \expval{-\log p_x}_x =
  -\sum_x p_x \log p_x$, which defines the Shannon entropy
\cite{shannonMathematicalTheoryCommunication1948}.}
Consequently, we can write down a definition of the Von Neumann entropy.
\begin{defn}\label{defn:vonneumann}
Given a general (i.e. possibly mixed) quantum state described by the density matrix
$\rho$, the Von Neumann entropy can be defined as
\begin{align}\label{eq:entropy-vn}
  S\left(\rho\right) \equiv \expval{-\log \rho} = -\Tr[\rho\log\rho]
.\end{align}
\end{defn}
For diagonalizable matrices (as we expect density matrices to be) we can also
express \cref{eq:entropy-vn} as a sum over eigenvalues $\lambda_k$, i.e.
\begin{align}
  S\left(\rho\right) = -\sum_k \lambda_k \log \lambda_k
.\end{align}
One property we want to emphasize in particular is that $S(\rho)$ is bounded
from above and below; it is $0$ for a pure state and $\log d$ for a maximally
mixed state in a $d$-dimensional Hilbert space.
\subsubsection{Quantum relative entropy}
Lastly, we want to introduce the quantum relative entropy. The same way we
motivated the Von Neumann entropy, we want to define a quantum mechanical
analogue to the classical relative entropy. 

\begin{defn}\label{defn:rel-ent}
  Given two density matrices $\rho$ and $\sigma$, 

  $S: H^{\otimes N} \times H^{\otimes N} \to \mathbb{R}^+_0 \cup \{\infty\}$

  \begin{align}
    S\left(\rho\mid\mid\sigma\right) \equiv \Tr[\rho\log\rho] -
    \Tr[\rho\log\sigma] = -S(\rho) - \Tr[\rho\log\sigma]
  .\end{align}
\end{defn}


\subsection{Klein's Inequality}
In \cite{garrattProbingPostmeasurementEntanglement2023} they introduce a method
to obtain an upper bound on the entanglement entropy through classical means.
In particular, they apply Klein's inequality
\cite{nielsenQuantumComputationQuantum2010}
\begin{align}
  \label{eq:kleins-ineq}
  S(\rho \mid\mid \sigma) \geq 0 \Leftrightarrow -\Tr[\rho\log\sigma]\geq
  S_\mathrm{E}
\end{align}
with the relative and entanglement entropy $S(\rho\mid\mid\sigma)$ and
$S_\mathrm{E}$ respectively.
Among other things, they make use of shadow tomography to circumvent the
post-selection problem. 
However, they do MPS simulations, and not stabilizer. This, of course, calls
for an adaptation (oh well\ldots).
\subsection{Stabilizers}

\subsubsection{Aside: On References}
I like how references are provided in physics: all those little numbers in
little brackets or as superscript above a claim to support it. It allows me to
follow the bibliography in parallel to the paper and check some pertinent
publications easily. This is, of course, very different in many other
disciplines, i.e. psychology or philosophy, where it is common-practice to give
your sources in alphabetical order sorted by the author's surname.

I claim that physicists are not appreciative enough of this. It will never fail
to frustrate me if people cite some shit that has nothing at all to do with the
claim in question. This is especially so if the claim is better supported by
the results of other papers; papers that a lot of times make it into the
\texttt{.bib} file of at least one of the authors, or why else would they be
prominently cited in a section, where it doesnt make sense to do so.
Ultimately, the highest degree of frustration for me is if the references are
off by one, or cite a previous project by the same authors. This kind of
situation lies in the uncanny valley of "is it them or me?". And more often
than not, it should be \emph{me}, right? But then you go on and on, read the
cited papers thoroughly, try to understand the context in which it is
referenced in the work you originally went through, and now, multiple layers
deep in the reference rabbithole are you ultimately forced to realize that it's
not you, it's them. 

%\subsection{Shadow Tomography}
%these papers are cited in
%\cite{garrattProbingPostmeasurementEntanglement2023} for shadow tomography:
%\begin{itemize}
%  \item \citetitle{elbenRandomizedMeasurementToolbox2022}
%    \cite{elbenRandomizedMeasurementToolbox2022}
%
%    Review paper. Context: "It will be useful to first outline shadow
%    tomography \cite{elbenRandomizedMeasurementToolbox2022} in its simplest
%    incarnation for a single qubit"
%
%    It has indeed been useful, but I gotta admit, i didn't get my outline from
%    this one, but from:
%  \item \citetitle{huangPredictingManyProperties2020}
%    \cite{huangPredictingManyProperties2020} 
%
%    In this paper, shadow tomography is the main thing investigated. Context:
%    "Generalizing \ldots to multiple qubits is straightforward
%    \cite{huangPredictingManyProperties2020}: one possibility is to construct
%    $N$-qubit shadows as the tensor products of $N$ objects with the structure
%    \ldots"
%\end{itemize}
%
%Some more shadow tomography papers:
%\begin{itemize}
%  \item \cite{aaronsonShadowTomographyQuantum2018} shadow tomography defined
%    for the first time
%  \item IDK where i want to include this one, but i think it will be
%    interesting to study a bit more:
%    \cite{tothEntanglementDetectionStabilizer2005} Entanglement witness
%\end{itemize}
%
%This one is not on shadow tomography, but is cited in
%\cite{garrattProbingPostmeasurementEntanglement2023}, which got me hella
%frustrated, since I did not know what the hell they meant with "Our discussion
%in this section largely follows Ref.
%\cite{garrattMeasurementsConspireNonlocally2023}" if it does not really match
%the vibe of \emph{this section}.
%
\section{Na\"ive approach}
spoiler: it (sort of) doesn't work out (as expected).

One property of $S(\rho\mid\mid\sigma)$ is
\cite{leditzkyRelativeEntropiesTheir2016}
\begin{align}\label{eq:inf-cond}
  S(\rho\mid\mid\sigma) < \infty \Longleftrightarrow \mathrm{supp}(\rho) \subseteq
  \mathrm{supp}(\sigma)
,\end{align}
where supp$(\bullet)\equiv\ $ker$(\bullet)^\perp$ is the support of a linear
operator, which is defined as the orthogonal complement to the kernel.
Alternatively, for diagonalizable matrices, it is the subspace spanned by
eigenvectors with non-zero eigenvalues
\cite{schumacherRelativeEntropyQuantum2000}.

In this section we will provide and prove a necessary and sufficient condition
for $S(\rho\mid\mid\sigma)<\infty$ when $\rho$ and $\sigma$ are stabilizer
density matrices (\cref{prop:subgroup}). 
To prove \cref{prop:subgroup} we will need an auxilliary lemma, which will
be stated and proven in the following.
\begin{lem}\label{lem:supp-is-vs}
  Let $\rho$ be an $N$-qubit stabilizer density matrix with stabilizer group $S = \langle
  g_1, \ldots, g_n \rangle$, and $0\leq n \leq N$.
  Then \[ \mathrm{supp}(\rho) = V_{S}.\]
\end{lem}
\begin{proof}[Proof of \cref{lem:supp-is-vs}]
  By definition (cf. \cref{ch:basics}) we take $V_{S}$ to be the vector space
  stabilized by $S$.  Further, it is the intersection of subspaces fixed by
  each operator in $S$, i.e. the eigenvalue one eigenspaces of elements of $S$
  (cf.  \cref{ch:basics} and \cite{nielsenQuantumComputationQuantum2010}). More
  formally we can write
  \[ 
    V_S = \bigcap_{g \in S}  \left\{\ket{\psi} \mid g\ket{\psi} =
    \ket{\psi}\right\} = \left\{\ket{\psi} \mid g\ket{\psi} =
    \ket{\psi} \forall g \in S\right\}.
  \]
  This subspace is projected onto by
  \[ P_S \equiv \frac{1}{2^n} \prod_{g\in S} \left(\mathds{1} + g\right).\]
  Note that $\rho$ can be constructed from a product of projectors
  \cite{gottesmanStabilizerCodesQuantum1997}
  \[ \rho = \frac{1}{2^N} \prod_{g \in S} \left(\mathds{1} + g\right) = 2^{n-N} P_S \]
  We thus have
  \[ \mathrm{supp}(\rho) = \mathrm{supp}(P_S) = V_S. \]
\end{proof}

\begin{prop}\label{prop:subgroup}
  Let $\rho$ and $\sigma$ be $N$-qubit stabilizer density matrices with
  respective stabilizer groups $S_\rho$ and $S_\sigma$. Then
  \[ \mathrm{supp}(\rho)\subseteq \mathrm{supp}(\sigma) \Longleftrightarrow
  S_\sigma \leq S_\rho. \]
  That is, with \cref{eq:inf-cond}, $S(\rho\mid\mid\sigma)$ takes on finite values iff. $S_\sigma$ is a
  subgroup of $S_\rho$.
\end{prop}

\begin{proof}[Proof of \cref{prop:subgroup}]
  We will prove implication from both directions to prove equivalence.

  \enquote{$\Leftarrow$} Let $S_\sigma \leq S_\rho$. Then
  \begin{align*}
    \mathrm{supp}(\rho) = V_{S,\rho} &= \bigcap_{g\in S_\rho} \left\{ \ket{\psi} \mid
    g\ket{\psi} = \ket{\psi} \right\} \\
        &\overset{S_\sigma \leq S_\rho}{=} \underbrace{\bigcap_{g\in S_\sigma}
        \left\{\ket{\psi} \mid g \ket{\psi} =
        \ket{\psi}\right\}}_{V_{S,\sigma}}\ \ \ \cap \bigcap_{g \in S_\rho \setminus
        S_\sigma} \left\{\ket{\psi} \mid g \ket{\psi} =
        \ket{\psi}\right\}\\
        &= V_{S,\sigma} \cap \bigcap_{g \in S_\rho \setminus
        S_\sigma} \left\{\ket{\psi} \mid g \ket{\psi} =
        \ket{\psi}\right\}\\
        &\subseteq V_{S,\sigma} = \mathrm{supp}(\sigma)
  .\end{align*}
  which finishes the proof of this direction.

  \enquote{$\Rightarrow$} Let $V_{S,\rho} = \mathrm{supp}(\rho) \subseteq
  \mathrm{supp}(\sigma) = V_{S,\sigma}$. 

  \begin{tikzpicture}
  \matrix (m) [matrix of math nodes,row sep=3em,column sep=4em,minimum width=2em]
  {
    H^{\otimes N} & V_\rho \\
     V_\sigma &  \\};
  \path[-stealth]
    (m-1-1) edge node [left] {$P_\sigma$} (m-2-1)
            edge node [above] {$P_\rho$} (m-1-2)
    (m-2-1) edge node [below] {$P_\rho$} (m-1-2);
  \end{tikzpicture}

  We can read above diagram as follows: From the $N$-qubit Hilbert space we can
  project onto the vector space stabilized by $S_\sigma$, $V_\sigma$, by means of a
  projection operator. Likewise we can do the same for $V_\rho$. Since we
  require $V_{S,\rho} \subseteq V_{S,\sigma}$, the same projection that takes
  us from $H^{\otimes N}$ to $V_{S,\rho}$ will take us from $V_{S,\sigma}$ to
  $V_{S,\rho}$. It follows that $P_\sigma P_\rho = P_\rho$.

  Let $\ket{\psi} \in V_\rho$. We thus have
  \[
    \ket{\psi} = P_\rho \ket{\psi} = P_\sigma P_\rho \ket{\psi} = P_\sigma
    \ket{\psi}.
  \]
  Since $\ket{\psi}$ was an arbirtary element from $V_{S,\rho}$ and $P_\sigma =
  \frac{1}{2^n} \prod_{i=1}^n \left(\mathds{1} + h_i\right)$ with $h_i \in
  S_\sigma$ it follows that all stabilizers of $\sigma$ also stabilize $\rho$
  and thus
  \[
    S_\sigma \leq S_\rho.
  \]

%  $\ket{\psi} \in V_{S,\rho}$ and $\ket{\phi} \in
%  V_{S,\sigma} \setminus
%  V_{S,\rho}$. \"o Since the stabilizer group is the
%  symmetry group of $V_S$, we can rewrite $S_\sigma$ as
%  \[ 
%    S_\sigma = \left\{ g \in G_N \mid g \ket{\psi} = \ket{\psi} \wedge g\ket{\phi} =
%  \ket{\phi} \right\},
%  \]
%  (where $G_N$ is the $N$-qubit Pauli group) 
%  as we want $S_\sigma$ to stabilize both $V_{S,\sigma}$ and its subset
%  $V_{S,\rho}$. We can then split up the condition into the intersection of two
%  sets, i.e.
%  \begin{align*}
%    S_\sigma &= \left\{ g \in G_N \mid g \ket{\psi} = \ket{\psi} \wedge g\ket{\phi} =
%  \ket{\phi} \right\}\\
%             &= \underbrace{\left\{ g \in G_N \mid g\ket{\psi}= \ket{\psi}
%             \right\}}_{S_\rho} \cap \left\{ g \in G_N \mid g\ket{\phi} = \ket{\phi} \right\} \\
%             &= S_\rho \cap \left\{ g \in G_N \mid g\ket{\phi} = \ket{\phi} \right\} \\
%               &\leq S_\rho 
%  .\end{align*}
  This concludes the proof.
\end{proof}
This begs the question: Can we derive a simple expression for relative entropy
in the case of $\rho$ and $\sigma$ being stabilizer density matrices?
It turns out that we can.

\begin{itemize}
  \item $\rho$ and $\sigma$ are both stabilizer density matrices
  \item We have a finite cross entropy iff. $S_\sigma \leq S_\rho$
  \item Stabilizer density matrices can be written as a product of projectors
    onto their generators, i.e.
    \begin{align}
    \rho = \frac{1}{2^N}\prod_{i=1}^r \left(\mathds{1} + g_i\right)
    ,\end{align}
    with $r\leq N$.
  \item Since $S_\sigma \leq S_\rho$, we can construct a generating set
    $G_\sigma$ of $S_\sigma$, where each element of $G_\sigma$ commutes with
    the generating set of $G_\rho$.
  \item Since the respective density matrices are constructed from the group
    generators, it follows that
    \begin{align}
      \left[\rho, \sigma\right] = 0
    .\end{align}
  \item It is a well-known fact from linear algebra that matrices that commute
    can be diagonalized simultaneously. A less-known fact is that the matrix
    log of a diagonalizable matrix is $\log\sigma = U \log D_\sigma U^{-1}$. We
    thus get
    \begin{align}
      -\Tr[\rho \log\sigma] = -\Tr[U D_\rho U^{-1} U \log D_\sigma U^{-1}] =
      -\Tr[D_\rho \log D_\sigma] 
    .\end{align}
  \item With $D_\rho$ and $D_\sigma$ diagonal we can write the trace as
    \begin{align}
      -\Tr[D_\rho \log D_\sigma] = -\sum_i \lambda_i \log \mu_i
    .\end{align}
  \item $\lambda_i$ will be uniformly distributed ($\rho$ stabilizer), and thus
    $\lambda_i = 2^{r-N}$ for $i=1,\ldots,2^{N-r}$ and $0$ otherwise.
  \item Each non-zero entry in $D_\rho$ corresponds to a non-zero entry in
    $D_\sigma$. 
  \item The sum then becomes
    \begin{align}
      -\sum_i 2^{r-N} \log 2^{r'-N} \mathds{1} =
      -2^{N-r}2^{r-N}\left(r'-N\right)= N-r'
    .\end{align}
  \item As a corollary we have a form for relative entropy in the stabilizer
    formalism
    \begin{align}
      S(\rho \mid\mid \sigma) = r - r'
    ,\end{align}
    where $r$ is the rank of $S_\rho$ and $r'$ is the rank of $S_\sigma$.
  \item The pertinent inequality for our problem, \cref{eq:kleins-ineq}, now
    becomes
    \begin{align}
      S_E \leq N-r'
    .\end{align}
\end{itemize}

\clearpage
\subsection{\enquote{Failure} of the na\"ive approach}
Hier auch auf jeden Fall n Plot mit dem naive approach drin
\begin{itemize}
  \item subgroup condition requires errors to be detected (and dealt with in
    some way)
  \item errors which are not succeeded by a measurement will lead to a change
    in the group structure of $S_\rho$ such that $S_\sigma$ \enquote{never has
    the chance to adapt}
  \item maybe we should also implement some ideas from the shadow tomography
    dings, such that errors are circumvented
  \item Undetectable change in group structure happens if an error happens on a
    qubit in a state stabilized by an orthogonal operator, e.g. an $X$-error if
    the qubit is stabilized by $Z$.
  \item The probability of any error happening in the last layer (i.e. after
    all measurements in the circuit happenend) is $1-$ the probability of no
    error, which is given by the binomial distribution
    \[ P(\#\mathrm{err}>0) = 1-\binom{N}{0} (1-q)^N=1-(1-q)^N. \]
  \item For $q=\num{0.01}$ and $N=32$ we have $P\simeq .275$
  \item Of course, this probability says nothing about the actual simulation
    (except for $p=0$) as it could be an error which was preceeded by an $X$
    measurement.
  \item That being said, you could technically compute this probability
    analytically, by going through each layer and multiplying the probabilites
    for \(X\) and not $ZZ$ happening to get the actual probability of failure
   \item This is way too complicated, just to get an estimate on how many
     simulations will end in $\infty$ for \cref{eq:kleins-ineq}. 
\end{itemize}
\clearpage
\section{Other Approaches}
\subsection{Algorithm \#1}
\subsection{Algorithm \#2}
\begin{itemize}
  \item We would like to compare $\rho$, which is the \enquote{experiment}
    (i.e. a quantum simulator), with $\sigma$, which is the classical
    simulation.
  \item The algorithm follows a similar structure to
    \cite{liCrossEntropyBenchmark2023}, wherein we project the measurement
    outcomes from $\rho$ onto $\sigma$ where it is possible. The difference
    being that we do not start with orthogonal GHZ states, but with identical
    GHZ states. That way, there will be no incompatible measurement as long as
    there are no errors
  \item but what if there are errors?
  \item consider the case that an $X$-error occurred after a $ZZ$-measurement.
  \item Half of the time we will get an incompatible projection, which would
    project onto the zero-vector.
  \item In those cases, we replace the stabilizer by $\mathds{1}$, where the
    incompatibility was detected, instead of replacing it by the measurement
    outcome or the projection.
  \item This ensures that the subgroup condition still holds \emph{and} that
    subsequent measurements on the same site cannot be interfered with, i.e.
    that this qubits stabilizer is $\mathds{1}$ until it gets measured again.
  %\item However, we rolled a nat20 on a sleight of hand skill check: in the
  %  cases, where we \emph{would} get $\infty$, we replace $\sigma$ (whatever
  %  it may be at this point) by $2^{-N} \mathds{1}$
  %\item This is also in line with the general philosophy of the density
  %  matrix. We do not know where the error occurred, and are therefore in a
  %  mixture of all possible quantum states
\end{itemize}
\section{Results}
\section{Summary}

\clearpage
\section{some relevant references}
\begin{enumerate}
  \item \textcolor{red}{\citetitle{aaronsonImprovedSimulationStabilizer2004}
      fuck why didn't i read it in its entirety earlier: section 7 does the
    job\ldots}
  \item \citetitle{veitchResourceTheoryStabilizer2014}, entropic quantities of
    magic states; quantifying non-stabilizerness. This paper is with gottesman
    himself, so i thought it might be interesting to study. 
  \item \citetitle{niekampEntropicUncertaintyRelations2012}, more or less what
    it says. No regularization scheme or whatever. I think stabilizers are
    mentioned tangentially, thats why i found it.
  \item \citetitle{lashkariRelativeEntropiesConformal2014}, $\leftarrow$ me
    grasping for straws.
  \item \citetitle{leonePhaseTransitionStabilizer2024}, very recent, not too
    particularily relevant to our stuff as far as i can tell.
  \item \citetitle{wuEntanglementUpperBound2011} idfk\ldots
  \item \citetitle{vedralEntanglementMeasuresPurification1998} one of the more
    promising papers i've come across. probably helpful in proving the failure
    of the na\"ive approach. However, it was published in 1998. It most likely
    has nothing to do with stabilizers.
  \item \citetitle{lindenQuantumEntropyCone2013}, probably some interesting
    stuff in there, but not the thing i want.
  \item \citetitle{buExtremalityStabilizerStates2024}, this paper is where most
    of my confusions come from. 
\end{enumerate}
