\chapter{Fragments}
\label{ch:fragments}

\section*{On tangents --- or: what is this chapter?}
While writing this thesis my thoughts have wandered off once or twice. Almost
always while writing out something completely different. Incidentally, this is
how the paragraph you're reading right now came to be.

Sometimes I'd write down something that amuses me, sometimes I'd get frustrated
and investigate my frustrations in text. Maybe I had even had a profound
thought once in a while.  Regardless of how these tangents got there, or what
they are about, off-topic tangents are not something you'll want to put in the
main body of your thesis, no matter how ingenious or witty they are.  Come to
think of it, they probably do not belong in the appendix either.

Either way, I would have been quite distraught to delete these paragraphs
without any of them ever seeing the inside of a printer. So, in order for them
not to go to waste, I decided to collect them here for mine and my friends'
amusement. These texts really are silly sometimes.

\section*{On references}
I like how references are provided in physics: all those little numbers in
little brackets or as superscript above a claim to support it. It allows me to
follow the bibliography in parallel to the paper and check some pertinent
publications easily. This is, of course, very different in many other
disciplines, e.g. psychology or philosophy, where it is common-practice to give
your sources in alphabetical order sorted by the author's surname.

I claim that physicists are not appreciative enough of this. It will never fail
to frustrate me if people cite some sh-t that has nothing at all to do with the
claim in question. This is especially so if the claim is better supported by
the results of other papers; papers that a lot of times make it into the
\texttt{.bib} file of at least one of the authors, or why else would they be
prominently cited in a section, where it doesnt make sense to do so.
Ultimately, the highest degree of frustration for me is if the references are
off by one, or cite a previous project by the same authors. This kind of
situation lies in the uncanny valley of "is it them or me?". And more often
than not, it should be \emph{me}, right? But then you go on and on, read the
cited papers thoroughly, try to understand the context in which it is
referenced in the work you originally went through, and now, multiple layers
deep in the reference rabbithole are you ultimately forced to realize that it's
not you, it's them. 

%\section*{On the sampling problem}
%
%The metaphysics of this endeavour (i.e. this thesis) can
%be condensed in the following way; we (a) know from experiments how quantum
%systems behave under certain conditions, and (b) predict through theoretical
%calculations what these systems might do in another experimental setting. In
%the latter case however, there is an uncanny regime of utility, where we either
%(a) cannot precisely pass predictions or (b) cannot perform the experiment on
%the grounds of hardware limitations\footnote{The Higgs particle was predicted
%40 years before it was discovered \textcolor{orange}{citation}} or (c) try to
%predict the behavior of quantities not directly measurable. In the case of
%quantum computation, and especially in the field of entanglement transitions,
%we face these bottlenecks in increasing severity. To make do with them, we
%employ classical computer simulations. That is, we perform numerical
%experiments. While traditional experiments still serve as the sole proprietor of claim to
%ontology, numerical experiments can play a supporting role, 
%
%In the year of our
%lord 2024, we phyisicists are thankfully able to perform experiments at home
%with cleverly assembled silicon.  That is, nowadays we make do with these
%bottlenecks by performing numerical experiments. This is not to discredit the
%utility of experiments as such, on the contrary! We want to assist our
%experimental colleagues by providing more accurate predictions for them to
%test. 
%
%However, the predictions sometimes surpass even the limitations of nature
%itself. 

\section*{On the utility of simulations}
One of my favourite video games is \emph{Cities: Skylines}. As the name
suggests, it is a city-building game (which graciously runs natively on Linux).
In it, you act as an almost omnipotent mayor and are tasked with developing and
maintaining a settlement raised from scratch. Initially, one has a single plot
of buildable area, located in a vast landscape, acting as a canvas for you to
design your city as you see fit. The initial plot of land usually conveniently
features a highway exit, where the canonical first step is the design of a road
network with appropriate residential, commercial, and industrial zoning for
citizens to move in, and for industry and businesses to ship cargo in and out.
As the game progresses your city grows, unlocking you more buildings and services
such as schools, hospitals, or public transit. (If the populace is sufficiently
educated, one can even unlock a \enquote{Large Hadron Collider} in the later
stages of the game.) However, one also faces the challenges of a growing city,
such as the increasing demands for basic utilities, and most importantly,
traffic congestion and its consequences.  All this is to say: it simulates the
progression from a village to a metropolis. 

At the end of the day, however, it is only a simulation of what is happening in
the real world. While the developers made sure to accurately model behavior of
the masses, daily routines of thousands---if not millions---of citizens, and
their commutes constituted by about as much vehicles, they at the same time
made sure that I can play this game on my used ThinkPad, more than 10 years of age. It is
therefore only natural that some mechanisms of human behavior had to be
\enquote{coarse grained} away for the game to be (a) playable and (b)
enjoyable. (Think about walking around in Stuttgart: construction sites as far
as the eye can see. Confer \emph{Cities: Skylines}, where buildings just
spawn out of nowhere as soon as you command them to be placed.)
It would be highly absurd to suggest that every nuance of life in a city can
be mapped one to one to a program.

Yet as it turns out, the game does a surprisingly good job of accurately
depicting traffic flow and the dynamics of city life, now being used to teach
urban planning and landscaping students\footnote{Educational software, now in
your local steam library.}
\cite{khanPerceptionsStudentsGamification2021,haahtelaGamificationEducationCities2015}.
Picture this: Cleverly assembled instructions run on cleverly assembled silicon
circuits teaching urban planning students that the road network they conceived
of was not-so cleverly assembled. An almost kafkaesque sight to behold.

Now, what can we as physicists learn from this excursion into my video game
preferences? The daily routine of many physicists today, even at the LHC,
consists of running simulations on (almost) the same x86 instruction set that
allows me to build the Large Hadron Collider in my virtual city with the press
of a button. That is, the computer
hardware is completely agnostic to the simulation it is running. Matter of
fact, within the confines of computability, we can play God with the processors,
making them produce any result we want.

This begs the question: what kind of results \emph{should} we want from a
simulation? Remember, the truly remarkable thing about the game Cities:
Skylines is the way it accurately portrays \emph{the real world}, while still
being a video game. Should our demands on simulations of natural phenomena
therefore be held to the same standard? If that were strictly so, a radical conclusion
one might draw is that the epistemic value of a simulation that models
something not translateable to an experiment is nill. It implies that
the simulation tells me about as much about the phenomenon it is supposed to
model as the following \texttt{C} code:
%That is, we want simulations to accurately predict outcomes of experiments. A
%radical conclusion we might draw from this is that a simulation that models
%something not translateable to an experimental setup tells me about as much
%about the system it is supposed to model as the following \texttt{C} code:
\begin{verbatim}
#include <stdio.h>
int main(int argc, char *argv[]) {
  printf("Hello World!");
  return 0;
}
\end{verbatim}
Or maybe it tells me that whatever phenomenon whose existence I advocate for,
is not \enquote{real}---whatever that means for you.

I claim that this, albeit a straw man argument, is short-sighted. The
extrapolation from well-established models to potentially untestable theories
that only exist on paper or computer simulations \emph{can} have epistemic
value.\footnote{Maybe this is just me coping with the fact that there is no
pompous publishable result in my thesis, but I digress.} One might
still be able to answer \emph{how} they are untestable, potentially gaining
insight in the workings of nature by understanding the restrictions it
shackles us with. 

\section*{On Cargo Cult Science}
In my time at the institute I was fortunate enough to be able to attend the
group retreat. For this I prepared a talk on Feynman's notion of Cargo Cult
Science and how it connects to Max Horkheimer and Theodor W. Adorno's Dialectic
of Englighenment. This might be the biggest tangent i went on, since I wrote an
entire transcript of what I was going to say in the talk. This, of course,
isn't an exact transcription of what was presented, but a rough outline and
still a major tangent when writing this thesis.

\subsection*{The Talk}
Hello everyone, it humbles me to start this day of talks by fellow members of
this institute with some topic merely tangentially related to the field of
theoretical physics. Nonetheless, I hope you'll enjoy the talk and the topic interesting
and thought-provoking. 
And alas, we will not leave the realm of theory: As you
can see, the title of my talk is \emph{Cargo Cult Science and the Dialectic of
Enlightenment}, and in it I wanna give a brief, and hopefully entertaining,
overview of critical theory's foundations.

To set the stage, I’d like to begin with a quote from a figure we all hold in
high esteem: Richard Feynman. In Caltech’s 1974 commencement address, he coined
the term I used for the first half of this talk's title. It is
a reflection that resonates suprisingly well with the theme of today’s talk. He
said \cite{feynmanCargoCultScience}:
%I wanna start this talk by giving you a quote from Richard Feynman. In
%Caltech's 1974 commencement address he had this to say: 
\blockquote{
  During the Middle Ages there were all kinds of crazy ideas, such as that a
  piece of rhinceros horn would increase potency. Then a method was discovered
  for separating the ideas--- which was to try one to see if it worked, and if
  it didn't work, to eliminate it. This method became organized, of course,
  into science. And it developed very well, so that we are now in the
  scientific age. [\ldots]

  But even today I meet lots of people who sooner or later get me into a
  conversation about [\ldots] some form of mysticism.
}

Feynman, whose brilliance in both physics and pedagogy I think we all deeply admire,
captures a central tension that has persisted throughout the history of human
thought. His observation, highlighting the transition from superstition to
science, echoes the broader narrative of the Enlightenment---a period that
heralded the triumph of reason and empirical inquiry over myth and mysticism.

%Indeed, the scientific method, which we hold as the bedrock of our work, is a
%direct legacy of Enlightenment thought. It embodies the Enlightenment’s thesis:
%that through reason and evidence, we can uncover the truths of the natural
%world, standing in stark contrast, in antithesis, to the belief systems of
%UFOs, astrology, and other forms of pseudoscience.

Because surely, this method, the scientific
method as it were, which we hold as bedrock of our endeavors,
is a product of the Enlightenment era, of enlightenment
thought. And surely, this thesis of pure reason and rationality, stands in
contradiction, in antithesis, to UFOs or astrology.
%As someone who views Feynman as one of the greatest physicists and a humbling
%idol, I was curious about how the sentiment expressed in the opening of his
%speech mirrors that of a Dialectic of Enlightenment. 

But here’s where it gets interesting: the story of the Enlightenment is not as
straightforward as a simple victory of light over darkness.  As it turns out,
two german philosophers, Max Horkheimer (who was born in Zuffenhausen) and
Theodor W. Adorno have decades prior also grappled with the complexities and
contradictions inherent in Enlightenment thought. So much so that, when they
were forced into exile by the emergence of fascism in the form of national
socialism, they wrote a book about it, entitled \emph{Dialectic of
Enlightenment}.

In this work, Horkheimer and Adorno argue that the very rationality and
scientific progress that emerged from the englightenment era, contain within
them the seeds of their own undoing. Or to say it in their words
\cite{horkheimerGesammelteSchriftenBand1987,horkheimerDialecticEnlightenmentPhilosophical2002}:
\blockquote{
  Seit jeher hat die Aufkl\"arung im umfassendsten Sinn fortschreitenden Denkens
  das Ziel verfolgt, von den Menschen die Furcht zu nehmen und sie als Herren
  einzusetzen. Aber die vollends aufgekl\"arte Erde erstrahlt im Zeichen
  triumphalen Unheils.
} 
\blockquote{
Enlightenment, understood in the widest sense as the advance of
thought, has always aimed at liberating human beings from fear and
installing them as masters. Yet the wholly enlightened earth is radiant with
triumphant calamity. 
}

\subsubsection*{Enlightenment}
So what \emph{is} a \emph{Dialectic of Enlightenment}? Well, let us maybe begin
with the Age of Enlightenment. If I would do this talk in german, I could refer
to the english translation of the term \emph{Aufklärung}. It is sometimes
referred to as the \emph{Age of Reason} \cite{bristowEnlightenment2023}, and caused
quite a ruckus in 17th and 18th century europe. Think of the american and
french revolution. The rise of the first democracies, secularism and also:
industrialization. The invention of the steam engine and so on.
The most prominent
englightenment thinkers include characters such as Immanuel Kant, Francis
Bacon, David Hume and this one, I am sure you are all well familiar with the
coordinate system named after him.

Enlightenment at its core represented a radical shift away from the teachings
of priests and witch doctors, explanations from religion and myth.
Enlightenment, in Kant's words, is \enquote{the human being's emergence from
self-incurred minority. Minority is inability to make use of one's own
understanding without direction from another.}; \enquote{Understanding without
direction from another} is understanding guided by reason
\cite{kantBeantwortungFrageWas1784,kantKritikReinenVernunft1781}. So instead of
relying on horror stories, the likes of Galileo, Newton or Kepler made use of
their own understanding without direction from another.

Francis Bacon, one of the intellectual architects of this movement, famously
argued that mythical thinking stood in the way of what he called \enquote{the happy
match between the mind of man and the nature of things}
\cite{baconWorksFrancisBacon2011}. For Bacon, and indeed
for many Enlightenment thinkers, myths and superstitions were obstacles that
clouded human understanding. The goal was to clear these obstacles, to purify
thought by aligning it more closely with the natural world, which could be
understood through careful observation, experimentation, and the application of
reason.

%But this project of disenchantment was not without its complexities and
%contradictions, which Horkheimer and Adorno were keen to explore. While the
%Enlightenment sought to liberate humanity from the shadows of ignorance, it
%also introduced a new way of thinking that could, paradoxically, lead to new
%forms of domination. The very tools of reason and science that promised to free
%us could also be turned into instruments of control—subjugating not only nature
%but, potentially, human beings themselves.

%This dialectical process—the tension between liberation and domination, between
%reason and its potential for unreason—is what Horkheimer and Adorno sought to
%unpack in their critique. As we will see, the Enlightenment's legacy is a
%complex one, filled with both remarkable achievements and profound challenges.
%
%As Horkheimer and Adorno write, Enlightenments program was the disenchantment
%of the world. It wanted to dispel myths, to overthrow fantasy with knowledge.
%They cite Francis Bacon, according to whom this mythical thinking stood in the
%way of \enquote{the happy match between the mind of man and the nature of
%things}.

But as you could probably guess, this profound shift in thought was not without
its complexities and contradictions. After all, why would someone as deeply
embedded in the scientific tradition as Richard Feynman identify a phenomenon
like Cargo Cult Science if we are truly living in a scientific age? Did the
Enlightenment not champion ideals of liberty, progress, and rationality that
should have eradicated such pseudo-scientific practices?

\subsubsection*{Instrumental reason}
Horkheimer and Adorno would argue that yes, the Enlightenment indeed promoted
these ideals. It encouraged a pursuit of knowledge that liberated humanity from
the shackles of superstition and myth, setting the stage for unprecedented
progress in science, technology, and human understanding. However, they also
caution that this progress came with unintended consequences. Alongside the
liberation that Enlightenment thought provided, it also introduced new forms of
control and domination, encapsulated in what they describe as \enquote{instrumental
reason}.

To quote the Dialectic once more, Horkheimer and Adorno write: 
\blockquote{
  Myth becomes enlightenment and nature mere objectivity. Human
beings purchase the increase in their power with estrangement from that
over which it is exerted. Enlightenment stands in the same relationship to
things as the dictator to human beings. He knows them to the extent that
he can manipulate them. The man of science knows things to the extent
that he can make them. Their \enquote{in-itself} becomes \enquote{for him}.
}

Instrumental reason refers to a way of thinking that prioritizes efficiency,
utility, and control above all else. It’s a kind of reasoning that asks not
\enquote{What is true?} but rather \enquote{What works?} or \enquote{What can
we do with this?} While this approach can lead to remarkable technological
advancements and practical applications, it can also reduce human thought to a
mere tool for manipulating the world, stripping it of deeper meaning or ethical
consideration \cite{horkheimerZurKritikInstrumentellen2007}.

A seemingly harmless manifestation of this instrumental reason might be
familiar to many of us—think of the gut reaction from your relatives when they
ask you what you’re currently working on, perhaps during a holiday dinner.
After hearing your explanation, they might immediately ask, "What can you do
with it? How can it be applied?" This reaction, though well-meaning, reflects a
broader societal tendency to value knowledge primarily for its practical
utility rather than for its intrinsic worth or the deeper understanding it
provides.

And even Feynman can provide us with an example for instrumental reason: In the
commencement address he says this while talking about scientific integrity:
\blockquote{ For
example, I was a little surprised when I was talking to a friend who was going
to go on the radio. He does work on cosmology and astronomy, and he wondered
how he would explain what the applications of this work were. \enquote{Well,}
I said, \enquote{there aren't any.} He said, \enquote{Yes, but then we won't
get support for more research of this kind.} I think that's kind of dishonest.}

This instrumental mindset, while useful in many contexts---and I cannot
stress this enough: Yes of course, the human condition has been immensely
improved by modern medicine and the likes---, also has the
potential to narrow our vision. It can lead to a world where the worth of ideas
is measured solely by their immediate applicability, where knowledge is pursued
not for the sake of enlightenment but for the sake of control—whether that be
control over nature, over society, or even over ourselves.

In this way, the Enlightenment's legacy is double-edged. While it has given us
the tools to understand and shape the world in ways that were previously
unimaginable, it also carries the risk of reducing all human thought and
endeavor to mere instruments of utility, potentially leading us down a path
where the original humanistic goals of the Enlightenment—liberty, progress, and
the pursuit of truth—are overshadowed by the drive for efficiency and control.

Think of the developments of the information age: personal computers,
cellphones, the internet. What has started in the pursuit of science has ended
up within the total computability of human behavior. The incommensurable
rigidity of the bureaucratic apparatus is also one striking example we can
name. It is the result of a long tradition of rationalization. One of my
favourite movies, Brazil by Terry Gilliam from 1985, brings this fetishization
of bureaucracy to its radical, fascistoid conclusion.

\subsubsection*{Cargo cult science}
This is but one of the Dialectics of Enlightenment, the thesis of liberation
and the antithesis of control. To finish off, let us now return back to Feynman
and his Cargo Cult Science. His idea is not too far off from the ones of
Horkheimer and Adorno, in the sense that an element of the mythical is still
remanent in the age of science. We have pseudoscience co-opting the language,
the vocabulary of the \enquote{proper} sciences, and they even experiment and
observe to test for their hypotheses. 

This persistence of the mythical within the scientific age is a crucial point.
Feynman’s concept of Cargo Cult Science is not merely about quackery or obvious
pseudosciences; it is a warning about the dangers that arise when the form of
scientific inquiry is imitated without the substance.

In Cargo Cult Science, the methods of science—experimentation, observation, and
even the use of scientific terminology—are employed, but they are disconnected
from the rigorous skepticism, critical thinking, and openness to
disconfirmation that define genuine scientific inquiry. The result is something
that looks like science on the surface, but lacks the integrity and depth of
true scientific understanding.

This is where the connection to Horkheimer and Adorno's critique becomes
particularly relevant. Just as the Enlightenment's promise of liberation can be
perverted into new forms of control, so too can the tools and language of
science be co-opted in ways that undermine its true purpose. The rise of
pseudoscience is a testament to this dialectical tension. Pseudoscience often
flourishes not in opposition to science, but by mimicking it, by adopting its
outward forms while stripping away its inner rigor.

Take, for example, the spread of misinformation in the digital age. The
internet, a product of Enlightenment ideals about the free exchange of
information, has become a breeding ground for both genuine scientific knowledge
and dangerous pseudoscientific ideas. The same technology that allows us to
access vast repositories of knowledge can also lead us down rabbit holes of
conspiracy theories and falsehoods, all cloaked in the guise of scientific
legitimacy.

Moreover, the bureaucratic rationalization we discussed earlier plays a role
here as well. In the rush to measure, quantify, and control every aspect of
human life, we may inadvertently create environments where pseudoscientific
ideas can thrive. The pressure to produce results, the emphasis on measurable
outcomes, and the bureaucratic obsession with procedures over substance can all
contribute to a culture where Cargo Cult Science becomes increasingly
prevalent.

Neural networks, machine learning algorithms, and large language models are the
most recent culmination of Enlightenment thinking. However, every time we open
the ‘For You page’ on our silicon-controlled block of glass and plastic, we
find ourselves incapacitated anew.

\subsubsection*{wrapping up}
So, what does this mean for us as scientists? It means we must remain vigilant.
The Enlightenment gave us powerful tools for understanding the world, but it
also gave us the responsibility to use those tools wisely. We must constantly
remind ourselves of the difference between the appearance of science and its
reality, between the mere performance of scientific procedures and the genuine
pursuit of knowledge.

In conclusion, Feynman’s cautionary tale about Cargo Cult Science is more
relevant than ever. As we navigate the complexities of the modern world, filled
with both the promise of technological advancement and the perils of
pseudoscience, we must strive to uphold the true spirit of the Enlightenment.
This involves not only applying rigorous scientific methods but also engaging
in critical self-reflection, ensuring that our pursuit of knowledge remains
aligned with the ideals of reason, progress, and, ultimately, human freedom.
