\chapter{Abstract}
\label{ch:intro}

\begin{itemize}
  \item Entanglement transitions are cool
  \item Entanglement is backbone of quantum computing
  \item Something something quantum error correction
  \item Entanglement exponentially hard to measure (requires exponentially many
    copies of state)
  \item Classical simulations predict phase transition in many hybrid random
    circuits
  \item PTIM \cite{langEntanglementTransitionProjective2020} measurement only
    circuit, that maps to majorana fermions, the colored cluster model and
    2-dimensional percolation
  \item 2D percolation and the CCM has been widely studied, from this we
    predict there to be the quantum phase transition from one area law to the
    other
%  \item \textcolor{red}{check das mit dem area law ding auf jeden fall nochmal,
%    ich glaub da wird schon auch wert drauf gelegt}
  \item No way to measure entanglement in experiment, since we can only measure
    observables
  \item Many attempts of probing the critical point of the phase transitions
    have been made
  \item Fisher's LXE: \cite{liCrossEntropyBenchmark2023}
  \item Fisher's LXE for PTIM:
    \cite{tikhanovskayaUniversalityCrossEntropy2023}
  \item In-House attempts also exist:
  \item The decoder in \cite{roserDecodingProjectiveTransverse2023} lower
    bounds the critical point (decoding threshhold)
  \item Can we upper bound it?
  \item In principle yes, klein's inequality (goes by \enquote{gibb's
    inequality} and \enquote{information inequality} as well) upper bounds
    von Neumann, and consequently entanglement entropy.
  \item In \cite{garrattProbingPostmeasurementEntanglement2023} this is done
    with hybrid circuits.
\end{itemize}

This thesis consists of four main chapters and is structured as follows
\begin{description}
  \item[\cref{ch:basics}] In this chapter we give an introduction to the
    concepts relevant throughout the thesis. We begin by outlining the
    stabilizer formalism, as it is a useful representation of quantum states
    and the foundation of the simulation algorithm employed for the numerical
    approaches in the thesis. We then provide an overview of the concept of entanglement
    transitions in general, before specifically introducing the projective
    transverse-field Ising model. This model serves as the primary subject of
    our studies. We then discuss the post-selection or sampling problem,
    motivating the rest of the thesis.
  \item[\cref{ch:lxe}] In this chapter we examine the quantity defined by
    \citeauthor{liCrossEntropyBenchmark2023} in the context of the projective
    transverse-field Ising model. We particularily motivate post-selection
    algorithms and compare our results to
    previous investigations of similar nature (cf. Ref.
    \cite{tikhanovskayaUniversalityCrossEntropy2023}), while also raising
    critiques.
  \item[\cref{ch:rel-ent}] In this chapter we investigate the approach of
    \citeauthor{garrattProbingPostmeasurementEntanglement2023} to detect the
    entanglement transition in the projective transverse-field Ising model. We
    first derive an expression for the cross- and relative entropy in the
    stabilizer formalism, before trying different numerical approaches to
    circumvent the post-selection problem. We finalize the chapter by outlining
    two regularization approaches to deal with appearing infinities.
  \item[\cref{ch:mixed}] In this chapter we introduce the simulation algorithm,
    which is based on the stabilizer formalism outlined in \cref{ch:basics}. We
    first give a brief summary of the existing algorithms. We then introduce
    the extensions necessary for the numerical simulations performed throughout
    the thesis, especially concerning mixed states and the different entropies
    derived in \cref{ch:rel-ent}.
\end{description}

\lipsum[0-1]
