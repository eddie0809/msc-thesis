\chapter{Implementing mixed states}
\label{ch:mixed}
\epigraph{Always programming a new type of antidote in your perimeter}{
\citeauthor{quasimotoDiscipline99Pt2000}
%on \citefield{quasimotoDiscipline99Pt2000}{title}
}

Ref. \cite{aaronsonImprovedSimulationStabilizer2004} will need to do some heavy
lifting in giving primers on stabilizer simulation.

It will later prove useful to implement mixed states into the existing
stabilizer simulator. As of yet, this does not exist.

In this paper \cite{audenaertEntanglementMixedStabilizer2005} they present an
algorithm to perform a partial trace on a stabilizer state. They need an
auxilliary algorithm, the Reduced row echelon form algorithm. I think this is a
promising approach, but needs implementation of the RREF and the ptrace
algorithm. From what I've seen, the first one is going to take longer than the
second to implement. Maybe look for this kind of alg in the existing codebase?

Note from 10 minutes later: its basically what we already had, only a bit more
formalized. 

Lecture notes on stabilizers: \cite{arabLectureNotesQuantum2024}

This chapter serves as a primer on the nitty-gritty details of simulating
stabilizer circuits on a classical computer. As already touched on in
\cref{ch:basics}, according to the \emph{Gottesman-Knill Theorem} we can
perform efficient classical simulations of quantum circuits using only Clifford
gates. In the following sections we will introduce the so-called tableau
algorithm, which allows simulations in polynomial time, and expand it to
properly deal with mixed states. We assume the reader to be familiar with the
corresponding sections in \cref{sec:stab-basics}.

\subsubsection*{A complexity theory primer}
%%% WAS HIER URSPRÜNGLICH MAL STAND:
% This chapter addresses classical computation and algorithms, focussing on the
% simulation of a particular class of quantum circuits using classical computers.
% As such, it will necessarily feature terminology commonly associated with the
% field of computer science. While this thesis as a whole is a work on natural
% science – i.e. theoretical physics – our focus lies more on simulations as a
% tool rather than as the primary object of study. Nonetheless, to ensure
% clarity, we will briefly introduce key terminology from computer science.
%%%
This chapter is something of an outlier compared to the other chapters. While
the previous part of this work pertains to \emph{natural} science -- namely,
theoretical physics -- where simulations served us as a tool rather than as the
primary object of study, a substantial portion of this thesis has been
dedicated to expanding an existing stabilizer simulator to deal with new
problems (cf.  \cref{sec:lxe-numeric,sec:rel-ent-stab}). As a result, our
discussion thereof will necessarily feature
%cursed language devised by the utterly deranged: computer scientists. To
%prepare for this, we will introduce some of its vocabulary here.
terminology commonly associated with the field of \emph{computer} science.
To ensure clarity, we will briefly
introduce key computer science concepts relevant to our discussion.  

The first concept we introduce is asymptotic behavior, along with the so-called
\emph{Big O} notation. It is not exclusive to computer science as such, as we
often deal with asymptotic behavior in physics as well.  In computer science,
the Big O notation is used to describe the space (i.e. data storage) and
time requirements of algorithms with increasing input size. The notation is
defined precisely in \cref{defn:bigo-theta}.
\begin{defn}[$\Theta$ and $\mathcal{O}$ notation
  \cite{cormenIntroductionAlgorithms2009}]\label{defn:bigo-theta}
  Let $g(n)$ be a positive function. Then $\Theta(g(n))$ is the set of
  functions 
  \begin{align*}
    \Theta(g(n)) = \left\{ \ f(n) \mid \liminf_{n\to\infty} \frac{f(n)}{g(n)} > 0 \
    \wedge \ \limsup_{n\to\infty} \frac{f(n)}{g(n)} < \infty \ \right\}
  .\end{align*}
  Similarly, $\mathcal{O}(g(n))$ is the set of functions 
  \begin{align*}
    \mathcal{O}(g(n)) = \left\{ \ f(n) \mid 
    \limsup_{n\to\infty} \frac{f(n)}{g(n)} < \infty \ \right\}
  .\end{align*}
\end{defn}
We will later see some examples of this notation in use. For now it suffices to
note that different problems and the algorithms we use to solve them (or verify
the validity of a solution) are grouped into a variety of \emph{complexity
classes}. For instance, an algorithm is said to scale polynomially (in time) if
its asymptotic behavior is $\mathcal{O}(n^\alpha)$ with $\alpha \geq 1$, and
exponentially if it is $\mathcal{O}(2^n)$. These classes constitute an
important subset in the field of theoretical computer science, in the form of
complexity theory, where numerous classes exist.\footnote{A continuously
updated list can be found in \cite{ComplexityZoo}.} The two examples from above
fall in the classes \textsf{P} (polynomial time) and \textsf{EXP} (exponential
time) respectively.  While there are lots of caveats to this highly simplified
explanation, a deeper examination of complexity theory lies beyond the scope of
this thesis. However, the key takeaway is summarized in
\cref{defn:efficient-alg}.
\begin{defn}[Algorithmic efficiency \cite{ComplexityZoo}]\label{defn:efficient-alg}
  A problem is considered efficiently solvable on a classical computer if it
  belongs to the complexity class \textsf{P}.
\end{defn}
Having laid out these foundational concepts, we can now delve deeper into the
subtleties of stabilizers and the problem of simulating stabilizer
circuits, potentially gaining a new appreciation for their intricacies.

\section{Classical simulation of stabilizer circuits}\label{sec:sim-stab}
In this section we explore the implications of \cref{thm:gottesman-knill} and
provide a proper introduction to the simulation algorithm that forms the
foundation for the numerical experiments in this thesis.  In particular, we
present the \emph{tableau algorithm}, as proposed by Aaronson and Gottesman in
\cite{aaronsonImprovedSimulationStabilizer2004}.  Consequently, this section
draws largely from outside sources, namely refs.
\cite{aaronsonImprovedSimulationStabilizer2004, arabLectureNotesQuantum2024,
gottesmanStabilizerCodesQuantum1997,
gottesmanHeisenbergRepresentationQuantum1998}. As it serves an introductory
purpose, this section can (and should) be read as a sequel to
\cref{sec:stab-basics}.
%This section provides an introduction to the tableau algorithm. It can (and
%should)
%be read as the sequel to \cref{sec:stab-basics}. Note that this section is
%based on the work of \citeauthor{aaronsonImprovedSimulationStabilizer2004} in
%\cite{aaronsonImprovedSimulationStabilizer2004}.

We begin by recalling that the stabilizer group does not need to be stored in
full to unambiguously describe the state. Since the stabilizer group is finite,
its structure can be fully encapsulated by storing only its \emph{generators}
in memory. This reduces the amount of data to be stored to memory from $2^n$
to $n$, owing to the well-known fact from group theory that a finite group $G$
has a generating set of size $\log \abs{G}$. That is, an $n$-qubit pure state
$\ket{\phi}$ with stabilizer group $S\left(\ket{\phi}\right)$ has a generating
set of size $\log 2^n = n$.

To determine the actual memory requirements we examine the generators
themselves.  Each generator consists of an array of $n$ Pauli matrices and a
sign. Since there are $4$ Pauli matrices (including the identity), we require 2
bits to encode each of them, along with an additional bit for the sign.
Consequently, the memory requirements for encoding a pure state in the
stabilizer formalism is $n(2n+1)$. In other words, storing only the stabilizer
generators reduces the space complexity of stabilizer simulations from
$\mathcal{O}\left( 2^n \right)$ to $\mathcal{O}\left( n^2 \right)$.

At this point, we have merely stored the state in memory. However,
\cref{thm:gottesman-knill} implies that simulations of stabilizer cicruits can
also be done efficiently on a classical computer. As stated in
\cref{defn:efficient-alg}, for an algorithm to be
considered \enquote{efficient}, it must have polynomial time complexity.
Thus, the Gottesman-Knill theorem can be rephrased to state that the bits encoding
$\ket{\phi}$ can be updated in polynomial time after a Hadamard gate, phase
gate, controlled-\verb|NOT| gate, Pauli gate or Pauli measurement gate is
applied to $\ket{\phi}$.
\begin{thm}[Simulating stabilizer gates]\label{thm:sim-stab-comp}
  Simulating a Clifford gate on an $n$-qubit stabilizer state requires $\Theta(n)$
  time, while a measurement gate is simulated in $O(n^2)$ or $O(n^3)$ time for
  random and deterministic outcomes respectively.
\end{thm}
\begin{proof}
  An $n$-qubit stabilizer state $\ket{\phi}$ can be represented as an
  $n\times(2n+1)$ bit matrix $M$, where the rows are the $n$ generators in the
  afforementioned $(2n+1)$-bit representation. Any computational basis state can
  thus be represented by
  \[
    M = \mqty( 
    01 & 00 & 00 & \ldots & 00 & \pm \\
    00 & 01 & 00 & \ldots & 00 & \pm \\
    00 & 00 & 01 & \ldots & 00 & \pm \\
    \vdots & \vdots & \vdots & \vdots & \vdots & \vdots \\
    00 & 00 & 00 & \ldots & 01 & \pm \\
    ),
  \]
  where $Z$ and $I$ are encoded as $01$ and $00$ respectively, and $\pm$ is a
  placeholder for either $0$ or $1$ depending on the sign. Simulating a
  Clifford gate $U$ on $\ket{\phi}$ maps $g_i$, that is, the $i$-th row of $M$,
  to $U g_i U$. This operation updates at most two columns, since Clifford
  gates act on at most two qubits at once. Therefore, simulating $U$ takes $\Theta(n)$
  time.

  To show the scaling behavior of measurements we consider computational basis
  measurements without loss of generality. Each qubit in $\ket{\phi}$ is either
  in a computational basis state $\ket{0}$ or $\ket{1}$, where we have a
  deterministic outcome, or in a superposition
  of both states with equal amplitude, where either outcome is random with
  probability $p=1 / 2$. Recall from \cref{sec:stab-basics} that the process of
  determining the type of outcome (random or deterministic) is done by checking
  the commutation relations between the measurement operator and the generators
  of $\ket{\phi}$. If some generator anti-commutes the outcome is random,
  otherwise it is deterministic (this implies a scaling of $\mathcal{O}(n)$
  already, since we go through the entire matrix in the worst case). Suppose
  there is a generator $g_i$ that anticommutes with $Z_i$, implying a random outcome. As the
  outcome, we randomly choose $x\in\{0,1\}$. We then multiply any subsequent
  rows that anticommute with $Z_i$ by $g_i$ in order for them to commute with
  $Z_i$. We then replace $g_i$ by $\pm Z_i$ depending on the random outcome.
  Since this algorithm takes up to $n$ row multiplications, the runtime scales
  with $\mathcal{O}(n^2)$.

  If the outcome is deterministic, the 
\end{proof}
%But as it turns out, this improvement in space complexity is also accompanied
%by 
%
%
%We already know that we do not need the full stabilizer group to describe our
%state,
%This reduces the
%number of operators that need to be saved to $n=\log2^n$.
%Since stabilizer
%groups are built from Pauli matrices, we need $2n+1$ bits to specify 

\subsection{The Tableau Algorithm}\label{sec:tableau}
The name of the algorithm stems from the fact that a state is represented by a
\emph{tableau}. This tableau is a binary $2n\times2n+1$ matrix. 

\begin{prop}[Invariants of the tableau algorithm]\label{prop:comm-tab}
  The following are invariant under operations of the tableau algorithm
  \begin{enumerate}
    \item $R_{n+1},\ldots,R_{2n}$ generate $S(\ket{\phi})$, and $R_1, \ldots,
      R_{2n}$ generate $\mathcal{P}_n$.
    \item $R_1, \ldots, R_n$ commute.
    \item $\forall h \in \{1,\ldots,n\}, \ \{R_h, R_{h+n}\} = 0$
    \item $\forall i,h \in \{1,\ldots,n\}, \ \text{with } i\neq h, \ [R_i, R_{h+n}] = 0$
  \end{enumerate}
  
\end{prop}

One subroutine of major importance to the tableau algorithm is \texttt{rowsum}.

\subsection{\texttt{rowsum}}\label{sec:rowsum}

\section{Expanding the Tableau algorithm}\label{sec:expanding}
Tableau algorithm: \cite{aaronsonImprovedSimulationStabilizer2004}

\textcolor{red}{Hier w\"are es ganz nice f\"ur die ganzen algs so flowcharts
dabei zu haben, u.a. auch um listings aus dem weg zu gehen. Alg als flowchart
ist generischer als der (evtl unsauber geschriebene) \texttt{C++} code}

The tableau algorithm as outlined in \cref{sec:tableau} is only able to
simulate stabilizer circuits on pure states. This however, neglects the most
general class of quantum states, namely, mixed states. It will become apparent
in \cref{ch:rel-ent} why this extension is necessary for our purposes. 
%
%\subsection{sample flowchart!}
%delete this subsection later!
%\begin{figure}[H]
%  \centering
%  \input{fig/tikz/example.tex}
%  \caption{dings}
%  \label{fig:asdf}
%\end{figure}

\subsection{The \texttt{mix} attribute}
Each instance of \texttt{Qubit} now comes with a new attribute called
\texttt{mix}! Its like a buy one get one free deal!

\subsection{\texttt{get\_state\_type}}
The first algorithm we want to introduce is an auxilliary subroutine needed for
the \texttt{ptrace} algorithm, which we call \texttt{get\_state\_type}. It
takes a qubit position $a$ as input and outputs the number of unique stabilizer
generators minus one on that qubit. The name of the function stems from the
fact that we can have $3$ different state types: entangled, product and mixed.
A qubit with two unique stabilizer generators, i.e. $g_{ia} = Z$ and $g_{ja} =
X$ with $i\neq j$, will be in an entangled state with another qubit $b\neq a$.
If there is only one unique stabilizer generator, we have qubit $a$ in a
product state, where the state is the state stabilized by the generator.
Finally, no stabilizers correspond to a mixed state, since the empty set
generates the trivial group, which corresponds to a mixed state in the
stabilizer formalism. The algorithm works by checking the $q$-th column for
each stabilizer in the tableau. This is then decoded the same way we encoded
the Pauli matrices in the tableau algorithm ($00 \equiv I$, $01\equiv Z$, $10
\equiv X$, $11 \equiv Y$) and (in case of a non-zero value) stored into a
variable \texttt{dummy}. If we have two differing non-zero values for our Pauli
encoding, we know our qubit to be in an entangled state with at least one other
qubit, and we return $1$. If there are no other generators, we return $0$ and
if \texttt{dummy} is $0$, qubit $a$ is in a mixed state. This algorithm is
represented as pseudocode in \cref{alg:get_state_type} and as a flowchart in
\cref{fig:statetype-diag}.

\begin{algorithm}[H]
\caption{Determine State Type for Qubits}
\label{alg:get_state_type}
\begin{algorithmic}[1]
\REQUIRE qubit $a$, total number of qubits $N$, tableau $\mathcal{T}$
\ENSURE Returns 1, 0, or -1 based on the conditions
\STATE $\verb|dummy| \leftarrow 0$
\STATE $\verb|pauli| \leftarrow 0$
\FOR{$i \leftarrow 0$ \TO $(N - mix)$}
  \STATE $\verb|pauli| \leftarrow 2\cdot X_{ia} + Z_{ia}$
  \IF{$\texttt{pauli} \neq 0$ \AND $\texttt{dummy} = 0$}
        \STATE $\texttt{dummy} \leftarrow \texttt{pauli}$
        \ELSIF{$\texttt{pauli} \neq 0$ \AND $\texttt{pauli} \neq \texttt{dummy}$}
        \RETURN 1
    \ENDIF
\ENDFOR
\IF{$\texttt{dummy} \neq 0$}
    \RETURN 0
\ELSE
    \RETURN -1
\ENDIF
\end{algorithmic}
\end{algorithm}

\begin{figure}[H]
  \centering
  \begin{tikzpicture}[node distance=2cm, auto]

    % Nodes
  \node [startstop] (start) {Start};
  \node [block, below of=start] (init) {\verb|dummy| = 0};
  \node [process, below of=init] (loop) {Loop over stabilizers $i<(N-\verb|mix|)$};
  \node [decision, text width = 8em, below of=loop] (valnot0) {$2X_{ia} + Z_{ia} = \mathrm{val}
  \neq 0$?};
  \node [decision, below of=valnot0, yshift=-.5cm] (dummy0)
  {\verb|dummy| $=0$?};
  \node [decision, left of=valnot0, xshift=-.5cm] (loopend) {End of loop?};
  \node [block, left of=dummy0, xshift=-1.5cm] (setdummy) {\verb|dummy| $=2X_{ia} +
  Z_{ia}$};
  \node [decision, below of=dummy0, yshift=-.5cm] (dummypauli)
{\verb|dummy| $\neq\texttt{val}$?};
  \node [return, right of=dummypauli, xshift=1cm] (return1) {return 1};
  \node [decision, left of=setdummy, yshift=-4.5cm] (dummynot0)
  {\verb|dummy| $\neq 0$?};
  \node [return, right of=dummynot0, xshift=1cm] (ret0) {return 0};
  \node [return, below of=dummynot0, yshift=-.5cm] (ret-) {return $-1$};

    %\node (start) [startstop] {Start};
    %\node (init) [process, right of=start] {Init dummy, val = 0};
    %
    %\node (loop) [process, below of=init] {Loop: i = 0 to 2(N-mix)};
    %
    %\node (calc_val) [process, below of=loop] {Calculate val};
    %
    %\node (check_dummy_0) [decision, below of=calc_val] {val != 0 \& dummy == 0?};
    %\node (set_dummy) [process, right of=check_dummy_0, xshift=3cm] {Set dummy = val};
    %
    %\node (check_val_dummy) [decision, below of=check_dummy_0, yshift=-1cm] {val != 0 \& val != dummy?};
    %\node (return_1) [process, right of=check_val_dummy, xshift=3cm] {Return 1};
    %
    %\node (check_loop_end) [decision, below of=check_val_dummy, yshift=-1cm] {End of loop?};
    %
    %\node (check_dummy) [decision, below of=check_loop_end, yshift=-1cm] {dummy != 0?};
    %\node (return_0) [process, right of=check_dummy, xshift=3cm] {Return 0};
    %\node (return_neg1) [process, below of=check_dummy, yshift=-1cm] {Return -1};

    %\node (end) [startstop, below of=return_neg1, yshift=-1cm] {End};

    % Arrows
    \draw [arrow] (start) -- (init);
    \draw [arrow] (init) -- (loop);
    \draw [arrow] (loop) -- (valnot0);
    
    \draw [arrow] (valnot0) -- node[anchor=west] {yes} (dummy0);
    \draw [arrow] (valnot0) -- node[anchor=south] {no} (loopend);
    
    \draw [arrow] (dummy0) -- node[anchor=south] {yes} (setdummy);
    \draw [arrow] (dummy0) -- node[anchor=west] {no} (dummypauli);
    \draw [arrow-notip] (dummypauli) -| node[anchor=east] {no} (setdummy);
    \draw [arrow] (dummypauli) -- node[anchor=south] {yes} (return1);
    \draw [arrow] (setdummy) -- (loopend);
    
    \draw [arrow] (loopend) |- node[anchor=east] {no} (loop);
    \draw [arrow] (loopend) -| node[anchor=east] {yes} (dummynot0);
    
    \draw [arrow] (dummynot0) -- node[anchor=south] {yes} (ret0);
    \draw [arrow] (dummynot0) -- node[anchor=east] {no} (ret-);

\end{tikzpicture}

  \caption{flowchart for the \texttt{get\_state\_type} algorithm for qubit $a$.}
  \label{fig:statetype-diag}
\end{figure}

\subsection{\texttt{rowreduce}}
The next subroutine we expand the simulator with is \texttt{rowreduce}, which
is also vital to the \verb|ptrace| algorithm. Remember that the tableau
algorithm is based on the stabilizer generators and we already know that adding
two rows together, i.e. multiplying two generators, leaves the commutation
relations invariant. This means that we can perform row reduction to row
echelon form on our tableau without effect on the described state. However,
some subtleties need to be taken into account. In principle it is possible to
row reduce the entire tableau. But it turns out that we need only to reduce the
columns associated with one particular qubit, when epmloyed as subroutine to
\verb|ptrace|. Next, we need to pay attention to the fact that our stabilizer
tableau has dimensions $n\times 2n$. Ideally, \verb|rowreduce| should modify
our stabilizers in a way that there are at most one of $X$ or $Z$ stabilizers
for our qubit. A natural first step would then be to treat the respective $X$
and $Z$ column separately. This is where one needs to be careful. Although the $X$
and $Z$ stabilizers are in separate columns, they share the rows, e.g. in the
case where $X_{ia}=Z_{ia}=1\equiv Y_{ia}$, meaning that
a reduction of $X$ will influence the $Z$ column and vice versa. One way to
reconcile this is to reduce the $X$ stabilizers first, swapping the row
containing $X$ to the bottom if necessary, then doing $Z$ stabilizers. That way
we ensure that we do not introduce $X$ stabilizers when adding rows together
to get rid of $Z$ stabilizers. This already hints to the next subtlety we need
to take into account. A priori, one would probably perform gaussian
elimination to obtain an upper triangular form. The algorithm thereof is widely
studied and the plight of many computer science first year students. It would
thus be natural to assume that we want to have the reduced rows as first rows
of the matrix. Nevertheless, our case is different; since we exclusively call
\verb|rowreduce| to trace out a qubit, it will later prove convenient to have
the reduced rows on the \emph{bottom} of the tableau. This way, we can later
simply set \verb| N=N-1;|. We refer to \cref{sec:ptrace} for a more in-depth
explanation why this is done. The last subtlety we want to highlight is the
fact that with each modification of the stabilizer generators, we need an
appropriate modification of the \emph{anti}stabilizers to keep the commutation
relations of \cref{prop:comm-tab} intact. Although we do not technially need to
modify the antistabilizers for our purposes, it is still necessary if we want
to continue applying the tableau algorithm on the rowreduced tableau. To this
end, cf. \cref{prop:comm-tab-2}, where this statement is formalized and proven.
\begin{prop}\label{prop:comm-tab-2}
  Let $\mathcal{T}$ be a tableau with stabilizer and antistabilizer generators
  $S=\langle g_1, \ldots, g_n \rangle$ and $A=\langle h_1, \ldots, h_n \rangle$
  respectively, where the generators fulfil the commutation relations of
  \cref{prop:comm-tab}. Modifying $S$ by multiplication of two generators $g_i$
  and $g_j$ needs to be accompanied by an appropriate modification of $A$
\end{prop}
\begin{proof}
  We will conduct the proof by showing that each point in \cref{prop:comm-tab}
  still holds.
  \begin{enumerate}
    \item Since we merely multiplied generators, this holds by group theoretic
      arguments
    \item All of the antistabilizers did commute previously, therefore, their
      product commutes as well
    \item There are only two relations of relevance for this point
      \[
        \{h_i h_j, g_i\} \overset{\text{!}}{=} 0 \quad{\text{and}} \quad 
        \{h_j, g_i g_j\} \overset{\text{!}}{=} 0, 
      \]
      since all the other generators are left as they were.
      To show this anticommutation relation we employ the well-known identity
      \[ \{AB,C\} = A[B,C] + \{A,C\}B \]
      to obtain
      \begin{align*}
        \{h_i h_j, g_i \} &= h_i \underbrace{[h_j, g_i]}_{=0} +
        \underbrace{\{h_i, g_i\}}_{=0} h_j = 0 \qquad{\text{and}} \\
        \{h_j, g_i g_j \} &= \{g_i g_j, h_j\} = g_i \underbrace{[g_j, h_i]}_{=0} +
        \underbrace{\{g_i, h_i\}}_{=0} g_j = 0
      .\end{align*}
    \item As we started from a valid tableau, fulfilling the commutation
      relations, we need to verify this point only for one of the combinations,
      namely \[ [\tilde{g}_j, \tilde{h}_i] = [ g_i g_j, h_i h_j]. \]
      This is done with another commutator identiy,
      \[
        [AB,CD] = A[B,C]D + [A,C]BD + CA[B,D] + C[A,D]B.
      \]
      We thus have
      \begin{align*}
        [g_i g_j, h_i h_j] &=
          g_i \underbrace{[g_j, h_i]}_{=0} h_j + [g_i, h_i] g_j h_j + h_i g_i
          [g_j, h_j] + h_i \underbrace{[g_i, h_j]}_{=0} g_j \\
           &= (g_i h_i - h_i g_i)g_j h_j + h_i g_i (g_j h_j - h_j g_j) \\
           &= g_i h_i g_j h_j - h_i g_i g_j h_j + h_i g_i g_j h_j - h_i g_i h_j
           g_j \\
           &= \underbrace{\{g_i, h_i\}}_{=0} g_j h_j - h_i g_i
           \underbrace{\{g_j, h_j\}}_{=0} \\
           &= 0
      .\end{align*}
  \end{enumerate}
\end{proof}

With all the subtleties accounted for, we can begin to construct
\verb|rowreduce|. It should take a qubit position as input, and return
\verb|void|, since it merely modifies the matrix. We start by looping through
the $X$ stabilizers. The first time where $X_{ia}=1$, the row number $i$ is
stored in a variable \verb|first_x = i|. For each subsequent row with $X_{ka}=1$
we call \verb|rowsum(k,first_x)| and \verb|rowsum(first_x+1,k+1)|, where
\verb|h+1| is the associated antistabilizer to stabilizer \verb|h|. After
looping through the $X$ stabilizers, we move row \verb|first_x| to the bottom
if necessary. We then repeat the previous procedure with the $Z$ stabilizers,
also moving \verb|first_z| to the bottom if necessary. A pseudocode
representation of this algorithm is provided in \cref{alg:rowreduce}.

\begin{algorithm}[H]
\caption{Rowreduce stabilizers for qubit $a$}
\label{alg:rowreduce}
\begin{algorithmic}[1]
\REQUIRE qubit $a$, total number of qubits $N$, tableau $\mathcal{T}$
\STATE $\verb|first_x| \leftarrow -1$
\FOR{$i \leftarrow 0$ \TO $2N$ \textbf{step} 2}
\IF{$X_{ia} = 1$ \AND $\texttt{first\_x} = -1$}
    \STATE $\verb|first_x| \leftarrow i$
    \ELSIF{$X_{ia} = 1$ \AND $\texttt{first\_x} > -1$}
    \STATE \verb|rowsum(i, first_x)|
    \STATE \verb|rowsum(first_x+1, i+1)|
  \ENDIF
\ENDFOR
\IF{ $-1 < \texttt{first\_x} < 2N$}
  \FOR{$i \leftarrow$ \texttt{first\_x} \TO $2N$}
    \STATE \verb|rowswap(i, i+2)|
  \ENDFOR
\ENDIF
\STATE $\verb|first_z| \leftarrow -1$
\FOR{$i \leftarrow 0$ \TO $2N$ \textbf{step} 2}
\IF{$X_{ia} = 1$ \AND $\texttt{first\_z} = -1$}
    \STATE $\verb|first_z| \leftarrow i$
    \ELSIF{$X_{ia} = 1$ \AND $\texttt{first\_z} > -1$}
    \STATE \verb|rowsum(i, first_z)|
    \STATE \verb|rowsum(first_z+1, i+1)|
  \ENDIF
\ENDFOR
\IF{ $-1 < \texttt{first\_z} < 2N$ \OR \texttt{first\_x} = $-1$}
  \FOR{$i \leftarrow$ \texttt{first\_z} \TO $2N$}
    \STATE \verb|rowswap(i, i+2)|
  \ENDFOR
\ENDIF
%\STATE $\verb|first_z| \leftarrow -1$
%\FOR{$i \leftarrow 0$ \TO $2N$ \STEP 2}
%  \IF{$X_{ia} = 1$ \AND \verb|first_z| = $-1$}
%    \STATE $\verb|first_x| \gets i$
%  \ELSIF{$X_{ia} = 1$ \AND \verb|first_z| > $-1$}
%    \STATE \verb|rowsum(i, first_z)|
%    \STATE \verb|rowsum(first_z+1, i+1)|
%  \ENDIF
%\ENDFOR
%\IF{$-1 < \verb|first_z| < 2N$ \OR \verb|first_x| = $-1$}
%  \FOR{$i \gets \verb|first_z| \TO 2N$}
%    \STATE \verb|rowswap(i, i+2)|
%  \ENDFOR
%\ENDIF
\end{algorithmic}
\end{algorithm}

\begin{figure}[H]
  \centering
  \begin{tikzpicture}[node distance=2cm, auto]

    % Nodes
  \node [startstop] (start) {Start};
  \node [block, below of=start] (init) {\verb|first_x| = $-1$, \verb|first_z| =
  $-1$};
  \node [process, below of=init] (xloop) {Loop over stabilizers $g_i,\ i\leq N$};
  \node [decision, below of=xloop] (checkx) {$X_{ia} = 1$?};
  \node [decision, below of=checkx] (checkfirstx) {\verb|first_x| = $-1$?};
  \node [decision, left of=checkx, xshift=-.5cm] (loopend) {End of loop?};
  \node [process, text width = 10em, right of=checkfirstx, xshift=2.2cm] (rowsumx)
  {rowsum(i, first\_x)\\rowsum(first\_x+1, i+1)};
  \node [block, left of=checkfirstx, xshift=-1.5cm] (setfirstx) {\verb|first_x|
$=i$};
  \node [decision, left of=setfirstx, yshift=-3cm] (xswap)
  {$-1 < \verb|first| < 2N$?};
  \node [block, right of=xswap, xshift=2cm] (swap) {swap \verb|first_x| to
  bottom};
  \node [process, below of=xswap, yshift=-.5cm] (repeat) {repeat loop with $Z$};
   
    %\node (start) [startstop] {Start};
    %\node (init) [process, right of=start, xshift=1cm] {Init variables};
    %
    %\node (x_loop) [process, right of=init, xshift=1.5cm] {Loop over qubit $a$};
    %\node (check_first_x) [decision, below of=x_loop] {$X_{ia}=1$?};
    %\node (set_first_x) [process, right of=check_first_x, xshift=2.5cm] {Set first\_x, inc num\_x};

    %\node (check_other_x) [decision, below of=check_first_x, yshift=-2cm] {Other x stabilizer?};
    %\node (rowsum_x) [process, right of=check_other_x, xshift=2.5cm] {rowsum};

    %\node (cond_x_swap) [decision, below of=check_other_x, yshift=-1cm] {x swap needed?};
    %\node (rowswap_x) [process, right of=cond_x_swap, xshift=2.5cm] {rowswap};

    %\node (z_loop) [process, below of=cond_x_swap, yshift=-1cm] {Loop over z stabilizers};
    %\node (check_first_z) [decision, below of=z_loop] {First z stabilizer?};
    %\node (set_first_z) [process, right of=check_first_z, xshift=2.5cm] {Set first\_z, inc num\_z};

    %\node (check_other_z) [decision, below of=check_first_z, yshift=-1cm] {Other z stabilizer?};
    %\node (rowsum_z) [process, right of=check_other_z, xshift=2.5cm] {rowsum};

    %\node (cond_z_swap) [decision, below of=check_other_z, yshift=-1cm] {z swap needed?};
    %\node (rowswap_z) [process, right of=cond_z_swap, xshift=2.5cm] {rowswap};

    %\node (end) [startstop, below of=cond_z_swap, yshift=-1cm] {End};

    % Arrows
    \draw [arrow] (start) -- (init);
    \draw [arrow] (init) -- (xloop);
    \draw [arrow] (xloop) -- (checkx);
    
    \draw [arrow] (checkx) -- node[anchor=west] {yes} (checkfirstx);
    \draw [arrow] (checkx) -- node[anchor=south] {no} (loopend);
    
    \draw [arrow] (checkfirstx) -- node[anchor=south] {yes} (setfirstx);
    \draw [arrow] (checkfirstx) -- node[anchor=south] {no} (rowsumx);
    \draw [line,dashed,gray] (rowsumx) -- (loopend);
    \draw [arrow] (setfirstx) -- (loopend);
    
    \draw [arrow] (loopend) |- node[anchor=east] {no} (loop);
    \draw [arrow] (loopend) -| node[anchor=east] {yes} (xswap);
    
    \draw [arrow] (xswap) -- node[anchor=south] {yes} (swap);
    \draw [arrow] (xswap) -- node[anchor=east] {no} (repeat);
    \draw [arrow] (swap) |- (repeat);
    
    %\path [line] (start) -- (init);
    %\path [line] (init) -- (x_loop);
    %\path [line] (x_loop) -- (check_first_x);
    %\path [line] (check_first_x) -- node[anchor=south] {yes} (set_first_x);
    %\path [line] (check_first_x) |- node[anchor=east] {no} (x_loop);
    %\path [line] (check_other_x) -- node[anchor=south] {yes} (rowsum_x);
    %\path [line] (check_other_x) -- node[anchor=east] {no} (cond_x_swap);
    %\path [line] (cond_x_swap) -- node[anchor=south] {yes} (rowswap_x);
    %\path [line] (cond_x_swap) -- node[anchor=east] {no} (z_loop);

    %\path [line] (z_loop) -- (check_first_z);
    %\path [line] (check_first_z) -- node[anchor=south] {yes} (set_first_z);
    %\path [line] (check_first_z) -- node[anchor=east] {no} (check_other_z);
    %\path [line] (check_other_z) -- node[anchor=south] {yes} (rowsum_z);
    %\path [line] (check_other_z) -- node[anchor=east] {no} (cond_z_swap);
    %\path [line] (cond_z_swap) -- node[anchor=south] {yes} (rowswap_z);
    %\path [line] (cond_z_swap) -- node[anchor=east] {no} (end);

    %\path [line] (set_first_x) |- (check_other_x);
    %\path [line] (rowsum_x) |- (cond_x_swap);

    %\path [line] (set_first_z) |- (check_other_z);
    %\path [line] (rowsum_z) |- (cond_z_swap);

    %\path [line] (rowswap_x) |- (z_loop);
    %\path [line] (rowswap_z) -- (end);

\end{tikzpicture}

  \caption{flowchart for rowreduce algorithm}
  \label{fig:rowreduce-diag}
\end{figure}

\subsection{\texttt{ptrace}}\label{sec:ptrace}
We now combine the two previous subroutines to the \verb|ptrace| algorithm.
The algorithm traces out one qubit and modifies the remaining stabilizers
accordingly. For instance, tracing out a qubit in a product state will just
remove this qubit, since it doesn't correlate with any other qubit. If we do
have correlations in the form of entanglement, the algorithm will modify the
remaining stabilizers in a way that increases the mixedness. 
Consider the paradigmatic example of the two-qubit Bell state
\[
  \ket{\phi} = \frac{\ket{00} + \ket{11}}{\sqrt{2}} \quad{\text{with density
  matrix}} \quad \rho = \dyad{\phi}.
\]
After tracing out any of the two qubits we are left with a mixed state, namely
$$\rho_i = \frac{1}{2} \left[\dyad{0}_i + \dyad{1}_i\right].$$


\begin{figure}[h]
  \centering
  \begin{tikzpicture}[node distance=2cm]

    % Nodes
    \node (start) [startstop] {Start};
    \node (assertions) [process, below of=start] {Check assertions (q, N)};
    
    \node (init_helper) [process, below of=assertions] {Init helper\_tab, last\_stab, last\_anti};
    
    \node (move_qubit_loop) [process, below of=init_helper] {Loop: Move qubit q to the end};
    
    \node (update_mix) [process, below of=move_qubit_loop] {Update mix};
    
    \node (rowreduce) [process, below of=update_mix] {Call rowreduce(N-1)};
    
    \node (decrement_n) [process, below of=rowreduce] {Decrement N};
    
    \node (end) [startstop, below of=decrement_n] {End};

    % Arrows
    \draw [arrow] (start) -- (assertions);
    \draw [arrow] (assertions) -- (init_helper);
    \draw [arrow] (init_helper) -- (move_qubit_loop);
    \draw [arrow] (move_qubit_loop) -- (update_mix);
    \draw [arrow] (update_mix) -- (rowreduce);
    \draw [arrow] (rowreduce) -- (decrement_n);
    \draw [arrow] (decrement_n) -- (end);

\end{tikzpicture}

  \caption{flowchart for ptrace algorithm}
  \label{fig:ptrace-dig}
\end{figure}

\subsection{\texttt{is\_subgroup\_of}}

\subsection{\texttt{project\_or\_mix}}

\subsection{\texttt{force\_projection}}
