\chapter{Abstract}

One of the most intriguing ideas in quantum theory is the concept of
entanglement. It is well-famed in its role in quantum computation and quantum
information, as well as for its puzzling peculiarities, challenging the way we
think about nature. In general, entanglement emerges from the interaction of
quantum systems with their environment. In a controlled setting, such as an
experiment, entangling two systems is usually achieved by the application of
unitary interactions.

A key feature of these non-local correlations is the fact that measuring a
constituent of an entangled state erases part of the underlying entanglement
structure. As such, the growth of entanglement in the lab by the application of
unitaries can be stunted by locally measuring parts of the system. Circuits
characterized by the interplay between unitaries and local measurements are
called \emph{hybrid circuits}.

In the past decade, the study of these hybrid circuits have unveiled intriguing
patterns in the dynamics of entanglement. If measurements are sufficiently
frequent, one finds that the degree of entanglement between subsystems follows
an area law, while in the limit of rare measurements one has a volume law
scaling. Remarkably, one can find that there is a phase transition between
these two regimes, with a critical measurement rate. These transitions are
called \emph{measurement-induced entanglement phase transition}, or
entanglement transition for short.

Entanglement transitions have not only been theorized in hybrid circuits, but also in
setups, where the (discrete) time evolution is driven by projective measurements alone,
where pairwise and local measurements compete. A paradigmatic measurement-only
model with such a transition is the \emph{projective transverse-field Ising
model}, which features an area law to area law phase transition.

However, detecting entanglement transitions in an experimental setting proves
to be a difficult, seemingly sysiphean task, due to the \emph{postselection
problem}---also referred to as \emph{sampling problem}. It
states that due to the probabilistic nature of quantum measurements, one is
exponentially unlikely to obtain the same state twice when repeatedly applying
the identical protocol. This then obfuscates the precise nature of the state,
when probing for it, which is required to quantify the degree of entanglement.

Recently, some protocols have been proposed to circumvent the sampling problem.
In particular, \citeauthor{liCrossEntropyBenchmark2023} proposed a quantity
which acts as order parameter for the measurement-induced phase transition in
hybrid circuits, called the \emph{linear cross entropy}. Furthermore,
\citeauthor{garrattProbingPostmeasurementEntanglement2023} proposed a different
method, employing an upper bound from quantum information theory known as
Klein's inequality.

In this thesis we put these protocols to test in the projective
transverse-field Ising model.
It consists of four main chapters and is structured as follows
\begin{description}
  \item[\cref{ch:basics}] In this chapter we give an introduction to the
    concepts relevant throughout the thesis. We begin by outlining the
    stabilizer formalism, as it is a useful representation of quantum states
    and the foundation of the simulation algorithm employed for the numerical
    approaches in the thesis. We then provide an overview of the concept of entanglement
    transitions in general, before specifically introducing the projective
    transverse-field Ising model. This model serves as the primary subject of
    our studies. We then discuss the post-selection or sampling problem,
    motivating the rest of the thesis.
  \item[\cref{ch:lxe}] In this chapter we examine the quantity defined by
    \citeauthor{liCrossEntropyBenchmark2023} in the context of the projective
    transverse-field Ising model. We particularily motivate post-selection
    algorithms and compare our results to
    previous investigations of similar nature (cf. Ref.
    \cite{tikhanovskayaUniversalityCrossEntropy2023}), while also raising
    critiques.
  \item[\cref{ch:rel-ent}] In this chapter we investigate the approach of
    \citeauthor{garrattProbingPostmeasurementEntanglement2023} to detect the
    entanglement transition in the projective transverse-field Ising model. We
    first derive an expression for the cross- and relative entropy in the
    stabilizer formalism, before trying different numerical approaches to
    circumvent the post-selection problem. We finalize the chapter by outlining
    two regularization approaches to deal with appearing infinities.
  \item[\cref{ch:mixed}] In this chapter we introduce the simulation algorithm,
    which is based on the stabilizer formalism outlined in \cref{ch:basics}. We
    first give a brief summary of the existing algorithms. We then introduce
    the extensions necessary for the numerical simulations performed throughout
    the thesis, especially concerning mixed states and the different entropies
    derived in \cref{ch:rel-ent}.
\end{description}


%an 
%In the lab, entanglement emerges from the application of
%unitary interactions
%these non-local
%correlations challenge the way we think about nature. 
%
%On the other hand, one of the most fascinating phenomena in classical theory is phase
%transitions
%
%Quantum entanglement is the bedrock of quantum computation and
%quantum information, as the non-local correlations inherent to entanglement
%allow for a broad range of 
%
%In recent times however, different approaches have been suggested to circumvent the
%postselection problem. These approaches 
%
%\begin{itemize}
%  \item Entanglement transitions are cool
%  \item Entanglement is backbone of quantum computing
%  \item Something something quantum error correction
%  \item Entanglement exponentially hard to measure (requires exponentially many
%    copies of state)
%  \item Classical simulations predict phase transition in many hybrid random
%    circuits
%  \item PTIM \cite{langEntanglementTransitionProjective2020} measurement only
%    circuit, that maps to majorana fermions, the colored cluster model and
%    2-dimensional percolation
%  \item 2D percolation and the CCM has been widely studied, from this we
%    predict there to be the quantum phase transition from one area law to the
%    other
%%  \item \textcolor{red}{check das mit dem area law ding auf jeden fall nochmal,
%%    ich glaub da wird schon auch wert drauf gelegt}
%  \item No way to measure entanglement in experiment, since we can only measure
%    observables
%  \item Many attempts of probing the critical point of the phase transitions
%    have been made
%  \item Fisher's LXE: \cite{liCrossEntropyBenchmark2023}
%  \item Fisher's LXE for PTIM:
%    \cite{tikhanovskayaUniversalityCrossEntropy2023}
%  \item In-House attempts also exist:
%  \item The decoder in \cite{roserDecodingProjectiveTransverse2023} lower
%    bounds the critical point (decoding threshhold)
%  \item Although ubiquitous in all of quantum computation, entanglement plays an
%    important role in quantum error correction. 
%  \item Can we upper bound it?
%  \item In principle yes, klein's inequality (goes by \enquote{gibb's
%    inequality} and \enquote{information inequality} as well) upper bounds
%    von Neumann, and consequently entanglement entropy.
%  \item In \cite{garrattProbingPostmeasurementEntanglement2023} this is done
%    with hybrid circuits.
%\end{itemize}
