\chapter*{Conclusion and Outlook}

In this thesis we investigated different approaches to detect the
measurement-induced entanglement phase transition in the projective
transverse-field Ising model. In particular, the approaches employed measures
related to cross entropy: a linearized cross entropy, and an upper bound on
the entanglement entropy via Klein's inequality, which features the cross
entropy.

In \cref{ch:basics} we provided a broad overview of the core concepts
pertaining to the thesis. We introduced the stabilizer formalism and its
connection to classical simulations of quantum circuits. We then gave an
introduction to the theory of entanglement transitions, with the projective
transverse-field Ising model (PTIM) as the primary paradigm used throughout the rest
of the thesis. We then explained the sampling problem in greater detail.

In \cref{ch:lxe} we discussed the first approach, the linear cross entropy. We
gave an intuition behind the quantity and derived a method to efficiently
compute it in Clifford circuits. We then utilized this method to investigate
the behavior of the linear cross entropy in classical simulations of the
projective transverse-field Ising model. Furthermore, we showed that the linear
cross entropy factorizes elegantly into a product of two separate quantities
dependent only on one type of measurement. Finally, we highlighted its behavior
in noisy realizations of the PTIM by employing an error model with different
error types. 

In \cref{ch:rel-ent} we examine the approach based on bounding the entanglement
entropy from above. In particular, we employed Klein's inequality to provide an
upper bound with the cross entropy. We first derived the inequality in general,
before giving an expression thereof in the stabilizer formalism. We then
employed the derived expression in stabilizer simulations of the PTIM, exploring the
utility of different post-processing algorithms with the previously presented
noise protocol. We finalize the chapter by proposing two
different regularization Ans\"atze to deal with infinities and discussing their
utility as well.

It is clear that stabilizer simulations played an important part in the scope
of this thesis. We therefore dedicated \cref{ch:mixed} to outlining the
simulation algorithm, as well as the introduction of new functions and
subroutines added to the simulator, specifically relevant to the thesis. These
most notably include the addition of mixed states to the simulator, as well as
the adaptation of the previously existing functionalities.

\section*{Outlook}

While we exhaustively covered the linear cross entropy and different numerical
approaches, one could more closely investigate the subgroup condition of the
cross entropy. While the proof of the condition is rigorous and makes sense in
the context of information theory, single qubits in orthogonal product states
equally do not contribute to the entanglement entropy. One could thus try an
approach, where one is agnostic to the signs when comparing product states on
the same site with the cross entropy.

Furthermore, one could try other approaches, which are not explicitly discussed
in this thesis. For instance, one could employ other measures, such as the
entanglement asymmetry, or imagine entirely new algorithms, possibly by
combining the approaches of the upper bound and the linear cross entropy.

